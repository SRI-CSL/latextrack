% !TEX root = manual.tex
\chapter{Using LTC}

The intended use of LTC is running the base system as a server and have a plugin of the user's preferred LaTeX editor connect to it.  The provided Emacs \Code{ltc-mode} is an example of such a plugin and its use is described in Section~\ref{sec:emacs}.  We hope to add more plugins for other popular LaTeX editors in the future.

As a minimum implementation of a user interface to LTC, we also provide a Java application ``LTC Editor'' that uses the base system API but does not rely on running a separate LTC server.  Its use is explained in Section~\ref{sec:java}.

Before we address using these specific user interfaces, we discuss how to setup your repository for a LaTeX writing project with git (Section~\ref{sec:git-use}) or svn (Section~\ref{sec:svn-use}) and then the general usage of LTC in Section~\ref{sec:general-use}.

\section{Using a Git Repository} \label{sec:git-use}

For each writing project, LTC expects the history of the .tex files contained in a git repository.  As git is a distributed version control system, this repository is local to your machine.  If you need to exchange data with a collaborating author, you will push your repository or pull their repository and merge it with your local copy.

This section only covers a small subset of what git can do with respect to setting up a repository for your writing project.  Please refer to other git documentation about general git usage and how to further manage your writing project with git.

\subsection{Initializing a Local Repository}

Assuming that your current LaTeX source files (and other files) are located in a directory structure called \Code{\$PROJECT}.  To initialize the top-level directory for git perform the following commands.
\begin{CodeVerbatim}
$> cd $PROJECT
$> git init 
Initialized empty Git repository in $PROJECT
\end{CodeVerbatim}

Decide, what the final build products in your project will be.  These should be ignored by git so as not to complain every time you recompile your LaTeX project.  Let's assume your project will create a file called ``proposal.pdf,'' then create a file called \Code{.gitignore} in this directory that contains in each line the name of every build product.  In a bash shell, you can do the following.
\begin{CodeVerbatim}
$ cat > .gitignore <<EOF
> proposal.pdf
> EOF
\end{CodeVerbatim}

Then check the contents of the file:
\begin{CodeVerbatim}
$> less .gitignore
proposal.pdf
\end{CodeVerbatim}

If you decide on more build products (e.g., files called ``proposal-vol1.pdf'' and ``proposal-vol2.pdf'') in the future, simply edit the \Code{.gitignore} file to include these file names in new lines.  Make sure to do this before using a command such as \Code{git add .}, which would mark any existing build products for addition.

Checking the current status of git, the output should be similar to the following.

\begin{CodeVerbatim}
$> git status
# On branch master
#
# Initial commit
#
# Untracked files:
#   (use "git add <file>..." to include in what will be committed)
#
#	.gitignore
#	proposal.tex
nothing added to commit but untracked files present (use "git add" to track)
\end{CodeVerbatim}

Then, add the files already in your directory as well as the newly created file \Code{.gitignore} and commit the first version:

\begin{CodeVerbatim}
$> git add .
$> git commit -a -m "Initial commit of project"
[master (root-commit) dfbf239] Initial commit of project
 2 files changed, 7 insertions(+), 0 deletions(-)
 create mode 100644 .gitignore
 create mode 100644 proposal.tex
\end{CodeVerbatim}

The option \Code{-m} stands for a brief message that you would like to attach to your commit. Note that you have to give some sort of message for every commit you make. Do \textit{not} try to skip the message part. Moreover, having meaningful one-line message for commit is always useful as other and you yourself can refer to later on to see what changes you made and why.

\subsection{Uploading Your Initial Repository}\label{sec:upload-git}

To share your local git repository, you can clone it to a shared file system or to a file server that each collaborator can access via \Code{ssh}.  Another option is that your system administrators provide you with a central git repository server.  Contact them for details on how to upload your git repository there.

In the following commands, we are giving the remote repository the name \Code{project.git} but you can choose whatever you want.  The ending \Code{.git} is somewhat standard, though, so we advise to keep it.  

To clone to a \textit{shared file system}, you will want to find a suitable directory \Code{\$SHARED\_PATH} where you want to create the shared repository. The following command is issued from inside your initial repository.

\begin{CodeVerbatim}
$> git clone --bare . $SHARED_PATH/project.git
\end{CodeVerbatim}

To clone to a \textit{file server accessible via ssh and scp}, you would do the following from the top-level directory of your initial repository. The remote repository should be located under \Code{\$REMOTE\_PATH} on the server.  We assume that you have performed at least one commit since initializing the git repository.
\begin{CodeVerbatim}
$> git status
# On branch master
nothing to commit, working directory clean
$> cd ..
$> git clone --bare $PROJECT project.git
Cloning into bare repository 'project.git'...
done.
$ touch project.git/git-daemon-export-ok
$ scp -rq project.git username@server:$REMOTE_PATH
\end{CodeVerbatim}

In either case, make sure that file permissions allow collaborators to access and read the files on the server.

Next, you will want to add a short name for the newly designated shared location called ``origin'' so that the push and pull command below work as if you had cloned the repository from someone else.  If you used a git server, your system administrators can tell you how to configure your original repository to include the new remote location, or you can clone the remote repository anew as shown in the next section.
\begin{CodeVerbatim}
$> git remote add origin $SHARED_PATH/project.git
\end{CodeVerbatim}
or
\begin{CodeVerbatim}
$> git remote add origin username@server:$REMOTE_PATH/project.git
\end{CodeVerbatim}

\subsection{Cloning from a Remote Repository}

To start a git repository from an existing one (either on a shared file server or a central repository), you want to clone it.  You need to know the remote location in terms of user name, server address and path to the git repository.  Your system administrator can tell you these in the form of \Code{username@server:path-to-git-repos/project.git}, or, if you used a shared file system as in Section~\ref{sec:upload-git}, you simply use \Code{\$SHARED\_PATH/project.git} instead of the address above. 

Assuming that your local copy should be located in a directory \Code{my\_project}, you would need to execute the following command from the parent directory of \Code{my\_project}.  Feel free to call your new working directory something else by substituting the last argument.
\begin{CodeVerbatim}
$> git clone username@server:path-to-git-repos/project.git my_project
Cloning into my_project...
done.
$> cd my_project
$> git remote -v
\end{CodeVerbatim}

\subsection{Push and Pull}

To exchange data with the central repository, you typically push and pull.  In the simplest case, the following should work (if this is the original and not a cloned repository, you must have added the new short name ``origin'' via the \Code{git remote add origin} command above).
\begin{CodeVerbatim}
$> git push origin master
$> git pull origin master
\end{CodeVerbatim}

If git complains about uncommitted changes and that the working copy is not clean, you may have to commit or stash changes before these commands can run successfully.

Please refer to the many online resources to learn more about git, or ask you system administrator.

\section{Using a Subversion Repository} \label{sec:svn-use}

[TODO: write this section]

\section{General Usage} \label{sec:general-use}

In this section we describe abstractly how one would use LTC in a pattern of a work cycle. %TODO: more

[TODO: graph operations; filtering]

\section{Using Emacs} \label{sec:emacs}

In this section, we explain how to use Emacs as the editor of the .tex-file with LTC.

\begin{center}
\fcolorbox{red}{yellow}{
\begin{minipage}[t]{.9\textwidth}
  \textbf{CAVEAT:} due to current bugs, it is best to use \Code{ltc-mode} defensively without editing the file in Emacs while LTC is turned on.  It is safe to view changes in the file and use different filtering settings.  If the text becomes scrambled, try to exit \Code{ltc-mode} by toggling it (e.g., using command \Code{M-x ltc-mode}) or quitting Emacs altogether before saving the file.  As long as you do not save while in \Code{ltc-mode} nothing bad should happen.
\end{minipage}}
\end{center}

\subsection{Starting the LTC Server}

Before you can use \Code{ltc-mode} in Emacs, you must start the LTC server first.  To do so, you call Java with the JAR-file that you downloaded. 

\begin{CodeVerbatim}
$> java -jar $LTC/LTC-${project.version}.jar
 | CONFIG: 	Default logging configuration complete
 | CONFIG: 	Logging configured to level CONFIG
 | CONFIG: 	git version: X.Y.Z
--log4j: [main] INFO  org.apache.xmlrpc.webserver.XmlRpcServlet - init
 | INFO: 	Started RPC server on port 7777.
\end{CodeVerbatim}

If you need to customize LTC, for example to change the port number, start the base system with option \Code{-p <PORT>}:

\begin{CodeVerbatim}
$> java -jar $LTC/LTC-${project.version}.jar -p 5555
[...]
 | INFO: 	Started RPC server on port 5555.
\end{CodeVerbatim}

The log messages from the LTC server are also saved in file \Code{~/.LTC.log}.  This log file is created every time that the server starts, so if you need the contents, please make a copy before starting the server anew.

\subsection{Entering and Exiting \texttt{ltc-mode}}

\begin{figure}[t]
\centering
\mygraphics{height=.37\textheight}{figures/emacs-latex-mode}
\hspace{2ex}
\begin{tikzpicture}
  \node [single arrow,draw] {};
\end{tikzpicture}
\hspace{2ex}
\mygraphics{height=.37\textheight}{figures/emacs-ltc-started}
\caption{Starting \texttt{ltc-mode} in Emacs} \label{fig:emacs-ltc-started}
\end{figure}

Once your Emacs has a .tex-file loaded in plain \Code{latex-mode} as seen on the left in Figure~\ref{fig:emacs-ltc-started}, you can invoke the minor-mode \Code{ltc-mode} in different ways.  Usually, the mode launching command \Code{M-x ltc-mode} is available.  If Emacs has its own window and supports context menus, you may be able to right-click on the word ``LaTeX'' in the mode line at the bottom of Emacs. Then, a list of minor modes appears from which you can select ``LTC Mode.''  Once the mode starts successfully, a number of status messages, a smaller window at the bottom of the text that contains LTC information, and changes in the text marked up by color and style appear.  Emacs then looks similar to the right in Figure~\ref{fig:emacs-ltc-started}.

The mode can be toggled, i.e., the same command starts and stops \Code{ltc-mode}.

Once the mode is started, a new menu ``LTC'' should appear in Emacs.  This can be part of Emacs' main menu or included in the mode line.  From there, you can interact with the LTC system as shown in the following examples.

\subsubsection{The ``LTC info'' Buffer}

The new, smaller window at the bottom of the frame is called ``LTC info'' and contains a table resembling the commit graph of the current file.  Newer versions of the file are at the top of the list.  An asterisk at the beginning  indicated that this version is taken into account when the history of changes is calculated.  Conversely, if the asterisk is missing and the whole row is shown in gray color, then this version is currently skipped when calculating changes.  The next column contains the SHA-1 key (abbreviated) and the third column the date  of the corresponding git revision.  The author colored in his or her key and finally the commit message follow.  The first row of the commit table contains the current author.

\subsubsection{Updating the Buffer}

An important command is \Code{M-x ltc-update} (currently bound to Ctrl-L U but this will change once we adhere better to Emacs conventions) or the menu item LTC $>$ Update buffer.  This will update the  view of the file based on current filtering and other settings.  Use this command when the ``LTC info'' buffer ended up in a different frame or tab, or when you make changes such as a git commit from the command line.

\subsection{Filtering Changes}

\begin{figure}[t]
\centering
\mygraphics{height=.37\textheight}{figures/emacs-hide-preamble}
\caption{Hiding changes in the preamble using the ``LTC'' menu} \label{fig:emacs-hide-preamble}
\end{figure}

To show and hide certain changes, simply choose from the menu LTC $>$ Show/Hide $>$ ... the appropriate entry.  Figure~\ref{fig:emacs-hide-preamble} shows for example hiding changes in the preamble via the context menu.  %If the menu is not available, the Emacs commands \Code{M-x } have the same effect.
After selecting any of the menu items to show or hide certain changes, the contents in the main Emacs window is immediately updated to reflect the filtering settings.

To change the default limitation of the file's history that is taken into account for viewing changes, you can do so by specifying a set of authors to limit to, and a revision number or date to indicate how far back in time you want to go.  These functions are available via the menu LTC $>$ Limit by $>$ ...  Each of these actions require additional input, therefore the menu item ends with ellipsis. 

To define a set of authors to limit the history to, select the menu item or use command \Code{M-x ltc-limit-authors}. This will prompt you to input the names of authors in the mini-buffer.  You may use auto-completion with the TAB key.  End the input of names by providing an empty name.  If the first name is empty, this will reset to taking all known authors into account (i.e., not limiting by authors anymore).

Similarly, you can define how far back the file history is taken into account.  Use commands \Code{M-x ltc-limit-rev} 
and \Code{M-x ltc-limit-date} respectively.  Again, user input is needed via the mini-buffer.  Use auto completion with the TAB key in order to automatically extend to known SHA-1 keys or revision dates.  To reset the limit to the default, simply enter an empty value when prompted.

\subsection{Author Color Keys}

To customize the color used to designate changes in the marked up text, click on the author name in the ``LTC info'' buffer. [TODO: MORE NEEDED]

\subsection{Emacs Help}

[TODO]

\subsection{Table of Commands}

[TODO]
\section{Using the ``LTC Editor''} \label{sec:java}

The LTC Editor is a Java application that allows to use LTC without a separate LTC server.  It may also serve as a reference implementation showing how to use our API.  Chapter~\ref{ch:plugins} contains more details on writing your own editor plugin using our LTC API.


