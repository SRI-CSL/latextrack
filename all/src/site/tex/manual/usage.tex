% !TEX root = manual.tex
\chapter{Using LTC}

The intended use of LTC is running the base system as a server and have a plugin of the user's preferred LaTeX editor connect to it.  The provided Emacs \Code{ltc-mode} is an example of such a plugin and its use is described in Section~\ref{sec:emacs}.  We hope to add more plugins for other popular LaTeX editors in the future.

As a minimum implementation of a user interface to LTC, we also provide a Java application ``LTC Editor'' that uses the base system API but does not rely on running a separate LTC server.  Its use is explained in Section~\ref{sec:java}.

Before we address using these specific user interfaces, we discuss how to setup your repository for a LaTeX writing project with git (Section~\ref{sec:git-use}) or svn (Section~\ref{sec:svn-use}) and then the general usage of LTC in Section~\ref{sec:general-use}.

% !TEX root = manual.tex
\section{Using a Git Repository} \label{sec:git-use}

For each writing project, LTC expects the history of the .tex files managed by a version control system, for example contained in a git repository.  As git is a distributed version control system, this repository is local to your machine.  If you need to exchange data with a collaborating author, you will push your repository or pull their repository and merge it with your local copy.

This section only covers a small subset of what git can do with respect to setting up a repository for your writing project.  Please refer to other git documentation about general git usage and how to further manage your writing project with git.  Also note our suggestions for general, one-time git configuration in Section~\ref{sec:git-install}.

\subsection{Initializing a Local Repository}

Assuming that your current LaTeX source files (and other files) are located in a directory structure called \Code{\$PROJECT}.  To initialize the top-level directory for git perform the following commands.
\begin{CodeVerbatim}
$> cd $PROJECT/
$> git init 
Initialized empty Git repository in $PROJECT
\end{CodeVerbatim}

Decide, what the final build products in your project will be.  These should be ignored by git so as not to complain every time you recompile your LaTeX project.  Let's assume your project will create a file called ``proposal.pdf,'' then create a file called \Code{.gitignore} in this directory that contains in each line the name of every build product.  In a bash shell, you can do the following.
\begin{CodeVerbatim}
$> cat > .gitignore <<EOF
> proposal.pdf
> EOF
\end{CodeVerbatim}

Then check the contents of the file:
\begin{CodeVerbatim}
$> less .gitignore
proposal.pdf
\end{CodeVerbatim}

If you decide on more build products (e.g., files called ``proposal-vol1.pdf'' and ``proposal-vol2.pdf'') in the future, simply edit the \Code{.gitignore} file to include these file names in new lines.  Make sure to do this before using a command such as \Code{git add .}, which would mark any existing build products for addition.

Checking the current status of git, the output should be similar to the following.

\begin{CodeVerbatim}
$> git status
# On branch master
#
# Initial commit
#
# Untracked files:
#   (use "git add <file>..." to include in what will be committed)
#
#	.gitignore
#	proposal.tex
nothing added to commit but untracked files present (use "git add" to track)
\end{CodeVerbatim}

Then, add the files already in your directory as well as the newly created file \Code{.gitignore} and commit the first version, for example this way:
\begin{CodeVerbatim}
$> git add .
$> git commit -a -m "initial commit of project"
[master (root-commit) dfbf239] initial commit of project
 2 files changed, 7 insertions(+), 0 deletions(-)
 create mode 100644 .gitignore
 create mode 100644 proposal.tex
\end{CodeVerbatim}

The option \Code{-m} stands for a brief message that you would like to attach to your commit. Note that you have to give some sort of message for every commit you make. Do \textit{not} try to skip the message part. Moreover, having meaningful one-line message for commit is always useful as other and you yourself can refer to later on to see what changes you made and why.

\subsection{Uploading Your Initial Repository}\label{sec:upload-git}

To share your local git repository, you can clone it to a shared file system or to a file server that each collaborator can access via \Code{ssh}.  Another option is that your system administrators provide you with a central git repository server.  Contact them for details on how to upload your git repository there.

In the following commands, we are giving the remote repository the name \Code{project.git} but you can choose whatever you want.  The ending \Code{.git} is somewhat standard, though, so we advise to keep it.  

To clone to a \textit{shared file system}, you will want to find a suitable directory \Code{\$SHARED\_PATH} where you want to create the shared repository. The following command is issued from inside your initial repository.

\begin{CodeVerbatim}
$> cd $PROJECT/
$> git clone --bare . $SHARED_PATH/project.git
\end{CodeVerbatim}

To clone to a \textit{file server accessible via ssh and scp}, you would do the following from the top-level directory of your initial repository. The remote repository should be located under \Code{\$REMOTE\_PATH} on the server.  We assume that you have performed at least one commit since initializing the git repository.
\begin{CodeVerbatim}
$> cd $PROJECT/..
$> git clone --bare $PROJECT project.git
Cloning into bare repository 'project.git'...
done.
$ touch project.git/git-daemon-export-ok
$ scp -rq project.git username@server:$REMOTE_PATH
\end{CodeVerbatim}

In either case, make sure that file permissions allow collaborators to access and read the files on the server.

Next, you will want to add a short name for the newly designated shared location called ``origin'' so that the push and pull command below work as if you had cloned the repository from someone else.  If you used a git server, your system administrators can tell you how to configure your original repository to include the new remote location, or you can clone the remote repository anew as shown in the next section.
\begin{CodeVerbatim}
$> git remote add origin $SHARED_PATH/project.git
\end{CodeVerbatim}
or
\begin{CodeVerbatim}
$> git remote add origin username@server:$REMOTE_PATH/project.git
\end{CodeVerbatim}

Finally, your system administrator may already have a server for git repositories set up that you and your collaborators can use.  Refer to their instructions on how to upload or create an initial repository.

\subsection{Cloning from a Remote Repository}

To start a git repository from an existing one (either on a shared file server or a central repository), you want to clone it.  You need to know the remote location in terms of user name, server address and path to the git repository.  Your system administrator can tell you these in the form of \Code{username@server:path-to-git-repos/project.git}, or, if you used a shared file system as in Section~\ref{sec:upload-git}, you simply use \Code{\$SHARED\_PATH/project.git} instead of the address above. 

Assuming that your local copy should be located in a directory \Code{my\_project}, you would need to execute the following command from the parent directory of \Code{my\_project}.  Feel free to call your new working directory something else by substituting the last argument.
\begin{CodeVerbatim}
$> git clone username@server:path-to-git-repos/project.git my_project
Cloning into my_project...
done.
$> cd my_project
$> git remote -v
\end{CodeVerbatim}

\subsection{Push and Pull}

To exchange data with the central repository, you typically push and pull.  In the simplest case, the following should work (if this is the original and not a cloned repository, you must have added the new short name ``origin'' via the \Code{git remote add origin} command above).
\begin{CodeVerbatim}
$> git push origin master
$> git pull origin master
\end{CodeVerbatim}

If git complains about uncommitted changes and that the working copy is not clean, you may have to commit or stash changes before these commands can run successfully.

Please refer to the many online resources to learn more about git, or ask you system administrator.

% !TEX root = manual.tex
\section{Using a Subversion Repository} \label{sec:svn-use}

For each writing project, LTC expects the history of the .tex files managed by a version control system, for example contained in a svn repository.  As svn is a centralized version control system, the repository is typically in a remote location.  To use LTC meaningfully, it has to download the different versions of the file history so you will need constant connectivity with the server.  If you need to exchange data with a collaborating author, you will update from and commit to your remote repository, which also requires online access.

This section only covers a small subset of what svn can do with respect to setting up a repository for your writing project.  Please refer to other svn documentation about general svn usage and how to further manage your writing project with svn.  Also note our suggestions for general, one-time svn configuration in Section~\ref{sec:svn-install}.

\subsection{Initializing a Repository}

To create a working copy of an existing svn repository your system administrator will tell you the URL where the repository is hosted.  Then, you will \textit{check out} a working copy in a directory, say \Code{\$PROJECT} with that URL, which we call \Code{\$REPOSITORY\_URL}.  From the directory where you want \Code{\$PROJECT} to reside, call:
\begin{CodeVerbatim}
$> svn checkout $REPOSITORY_URL $PROJECT
\end{CodeVerbatim}

If this a new writing project, you may want to perform some initializations.  For example, decide what the final build products in your project will be.  These should be ignored by svn so as not to complain every time you recompile your LaTeX project.  Let's assume your project will create a file called ``proposal.pdf,'' then perform the following.  First, we check whether there are already files ignored.  Then, we will set a property to ignore ``proposal.pdf'' using a few bash commands.  If you are running a different shell, you may have to adjust these commands.
\begin{CodeVerbatim}[commandchars=\|\{\}]
$> cd $PROJECT/
$> svn propget svn:ignore .
$> svn propedit svn:ignore .  # will open temporary editor in your terminal, \ 
                                where you type |textbf{proposal.pdf}, save and exit
Set new value for property 'svn:ignore' on '.'
\end{CodeVerbatim}

Setting or updating a property puts a modification flag on the current directory \Code{.}, which you will have to commit to the repository at the next opportunity for others to obtain this setting.  Also, check that the property is now active in your working copy.
\begin{CodeVerbatim}
$> svn status
 M      .
$> svn commit -m "ignoring build product proposal.pdf"
Sending        .
Committed revision XXX.
$> svn propget svn:ignore .
proposal.pdf
\end{CodeVerbatim}

\subsection{Other Typical Subversion Commands}

If you have a new \Code{FILE.tex} file to add to the repository, do
\begin{CodeVerbatim}
$> $ svn st
?       FILE.tex
$> svn add FILE.tex
A         FILE.tex
$> svn commit -m "adding first version of FILE.tex"
Adding         FILE.tex
Transmitting file data ...
Committed revision XXX.
\end{CodeVerbatim}

When editing a \Code{FILE.tex} file and saving it, it will have the modification flag set, which you can check using the \Code{status} command.  It is also a good idea to update your working copy before you start editing a file, in case others have committed any changes.
\begin{CodeVerbatim}
$> svn update
[...]  # any potential updates
At revision XXX.
$> svn status
M       FILE.tex
$> svn commit -m "<message about recent edits in FILE.tex>"
Sending        FILE.tex
Transmitting file data ...
Committed revision XXX.
\end{CodeVerbatim}

If you decide on more build products (e.g., files called ``proposal-vol1.pdf'' and ``proposal-vol2.pdf'') in the future, call \Code{svn propedit svn:ignore .} to edit the property and commit the changes to the svn repository.  Make sure to do this before using a command such as \Code{svn add *}, which would mark any existing build products for addition.

With a centralized repository, it is even more important to coordinate writing and editing activities among collaborators.  Many say ``commit early, commit often'' and also make it a habit to update your working copy regularly and before beginning work.  LaTeX is very well suited to be managed under version control as you can split the writing document into various files and then assign writing tasks on a one-author-per-file-at-a-time to avoid merge conflicts.

% !TEX root = manual.tex
\section{General Usage} \label{sec:general-use}
\newcommand{\generalscale}{0.9}  % scale for figures in the general section

In this section we describe abstractly how one would use LTC as a pattern of a work cycle. See the tutorials for more concrete examples.  Also, the Sections~\ref{sec:emacs} and \ref{sec:java} below contain more details on the specific user interface of interest, namely Emacs and the LTC Editor, respectively.

Typically, more than one author collaborate on a writing project that is kept under version control but it might be a good practice to put all your work under version control.  Especially git is a well suited version control system to run locally on your computer and keeping track of your own changes if you are just interested in how a .tex file evolves over time.

We are assuming that the .tex file or files of interest are kept under version control so as to obtain a history of significant changes that have been made in the past.  Significant changes are usually made through a ``commit'' action to the version control system.  This is in contrast to merely saving edits to the file on the local file system.  Such an operation can be done many more times just to preserve your current work in case of a problem with the editor or computer.

\begin{figure}[t]
\centering
  \subfloat{
    \mygraphics{width=0.48\linewidth}{figures/work-cycle}}
  \hfill %space{1em}
  \subfloat{
    \mygraphics{width=0.48\linewidth}{figures/work-cycle-with-LTC}}
\caption{A typical work cycle for a version controlled file and when using LaTeX Track Changes} \label{fig:work-cycle}
\label{fig:editor-condense-before-after}
\end{figure}

See Figure~\ref{fig:work-cycle} on the left hand side for a diagram that shows a typical work cycle for a version controlled file from the perspective of one author.  Often a user starts working by downloading changes that others have done---this step may be omitted if only one author is working with the revision control system, thus the action is drawn with dashed lines.  Then, an author may edit and save the file.  Finally, when significant changes have been made, it is often time to commit those and possibly upload them to a server where other authors can update from.

Now look at the right hand side of Figure~\ref{fig:work-cycle}; here we added a state for tracking changes.  The user typically switches from editing and saving into tracking changes.  In this mode, one can still edit and save the file.  And also perform version control commands such as downloading and uploading changes.  While in track changes mode, the .tex file of interest is marked up with information about changes in past versions of the file, so the text looks busier and can be longer when displaying deletions.  Thus, most authors will want to switch in and out of tracking changes in order to work at times with only the latest version of the file to avoid being overwhelmed by the information shown.

\subsection{Filtering What is Shown}

show/hide

heuristic of ``small'' changes

\subsection{History of a File}

git -- directed, acyclic graph with branches
svn -- often sequential history, so only a straight line

LTC currently chooses one path from the first to the latest version of the file, traversing branches in the order of the most recent commit of the last commits before merge.  In the future: user can select

condensing authors



\section{Using \texttt{ltc-mode}}

\section{Using the ``LTC Editor''} \label{sec:java}

The LTC Editor is a Java application that allows to use LTC without a separate LTC server.  It may also serve as a reference implementation showing how to use our API.  Chapter~\ref{ch:plugins} contains more details on writing your own editor plugin using our LTC API.


