\chapter{Tutorial} \label{ch:tutorial}

This tutorial uses an example git repository called ``independence.bundle,'' which can be downloaded from the \hlink{\baseurl download.html}{../../download.html}{LTC web site.}

\section{Creating an Example Git Repository}

First, save the bundled repository into a directory of your choice.  We call this directory \Code{\$TUTORIAL}.  Then, clone from this bundle to obtain a valid git working tree.
\begin{CodeVerbatim}
$> cd $TUTORIAL
$> git clone independence.bundle independence
Cloning into independence...
$> cd independence/
$> git status
# On branch master
nothing to commit (working directory clean)
$> git log --oneline
d3f904c sixth version
203e0ce fifth version
36eeab0 fourth version
fa2be39 third version
bac2f51 second version
d6d1cf8 first version
\end{CodeVerbatim}

Now we impersonate John Adams to work on this writing project for the Declaration of Independence.

\begin{CodeVerbatim}
$> git config --add user.name "John Adams"
$> git config --add user.email "adams@usa.gov"
$> $ git config --list | grep dams
user.name=John Adams
user.email=adams@usa.gov
\end{CodeVerbatim}

Another way to investigate the current git repository are graphical tools such as gitx under Mac OS X.  Figure~\ref{fig:gitx-screen} for using gitx on the just created repository.
\begin{figure}[t]
\centering
\mygraphics{width=\textwidth,height=.5\textheight,keepaspectratio}{figures/gitx-screen}
\caption{Investigating git repository with a graphical tool} \label{fig:gitx-screen}
\end{figure}
The other point to note here is the way that gitx displays the changes in the file \Code{independence.tex} when using the graphical git interface.  It shows the lines in the file that have changed (much like a standard Unix diff would) -- however, when looking at changes in LaTeX source code, the granularity of the line-based difference is much too coarse.  An author would most likely only care about the change in words of line 15 or even characters such as removing the mistaken `d' in the word ``Roger'' on line seven.

\section{Running the LTC Editor}

Now that we are comfortable with the git repository, let us start the LTC Editor to interact with LTC and track the changes of the file.  Assume you have saved the jar file as \Code{\$LTC/LTC-\version.jar}, we can look at the command line options of the editor:
\begin{CodeVerbatim}[commandchars=\\\{\}]
$> java -cp $LTC/LTC-\version.jar com.sri.ltc.editor.LTCEditor -h
usage: java -cp ... com.sri.ltc.editor.LTCEditor [options...] [FILE] 
with:
 FILE     : load given file to track changes
 -h       : display usage and exit
 -l LEVEL : set console log level
\end{CodeVerbatim}

To open our tutorial file when starting the editor, execute the following command.  This will open the editor as a window similar to the screen shot in Figure~\ref{fig:editor-open}.
\begin{CodeVerbatim}[samepage=true,commandchars=\\\{\}]
$> java -cp $LTC/LTC-\version.jar \\
   com.sri.ltc.editor.LTCEditor $TUTORIAL/independence/independence.tex
\end{CodeVerbatim}
\begin{figure}[t]
\centering
\mygraphics{width=\textwidth,height=.5\textheight,keepaspectratio}{figures/editor-open}
\caption{Initial opening of tutorial file in LTC Editor} \label{fig:editor-open}
\end{figure}
In this figure, we see a panel at the bottom-right that resembles the upper part of the gitx graphical interface to git.  There, we display the history of the .tex file of interest under git.  We can also see what git currently perceives as the current user -- John Adams because we had overridden the git settings in this repository.  (If you had another local git repository on your file system, the user information there is not changed.)

\section{Showing and Hiding Certain Changes}

The bottom-left panel of the editor allows us to customize the way LTC displays the changes of the file.  Section~\ref{sec:general-use} contains all the details of how LTC displays the changes including limiting the file history and filtering.  In this tutorial, we will just use some of the options and see their effect.

First, notice the list of known authors (as they are stored in the git repository) and the colors assigned to each of them.  Now perform a double-click on the colored square next to Roger Sherman to open a dialog and  choose a dark color such as brown (you will want something with contrast to the white background).  Notice how the text in the editor panel on the top changes color for those parts that are attributed to Roger Sherman's edits.

Next, focus on the typographical error in the command ``\textbackslash maketitle'' in line 11 and the beginning of the first paragraph in line thirteen.  If you uncheck the box for ``small'' changes and then the box for deletions, notice how the text in the editor panel is rendered.
\begin{figure}[t]
\centering
\mygraphics{keepaspectratio}{figures/editor-filter-small}
\caption{Effect of hiding ``small'' changes and deletions} \label{fig:editor-filter-small}
\end{figure}
Figure~\ref{fig:editor-filter-small} shows that ``\textbackslash maketitle'' is no longer marked up, and the deletion of ``If'' at the beginning of the paragraph is now omitted.

\section{Understand the Commit Graph}

Now draw your attention back to the graph with the history of the current file under git (located in the bottom-right panel).  In our example repository, this graph is just a line as the authors committed their versions in sequential order.

%TODO: write

\section{Limit History}

\section{Editing and Saving}

\section{Committing Version to Git}

\section{Collaborating}

\subsection{Resolving Merge Conflicts}