% !TEX root = manual.tex
\chapter{Tutorials} \label{ch:tutorials}

This chapter contains a number of tutorials adjusted to the user's preference of text editor and version control system.  The following diagram allows to easily identify the best fit for your situation and names the specific tutorial sections.  Whether you are using git or svn as a version control system, you will want to visit the respective section to setup your example repository first (Sec.~\ref{sec:example-git} or Sec.~\ref{sec:example-svn}).  Then move on to the respective section for the editor you will be using, Emacs or the bundled Java application LTC Editor (sections \ref{sec:tutorial-git-emacs}, \ref{sec:tutorial-git}, \ref{sec:tutorial-svn-emacs}, or \ref{sec:tutorial-svn}).

\begin{tikzpicture}[->,>=stealth',shorten >=1pt,font=\sffamily\small,auto]
	\tikzset{CircleStyle/.style={circle,draw=black,minimum size=0.2cm}}
	\tikzset{VertexStyle/.style={rectangle,rounded corners,draw=black,minimum size=0.2cm,text width=2.2cm}}
  	\tikzset{LabelStyle/.style={draw=none,fill=white,font=\sffamily\footnotesize,midway}}

	% first row
	\node[CircleStyle,initial above,initial text=,initial distance=0.3cm] (INIT) {};
	
	% second row  
	\node[VertexStyle,text width=3.5cm] (example-git) [below left=0.6cm of INIT,xshift=-0.8cm] 
	 {Sec.~\ref{sec:example-git}\\ \nolinknameref{sec:example-git}};
	\node[VertexStyle,text width=3.5cm] (example-svn) [below right=0.6cm of INIT,xshift=0.8cm]
	 {Sec.~\ref{sec:example-svn}\\ \nolinknameref{sec:example-svn}};
	
	% third row
	\node[CircleStyle] (git-EDITOR) [below=0.3cm of example-git] {};
	\node[CircleStyle] (svn-EDITOR) [below=0.3cm of example-svn] {};
	  
	% fourth row
	\node[VertexStyle] (git-emacs)  [below left=0.5cm of git-EDITOR]
	 {Sec.~\ref{sec:tutorial-git-emacs}\\ \nolinknameref{sec:tutorial-git-emacs}};
	\node[VertexStyle] (git-editor) [below right=0.5cm of git-EDITOR]
	 {Sec.~\ref{sec:tutorial-git}\\ \nolinknameref{sec:tutorial-git}};
	\node[VertexStyle] (svn-emacs)  [below left=0.5cm of svn-EDITOR] 
	 {Sec.~\ref{sec:tutorial-svn-emacs}\\ \nolinknameref{sec:tutorial-svn-emacs}};
	\node[VertexStyle] (svn-editor) [below right=0.5cm of svn-EDITOR]
	 {Sec.~\ref{sec:tutorial-svn}\\ \nolinknameref{sec:tutorial-svn}};
	
	% explanations
	\node[state] (expl-third) [left=2.7cm of git-EDITOR,draw=none] {which editor?};
	\node[state] (expl-first) [left=5cm of INIT,draw=none] {which version control?};
	
	\path
		(INIT)        edge node[LabelStyle,above left]  {git}        (example-git)
	  	(INIT)        edge node[LabelStyle,above right] {svn}        (example-svn)
		(example-git) edge (git-EDITOR)
		(example-svn) edge (svn-EDITOR)
		(git-EDITOR)  edge node[LabelStyle,above left]  {Emacs}      (git-emacs)
		(git-EDITOR)  edge node[LabelStyle,above right] {LTC Editor} (git-editor)
		(svn-EDITOR)  edge node[LabelStyle,above left]  {Emacs}      (svn-emacs)
		(svn-EDITOR)  edge node[LabelStyle,above right] {LTC Editor} (svn-editor)
		;
	\end{tikzpicture}

% !TEX root = manual.tex
\section{Creating the Example Git Repository} \label{sec:example-git}

The git-based tutorials use two example git repositories called ``independence.bundle'' and ``independence-sherman.bundle,'' which can be downloaded from 
\url{http://sourceforge.net/projects/latextrack/files/examples/}.
First, save the bundled repositories into a directory of your choice.  We call this directory \Code{\$TUTORIAL}.  Then, clone from the first bundle to obtain a valid git working tree.
\begin{CodeVerbatim}
$> cd $TUTORIAL/
$> git clone independence.bundle independence
Cloning into 'independence'...
Receiving objects: 100% (18/18), done.
Resolving deltas: 100% (4/4), done.
$> cd independence/
$> git status
# On branch master
nothing to commit (working directory clean)
$> git log --oneline
d3f904c sixth version
203e0ce fifth version
36eeab0 fourth version
fa2be39 third version
bac2f51 second version
d6d1cf8 first version
\end{CodeVerbatim}

Now we impersonate John Adams to work on this writing project for the Declaration of Independence.

\begin{CodeVerbatim}
$> git config --add user.name "John Adams"
$> git config --add user.email "adams@usa.gov"
$> git config --list | grep -e "[Aa]dams"
user.name=John Adams
user.email=adams@usa.gov
\end{CodeVerbatim}

Another way to investigate the current git repository are graphical tools such as gitk (comes with git distribution) or GitX under Mac OS X.  Note that GitX is not required to run LTC.  Figure~\ref{fig:gitx-screen} for using GitX on the just created repository.
\begin{figure}[t]
\centering
\mygraphics{width=\textwidth,height=.5\textheight,keepaspectratio}{figures/gitx-screen}
\caption{Investigating example git repository with a graphical tool such as GitX} \label{fig:gitx-screen}
\end{figure}

The other point to note here is the way that GitX displays the changes in the file \Code{independence.tex} when using the graphical git interface.  It shows the lines in the file that have changed (much like a standard Unix diff would) -- however, when looking at changes in LaTeX source code, the granularity of the line-based difference is much too coarse.  An author would most likely only care about the change in words of line 15 or even characters such as removing the mistaken `d' in the word ``Roger'' in line seven.

\subsection{Collaborating} \label{sec:collaborating}

Collaboration on your writing project mainly happens through git so we show how to setup an example here.  Your actual setup for writing projects may differ.  Whatever the configuration, it will probably show up under the list of registered remotes for your repository.  In our example, we cloned from the downloaded .bundle file, so looking at the remotes will look like this:
\begin{CodeVerbatim}
$> git remote -v
origin	$TUTORIAL/independence.bundle (fetch)
origin	$TUTORIAL/independence.bundle (push)
\end{CodeVerbatim}

As an example of interacting with another repository, let us create a second one on our local file system.  In practice, the remote repository will most likely be on a different computer and accessed via certain network protocols reflected in the address.  Feel free to adjust the file locations in the example below to your taste.

\begin{CodeVerbatim}
$> cd $TUTORIAL/
$> git clone independence-sherman.bundle independence-sherman
Cloning into 'independence-sherman'...
Receiving objects: 100% (24/24), done.
Resolving deltas: 100% (6/6), done.
$> cd independence-sherman/
$> git log --oneline 
39cd617 todo item for indictment
45710ff more text for preamble
d3f904c sixth version
203e0ce fifth version
36eeab0 fourth version
fa2be39 third version
bac2f51 second version
d6d1cf8 first version
\end{CodeVerbatim}

Now we impersonate Roger Sherman in the newly created repository above, and also check the setting for its remotes.
\begin{CodeVerbatim}
$> git config --add user.name "Roger Sherman"
$> git config --add user.email "sherman@usa.gov"
$> git config --list | grep -e "[Ss]herman"
remote.origin.url=$TUTORIAL/independence-sherman.bundle
user.name=Roger Sherman
user.email=sherman@usa.gov
$> git remote -v
origin	$TUTORIAL/independence-sherman.bundle (fetch)
origin	$TUTORIAL/independence-sherman.bundle (push)
\end{CodeVerbatim}

Next, we make the first repository aware of the second and vice versa.  At the same time, we may want to remove the reference to the original bundle so as to not get confused with which repository to synchronize.  So in both repositories do
\begin{CodeVerbatim}
$> git remote remove origin  # this is optional!
\end{CodeVerbatim}

Then, we go into the first one and add a new remote location there:
\begin{CodeVerbatim}
$> cd $TUTORIAL/independence/
$> git remote add sherman $TUTORIAL/independence-sherman
$> git remote -v
sherman	$TUTORIAL/independence-sherman (fetch)
sherman	$TUTORIAL/independence-sherman (push)
\end{CodeVerbatim}

Afterwards, we go into the second one and add a new remote location there:
\begin{CodeVerbatim}
$> cd $TUTORIAL/independence-sherman/
$> git remote add adams $TUTORIAL/independence
$> git remote -v
adams	$TUTORIAL/independence (fetch)
adams	$TUTORIAL/independence (push)
\end{CodeVerbatim}

Now you can pull from each directory what the other person has done.  Notice that you cannot push changes to the other directory, as these git repositories are not ``bare.''  This means, they contain working copies and thus cannot be altered remotely.  However, in most situations you may be using a central repository (such as GitHub or a server) that indeed contains a bare repository.  Then, you are typically able to pull and push changes with such a remote repository while your coauthors can do the same to synchronize your work.

We will see examples below in Sections~\ref{sec:tutorial-git-emacs:collab} and \ref{sec:tutorial-git:collab} how John Adams and Roger Sherman synchronize changes with each other.

%At this point, the two repositories are synched so the operations do not perform any changes.  These two commands are also available through the LTC Editor user interface---after choosing the remote from the pull-down menu in the lower-right corner, click the ``Pull'' or ``Push'' buttons.  If the menu is empty you may have to update the editor to obtain the latest git changes that you may have performed on the command line or using a graphical git tool.
%\begin{CodeVerbatim}
%$> git push philadelphia master
%Everything up-to-date
%$> git pull philadelphia master
%From /Users/linda/git/independence
% * branch            master     -> FETCH_HEAD
%Already up-to-date.
%\end{CodeVerbatim}

% TODO: create example that shows branches in commit graph => independence2.bundle?

%\subsection{Resolving Merge Conflicts}


% !TEX root = manual.tex
\section{Tutorial with Git and Emacs} \label{sec:tutorial-git-emacs}

In this section, we assume that the example git repository has been created according to the instructions in Section~\ref{sec:example-git} above.  And we assume that LTC has been installed using the optional Emacs directory, as well as Emacs configuration adjustments made that are mentioned in Section~\ref{sec:config-emacs}.


% !TEX root = manual.tex
\section{Tutorial with Git and LTC Editor} \label{sec:tutorial-git}

In this section, we assume that the example git repository has been created according to the instructions in Section~\ref{sec:example-git} above.

\subsection{Running the LTC Editor}

First, we start the LTC Editor to interact with LTC and track the changes of the file.  Assuming you have installed LTC in the directory \Code{\$LTC}, we can look at the command line options of the editor:
\begin{CodeVerbatim}[commandchars=\\\{\}]
$> java -cp $LTC/LTC.jar com.sri.ltc.editor.LTCEditor -h
LaTeX Track Changes (LTC)  Copyright (C) 2009-2013  SRI International
This program comes with ABSOLUTELY NO WARRANTY; for details use command line switch -c.
This is free software, and you are welcome to redistribute it under certain conditions.

usage: java -cp ... com.sri.ltc.editor.LTCEditor [options...] [FILE] 
with
 FILE     : load given file to track changes
 -c       : display copyright/license information and exit
 -h       : display usage and exit
 -l LEVEL : set console log level
            SEVERE, WARNING, INFO, CONFIG (default), FINE, FINER, FINEST
 -r       : reset to default settings
\end{CodeVerbatim}

To open our tutorial file at \Code{\$TUTORIAL/independence/independence.tex} when starting the editor, execute the following command.  This will open the editor as a window similar to the screen shot in Figure~\ref{fig:editor-open}.
\begin{CodeVerbatim}[samepage=true,commandchars=\\\{\}]
$> java -cp $LTC/LTC.jar \textbackslash
   com.sri.ltc.editor.LTCEditor $TUTORIAL/independence/independence.tex
\end{CodeVerbatim}
\begin{figure}[t]
\centering
\mygraphics{width=\textwidth,height=.5\textheight,keepaspectratio}{figures/editor-open}
\caption{Initial opening of tutorial file in LTC Editor} \label{fig:editor-open}
\end{figure}
In this figure, we see a panel at the bottom-right that resembles the upper part of the GitX graphical interface to git.  There, we display the history of the current LaTeX file under git.  We can also see what git currently perceives as the current user -- now John Adams because we had overridden the git settings in this tutorial repository.

\subsection{Showing and Hiding Certain Changes}

The bottom-left panels of the editor allows us to customize the way LTC displays the changes of the file.  Section~\ref{sec:general-use} contains all the details of how LTC displays the changes including limiting the file history and filtering.  In this tutorial, we will just use some of the options and see their effect.

First, notice the colors assigned to each of the authors.  To change an author color, for example Roger Sherman's,  perform a double-click on the colored square next to Roger Sherman to open a dialog and  choose a dark color such as brown (you will want something with contrast to the white background).  Notice how the text in the editor panel on the top changes color for those parts that are attributed to Roger Sherman's edits.

Next, focus on the typographical errors in the command ``\textbackslash maketitle'' in line 11 and the beginning of the first paragraph in line thirteen as well as the spelling errors in the word ``political.''  If you first uncheck the box for ``small'' changes and second, also the box for deletions, notice how the text rendering in the editor panel changes.
\begin{figure}[t]
  \centering
  \subfloat{
    \label{subfig:editor-filter-small1} 
    \mygraphics{width=.25\textwidth}{figures/editor-filter-small1}}
  \hspace{2em}
  \subfloat{
    \label{subfig:editor-filter-small2} 
    \mygraphics{width=.25\textwidth}{figures/editor-filter-small2}}
  \hspace{2em}
  \subfloat{
    \label{subfig:editor-filter-small3} 
    \mygraphics{width=.25\textwidth}{figures/editor-filter-small3}}
\caption{Effect of hiding ``small'' changes first (middle) and then also deletions (right)} \label{fig:editor-filter-small}
\end{figure}
Figures~\ref{fig:editor-filter-small} show that ``\textbackslash maketitle'' as well as the typos in the word ``political'' are no longer marked up, and in the third image, the deletion beginning with ``If'' at the beginning of the paragraph is now omitted.

\subsection{Understanding the Commit Graph}

Now draw your attention back to the graph with the history of the current file under git (located in the bottom-right panel).  In our current tutorial repository, this graph is just a line as the authors committed their versions in sequential order.  %For the tutorial example, in which the writing project contained only one .tex file, the history graph in the LTC Editor looks almost the same as viewing the git repository in a graphical tool such as GitX.  However, if the writing project contained more .tex files, these graphs would not look similar anymore.  The git repository is tracking the history of all files that have been added to it.  In contrast, the LTC Editor only displays the history of the current .tex file loaded.
\begin{figure}[t]
\centering
\mygraphics{scale=.5}{figures/commit-graph}
\caption{Example of commit graph} \label{fig:commit-graph}
\end{figure}
Refer to Figure~\ref{fig:commit-graph} for a screen shot of the example file history. Versions that are included in the tracked changes are printed in black and denoted with a filled circle.  How far we go back in history depends on some filtering settings, which are discussed further in Section~\ref{sec:limit-history} below.  By default, we first include all version of the current author at the top.  In our example with impersonating John Adams, there are currently no further recent commits of him.  Then, we continue down the path and collect all versions of different authors until we find the next version of John Adams in the commit with the message ``second version.''

%After we have obtained an ordered sequence of commits by traversing the graph from the top down, we ignore subsequent commits of the same author---only the latest version of the same author is considered.  The idea behind this is that we do not want to penalize an author who commits often to the repository.  Instead, we treat all his or her changes from the time the file was changed by a different author as one big change event.  In our example, see that Roger Sherman committed version 5 and 6 to the repository, but version 5 is depicted in gray and not taken into account when displaying the changes.  

\subsection{Limiting History} \label{sec:limit-history}

We allow the user to filter and customize how the potentially rich history of a .tex file is selected, so as to provide a better view of the tracked changes.  The user can show and hide changes as seen above, limit the authors of interest, and specify a date or revision number to tell LTC how far back in time the history should be considered.

%\subsubsection{Limiting by Authors}

\begin{figure}
\centering
  % first sub-figure
  \begin{minipage}[t]{0.35\linewidth}
  \centering
  \mygraphics{scale=.5}{figures/editor-select-authors}
  \caption{Selecting authors for filtering} \label{fig:editor-select-authors}
  \end{minipage}%
\hspace{0.04\linewidth}%
  % second sub-figure
  \begin{minipage}[t]{0.61\linewidth}
  \centering
  \mygraphics{scale=.5}{figures/editor-limit-authors}
  \caption{Effect of limiting authors to Roger Sherman and Thomas Jefferson after clicking ``Update''} \label{fig:editor-limit-authors}
  \end{minipage}  
\end{figure}
To limit the history by \textbf{authors}, select both authors Roger Sherman and Thomas Jefferson through clicking while holding down the CTRL or CMD key in the list of authors in the middle lower panel.  Then, click the button ``Limit'' below the list, which will gray out the unselected authors.  For a limiting action to take effect, you need to click ``Update.''  This is different from showing and hiding various changes as well as changing author colors, which is applied instantly.

Notice how any version by the ignored authors Benjamin Franklin and John Adams is now gray as only commits from the selected authors are considered.  Again, the history is only taken until the next revision of the current author but since he is being ignored, we go all the way back to the first revision. Compare your editor window with the screen shot in Figure~\ref{fig:editor-limit-authors} and see how the commit graph has changed.

Then, clicking the ``Reset'' button followed by ``Update'' will remove and limits on the history by author, so the original view is restored.

%\subsubsection{Limiting by Date or Revision}

Next, we apply limits on \textbf{revision} and \textbf{date} to control how far back the history of the file is considered.  As we had seen, the first version is not taken into account because it was committed before the next commit by the current author John Adams.  Let us now type the first few characters \Code{d6d} of the SHA-1 key of the first commit into the field labeled ``Start at revision:'' (refer to Figure~\ref{fig:editor-select-rev}) or simply drag the key from the entry in the commit graph, which will copy the complete SHA-1 sequence.  Now click ``Update'' and see how the first version is listed in black and considered in the tracked changes above as seen in Figure~\ref{fig:editor-limit-rev}.  Since changes by the current author John Adams from the first to the second version are now included, notice the text marked up in red appearing in the editor window. We see that John Adams must have added himself as an author in the LaTeX preamble among other edits.

\begin{figure}
\centering
  % first sub-figure
  \begin{minipage}[t]{0.35\linewidth}
  \centering
  \mygraphics{scale=.5}{figures/editor-select-rev}
  \caption{Selecting revision for filtering} \label{fig:editor-select-rev}
  \end{minipage}%
\hspace{0.04\linewidth}%
  % second sub-figure
  \begin{minipage}[t]{0.61\linewidth}
  \centering
  \mygraphics{scale=.5}{figures/editor-limit-rev}
  \caption{Effect of going back to first revision after clicking ``Update''} \label{fig:editor-limit-rev}
  \end{minipage}  
\end{figure}

To remove the limit by revision number, simply erase the text in the field ``Start at revision:'' and click ``Update'' again.

\begin{figure}
\centering
  % first sub-figure
  \begin{minipage}[t]{0.35\linewidth}
  \centering
  \mygraphics{scale=.5}{figures/editor-select-date}
  \caption{Selecting date for filtering} \label{fig:editor-select-date}
  \end{minipage}%
\hspace{0.04\linewidth}%
  % second sub-figure
  \begin{minipage}[t]{0.61\linewidth}
  \centering
  \mygraphics{scale=.5}{figures/editor-limit-date}
  \caption{Effect of limiting history to date of third version after clicking ``Update''} \label{fig:editor-limit-date}
  \end{minipage}  
\end{figure}

Limiting the history by date works similarly.  You may drag a date from the commit graph on the right, for example the date of the third version commit, and drop it into the field ``Start at date:'' on the left.  Or, type a date such as \Code{Jul 23, 2010 1:11p} into the field.  We employ some software to process times and dates in natural language, and if successful, the field will contain the date string as it was understood translated into the format used in the commit tree.
Again, you will need to click ``Update'' for the change to take effect or click RETURN while editing the text field.  See Figures~\ref{fig:editor-select-date} and \ref{fig:editor-limit-date} for screenshots. To remove the limit by date, erase all text in the text field and update.

\subsection{Condensing History}

\begin{figure}
\centering
  \begin{minipage}[b]{0.57\linewidth}
    \centering
    \mygraphics{scale=.5}{figures/editor-condense-on}
    \caption{Effect of condensing authors:\\ ignoring the fifth version by Roger Sherman} \label{fig:editor-condense-on}  
  \end{minipage}  
\hspace{0.04\linewidth}%
  \begin{minipage}[b]{0.38\linewidth}
    \subfloat{
      \label{subfig:editor-condense-before} 
      \mygraphics{width=0.8\linewidth}{figures/editor-condense-before}}
  \hspace{1em}
    \subfloat{
      \label{subfig:editor-condense-after} 
      \mygraphics{width=0.8\linewidth}{figures/editor-condense-after}}
    \caption{Example of markup change before (left) and after (right) condensing authors} \label{fig:editor-condense-before-after}
  \end{minipage}%
\end{figure}
Sometimes the list of commits considered is getting long and the resulting markup of the changes confusing.  One additional way to customize how the changes are displayed is a setting to ``condense authors.''  Find a check box with that name under the list of authors for filtering.  If checked, then only the latest version of an author of \textit{consecutive} commits is considered -- in our example, only the sixth version is shown in black while the fifth version by Roger Sherman is now grayed out as seen in Figure~\ref{fig:editor-condense-on}.

See Figure~\ref{fig:editor-condense-before-after} for an example of how condensing authors affects the markup.  Since we are only considering the changes that Roger Sherman made in the sixth version, his correction of the name is no longer shown.  Condensing authors makes sense when users commit versions often so that they do not loose too much history.  Their last version after a number of commits generally has the flavor of a ``final'' version, ready to be shared with others.  Hence, the changes made there compared to the last version of another author is commonly of most interest.

\subsection{Editing, Saving, and Committing}

Let us start the next step by resetting all filters to the default configuration, i.e., no limit by authors, date, and revision.  Then, we will edit the text in the editor panel to see the latest changes.

Click into the text field and enter some text, for example a LaTeX comment reminding John Adams to work on a list of charges against King George III in line nineteen:
\begin{FileVerbatim}
% list charges against King George III here
\end{FileVerbatim}
The added text will be rendered in red (or the color code for the current author) and underlined.  Notice how the commit graph adds a first line with the label ``modified'' and the ``Save'' button becomes enabled.  Now delete some of the characters you have just entered, for example the word \Code{here} at the end.  The characters simply disappear as they were added by the same author.

Now delete other characters that are either rendered black or a different color than red but not marked as deletions (strike-through font).  Notice how these characters remain visible but are now colored red and marked up with strike-through.  If you tried to delete anything that is already marked as deletion (i.e., anything in strike-through font), nothing will happen as this text is already deleted in a prior version.  See Figure~\ref{fig:editor-modified} for a screen shot of the above edits: text in red and underlined was added and text in red and strike-through was deleted.

\begin{figure}[t]
\centering
\mygraphics{width=\textwidth,height=.5\textheight,keepaspectratio}{figures/editor-modified}
\caption{After editing the text as John Adams} \label{fig:editor-modified}
\end{figure}

Finally, you will want to click ``Save'' to save the current file to disk.  This will cause the label ``modified'' to change to ``on disk.''  If you would then again edit, the label would switch back to ``modified'' of course.

Saving the file, however, does not tell git to create a new version under its management.  In order to commit the current file to git, first, make sure that the .tex file is saved and  the first line in the commit graph is set to ``on disk.''  Then, on a command line, switch to the directory with the tutorial file and perform the following commit command (printed in bold below).  You may want to check the status of git before and after the commit:
\begin{CodeVerbatim}[commandchars=\\\{\}]
$> git status
# On branch master
# Changes not staged for commit:
#   (use "git add <file>..." to update what will be committed)
#   (use "git checkout -- <file>..." to discard changes in working directory)
#
#	modified:   independence.tex
#
no changes added to commit (use "git add" and/or "git commit -a")
$> \textbf{git commit -am "added comment about list of charges"}
[master 930d080] added comment about list of charges
 1 file changed, 3 insertions(+), 1 deletion(-)
\end{CodeVerbatim}

To make LTC Editor aware of the underlying commit from the command line, click the ``Update'' button.  Notice how the recent commit gets included at the top of the list as seen in Figure~\ref{fig:commit-cmd-line}.
\begin{figure}[t]
\centering
\mygraphics{scale=.5}{figures/commit-cmd-line}
\caption{Updated commit graph after command line commit} \label{fig:commit-cmd-line}
\end{figure}

Now two things can happen depending on your setting for showing changes in comments (lower-left most panel).  If the checkbox was off, the newly added comment is no longer marked up in red with underlining.  After updating the editor from the commit history, the settings of which changes to show influence the markup of the text.  Now the newly entered comment is recognized as such, and if we hide changes in comments, the markup will not show.  If the box for ``changes in comments'' is checked, you will see your latest text still marked up as an addition.  Your editor should now look similar to the part shown in either Figure~\ref{fig:commit-no-comment} or Figure~\ref{fig:commit-comment}.
\begin{figure}
\centering
  % first sub-figure
  \begin{minipage}[t]{0.48\linewidth}
  \centering
  \mygraphics{width=\linewidth,height=.5\textheight,keepaspectratio}{figures/editor-commit-no-comment}
  \caption{Committing a comment but not showing it as an addition} \label{fig:commit-no-comment}
  \end{minipage}%
\hspace{0.04\linewidth}%
  % second sub-figure
  \begin{minipage}[t]{0.48\linewidth}
  \centering
  \mygraphics{width=\textwidth,height=.5\textheight,keepaspectratio}{figures/editor-commit-comment}
  \caption{Committing a comment and showing it as an addition} \label{fig:commit-comment}
  \end{minipage}  
\end{figure}

\subsection{Collaborating with Roger Sherman} \label{sec:tutorial-git:collab}

Next we perform an example collaboration with Roger Sherman's example repository as setup in Section~\ref{sec:collaborating} above. Remember that Roger Sherman's repository has had two more commits than the original one we used for John Adams.  Since we have made edits and commits as John Adams, both repositories have diverged.  To synchronize them, we first pull Roger Sherman's changes into our working copy after checking that we are in a good state:

\begin{CodeVerbatim}
$> git pull sherman master
remote: Counting objects: 8, done.
remote: Compressing objects: 100% (3/3), done.
remote: Total 6 (delta 2), reused 5 (delta 1)
Unpacking objects: 100% (6/6), done.
From $TUTORIAL/independence-sherman
 * branch            master     -> FETCH_HEAD
Auto-merging independence.tex
CONFLICT (content): Merge conflict in independence.tex
Automatic merge failed; fix conflicts and then commit the result.
\end{CodeVerbatim}

\begin{figure}
\centering
\mygraphics{scale=.35}{figures/editor-merge-conflict}
\caption{Git conflict markers in merged file} \label{fig:editor-merge-conflict}
\end{figure}
Unfortunately, the two repositories have diverged too much and a so-called ``merge conflict'' has arisen.  Now we have to tell git how to fix this before we can proceed.  So we look at the markers that git has put into our file.  You can click the ``Update'' button in LTC Editor to see these markers there similar to Figure~\ref{fig:editor-merge-conflict}.  On the command line, the file looks similar to this:
\begin{CodeVerbatim}
$ cat independence.tex
[...]

<<<<<<< HEAD
% list charges against King George III
=======
That to secure these rights, Governments are instituted among Men, [...] 

%TODO: indictment here
>>>>>>> 39cd6172613d1065a4cddc854cf30067869fc727
\end{CodeVerbatim}

We decide that the comments in the \Code{HEAD} version means the same as the last comment in the merged version \Code{39cd617...} so we modify the text so that it looks like this:
\begin{CodeVerbatim}
$> cat independence.tex 
[...]

That to secure these rights, Governments are instituted among Men, [...]

% list charges against King George III
\end{CodeVerbatim}
It is important to remove the git marker lines starting with \Code{<<<<<<<}, \Code{=======}, and \Code{>>>>>>>} for git to recognize that we have resolved the conflicts.  Now committing on the command line yields:
\begin{CodeVerbatim}
$> git commit -am "merging Roger Sherman's edits"
[master 2a5b3c3] merging Roger Sherman's edits
\end{CodeVerbatim}

\begin{figure}
\centering
\mygraphics{scale=.35}{figures/editor-merge-resolve}
\caption{After resolving conflict, committing, and updating in LTC Editor} \label{fig:editor-merge-resolve}
\end{figure}
This has resolved the conflict and incorporated Roger Sherman's prior changes, as a look at the git log with the graphing function reveals:
\begin{CodeVerbatim}
$> git log --oneline --graph --date-order
*   2a5b3c3 merging Roger Sherman's edits
|\  
* | 930d080 added comment about list of charges
| * 39cd617 todo item for indictment
| * 45710ff more text for preamble
|/  
* d3f904c sixth version
* 203e0ce fifth version
* 36eeab0 fourth version
* fa2be39 third version
* bac2f51 second version
* d6d1cf8 first version
\end{CodeVerbatim}
Once we update LTC Editor, we see the paragraph that was part of the conflicting region now correctly attributed to Roger Sherman.  Furthermore, the git commit graph has gotten more interesting with the branching and merging in the first column of the commit graph.  Refer to Figure~\ref{fig:editor-merge-resolve} for a screen shot of the git commit graph and text changes after resolving the conflict, committing and updating LTC Editor.


% !TEX root = manual.tex
\section{Creating the Example Subversion Repository} \label{sec:example-svn}

This tutorial uses an example svn repository, which is either hosted on the Internet or on your local computer.  The first Section~\ref{sec:example-svn-remote} shows how to use a publicly accessible repository with an example file.  This is the quickest way to try out LTC with Subversion but you cannot commit new versions to this repository so we cannot go through such advanced topics in the later parts of the tutorial.  The second Section~\ref{sec:example-svn-local} shows how to create a local svn server and populates it with an example repository.  This takes a bit more time to setup but then you can go through more advanced topics such as committing to the repository.

\subsection{Using the Remote Subversion Repository} \label{sec:example-svn-remote}

To create the example file that is under remote svn version control, go into a directory of your choice (say \Code{\$TUTORIAL}) and do the following.  If the server causes a certificate alert, you can accept it permanently by using \Code{p} as shown in bold below.

\begin{CodeVerbatim}[commandchars=\\\{\}]
$> cd $TUTORIAL
$> svn co https://spartan.csl.sri.com/svn/public/LTC/tutorial-svn independence 
Error validating server certificate for 'https://spartan.csl.sri.com:443':
 - The certificate is not issued by a trusted authority. Use the
   fingerprint to validate the certificate manually!
Certificate information:
 - Hostname: spartan.csl.sri.com
 - Valid: from Fri, 03 May 2013 00:00:00 GMT until Sat, 03 May 2014 23:59:59 GMT
 - Issuer: Thawte, Inc., US
 - Fingerprint: f9:3f:b3:27:65:89:c9:af:bf:05:1b:5f:60:f0:8c:df:4d:bc:47:7e
(R)eject, accept (t)emporarily or accept (p)ermanently? \textbf{p}
A    independence/independence.tex
Checked out revision 6.
\end{CodeVerbatim}

Now change into the new directory and confirm that the file has six revisions in its history
\begin{CodeVerbatim}
$> cd independence/
$> svn log -q independence.tex 
------------------------------------------------------------------------
r6 | sherman | 2012-11-13 13:01:00 -0600 (Tue, 13 Nov 2012)
------------------------------------------------------------------------
r5 | sherman | 2012-11-13 13:00:35 -0600 (Tue, 13 Nov 2012)
------------------------------------------------------------------------
r4 | jefferson | 2012-11-13 12:59:45 -0600 (Tue, 13 Nov 2012)
------------------------------------------------------------------------
r3 | franklin | 2012-11-13 12:59:03 -0600 (Tue, 13 Nov 2012)
------------------------------------------------------------------------
r2 | adams | 2012-11-13 12:58:04 -0600 (Tue, 13 Nov 2012)
------------------------------------------------------------------------
r1 | jefferson | 2012-11-13 12:51:35 -0600 (Tue, 13 Nov 2012)
------------------------------------------------------------------------
\end{CodeVerbatim}

Unfortunately, we cannot accept changes to this repository so the tutorials based on svn do not cover how to commit new revisions and how to collaborate.  We advise to install a local svn server and repository per the instructions below or to go through the git-based tutorial to cover those points.

\subsection{Using a Local Subversion Repository} \label{sec:example-svn-local}

%TODO
% !TEX root = manual.tex
\section{Tutorial with Subversion and Emacs} \label{sec:tutorial-svn-emacs}

In this section, we assume that the example svn repository has been created according to the instructions in Section~\ref{sec:example-svn} above.  And we assume that LTC has been installed using the Emacs directory, as well as Emacs configuration adjustments made that are mentioned in Section~\ref{sec:config-emacs}.

\subsection{Starting LTC Server and \texttt{ltc-mode}}

First, we start the LTC Server from the command line.  Assuming you have installed LTC in the directory \Code{\$LTC}, we run this command line for the server.  The output will be similar to the following.  Leave the server running while performing the rest of this tutorial.
\begin{CodeVerbatim}
$> java -jar $LTC/LTC.jar
LaTeX Track Changes (LTC)  Copyright (C) 2009-2013  SRI International
This program comes with ABSOLUTELY NO WARRANTY; for details use command line switch -c.
This is free software, and you are welcome to redistribute it under certain conditions.

<current date> | CONFIG:  Logging configured to level CONFIG
<current date> | CONFIG:  LTC version: <version info>
<current date> | INFO:    Started RPC server on port 7777.
\end{CodeVerbatim}

Next, we switch to Emacs and open the tutorial file \Code{\$TUTORIAL/independence/independence.tex}.  This should put Emacs into \Code{latex-mode} but any other mode should work as well.  Then, start LTC mode using the command \Code{M-x ltc-mode} in Emacs.  Beware that using LTC with a remote subversion server takes longer than using git or a local subversion server, as we have to query the distant server hosting the repository for each version of the .tex file.  You will see a few messages appearing briefly in the mini buffer (you can also look at them in the \Code{*Messages*} buffer), such as the following.  While the potential time intensive task of downloading versions from the remote server happen, the mini buffer should say \Code{Starting LTC update...}, which turns into \Code{LTC updates received} when the process is done.
\begin{FileVerbatim}
Starting LTC mode for file "$TUTORIAL/independence/independence.tex"...
Using `xml-rpc' package version: 1.6.8.2
LTC session ID = 1
Starting LTC update...
LTC updates received
\end{FileVerbatim}
Emacs should now look similar to the screen shot in Figure~\ref{fig:svn-emacs-open}.
\begin{figure}[t]
\centering
\mygraphics{scale=.35}{figures/svn-emacs-open}
\caption{Starting \Code{ltc-mode} in Emacs with tutorial file under svn} \label{fig:svn-emacs-open}
\end{figure}
In this figure, we see the changes marked up in various colors and fonts: underlined for additions and inverse for deletions.  There is also a smaller buffer called ``LTC info (session $N$)'' with $N$ as the session ID, at the bottom (or the right, if your Emacs is in landscape mode) of the buffer with the tutorial file.  There, we display the history of the current file under svn.  We can also see what svn perceives as the current user at the top of the graph -- here the name of the local user.

First, we will override what LTC thinks is the current author in order to make the following tutorial more meaningful.  In real life situations you will rarely have to use this command as you typically want the changes in the repository attributed to yourself.  In Emacs, type the command \Code{M-x ltc-set-self<RET>adams<RET>} to impersonate John Adams.  This updates the contents in the main buffer and info buffer at the bottom automatically, which may again take a little time with a remote subversion server.  The info buffer will then look like the screen shot in Figure~\ref{fig:svn-adams}.
\begin{figure}[t]
\centering
\mygraphics{scale=.5}{figures/svn-adams}
\caption{Emacs info buffer after setting current author to ``adams''} \label{fig:svn-adams}
\end{figure}

\subsection{Showing and Hiding Certain Changes}

The LTC menu and ``LTC info'' buffer in Emacs allow us to customize the way LTC displays the changes of the file.  Section~\ref{sec:general-use} contains all the details of how LTC displays the changes including limiting the file history and filtering.  In this tutorial, we will just use some of the options and see their effect.

First, notice the colors assigned to each of the authors.  To change an author color, for example Roger Sherman's,  perform a single left-click on the name of Roger Sherman.  This opens another buffer called \Code{*Colors*} with a preview of colors and their names.  Also look at the mini buffer that requests input.   You can enter a name or an RGB value in hex notation.  The color names can also be auto-completed, for example type \Code{Bro<TAB>} (if TAB is your completion key in Emacs) to see \Code{Brown}.  You will want something with contrast to the white background, so brown is a fine choice.  When clicking the RETURN key, notice how the text in the editor panel on the top changes color for those parts that are attributed to Roger Sherman's edits.  To abort choosing a color simply enter an empty value.

Next, focus on the typographical errors in the command ``\textbackslash maketitle'' in line 11 and the beginning of the first paragraph in line thirteen as well as the spelling errors in the word ``political.''  Open the LTC menu (in the menu bar and in the mode line) and then the sub-menu ``Show/Hide'' as seen in Figure~\ref{fig:svn-emacs-LTC-menu}.  
\begin{figure}[t]
\centering
\mygraphics{scale=.35}{figures/svn-emacs-LTC-menu}
\caption{Opening the LTC menu from the mode line in Emacs} \label{fig:svn-emacs-LTC-menu}
\end{figure}
If you first uncheck the item \Menu{Show/Hide;Show small changes}{LTC}, and second, also the item \Menu{Show/Hide;Show deletions}{LTC}, notice how the text rendering in the editor panel changes.  Again, if you work with the remote repository, it might take a little while until all the updates are received from the server and the mini buffer shows the message ``LTC updates received.''
\begin{figure}[t]
  \centering
  \subfloat{
    \label{subfig:svn-emacs-filter-small1} 
    \mygraphics{scale=.5}{figures/svn-emacs-filter-small1}}
  \hspace{2em}
  \subfloat{
    \label{subfig:svn-emacs-filter-small2} 
    \mygraphics{scale=.5}{figures/svn-emacs-filter-small2}}
  \hspace{2em}
  \subfloat{
    \label{subfig:svn-emacs-filter-small3} 
    \mygraphics{scale=.5}{figures/svn-emacs-filter-small3}}
\caption[Effect of hiding ``small'' changes and deletions]{Effect of hiding ``small'' changes first (middle) and then also deletions (right)} \label{fig:svn-emacs-filter-small}
\end{figure}
Figures~\ref{fig:svn-emacs-filter-small} show that ``\textbackslash maketitle'' as well as the typos in the word ``political'' are no longer marked up, and in the third image, the deletion beginning with ``If'' at the beginning of the paragraph is now omitted.

\subsection{Understanding the Commit Graph}

Now draw your attention back to LTC info buffer with the history of the current file under git (located at the bottom or right of your .tex file).  The Emacs representation is using small box characters to draw the graph and its edges.  In our current tutorial repository, there are no branches and the graph is a sequential line.  

Refer back to Figure~\ref{fig:svn-adams} for the screen shot of the example file history. Versions that are included in the tracked changes are not printed in gray.  How far we go back in history depends on some filtering settings, which are discussed further in Section~\ref{sec:svn-emacs-limit-history} below.  By default, we first include all version of the current author at the top.  In our example with impersonating John Adams with the user name ``adams,'' there are currently no further recent commits of him.  Then, we continue down the path and collect all versions of different authors until we find the next version of John Adams in the commit with the message ``second version.''

\subsection{Limiting History} \label{sec:svn-emacs-limit-history}

We allow the user to filter and customize how the potentially rich history of a .tex file is selected, so as to provide a better view of the tracked changes.  The user can show and hide changes as seen above, limit the authors of interest, and specify a date or revision number to tell LTC how far back in time the history should be considered.

\begin{figure}
\centering
\mygraphics{scale=.5}{figures/svn-emacs-limit-authors}
\caption[Effect on commit graph of limiting authors]{Effect on commit graph of limiting authors to ``sherman'' and ``jefferson''} \label{fig:svn-emacs-limit-authors}
\end{figure}
To limit the history by \textbf{authors}, choose menu item \Menu{Limit by;Set of authors...}{LTC}. This will prompt the user to enter author names to limit by in the mini buffer.  Again, automatic completion works, so you can enter \Code{sh<TAB> <RET>} and \Code{je<TAB> <RET> <RET>} to select authors ``sherman'' and ``jefferson'' and exit the dialog.  After the last author was selected, the system automatically updates the displayed changes.

Notice how any version by the ignored authors ``franklin'' and ``adams'' is now gray as only commits from the selected authors are considered.  The first line in the LTC info buffer still shows the currently active author ``adams,'' so this line is not gray.  Again, the history is only taken until the next revision of the current author but since he is being ignored, we go all the way back to the first revision. Compare your Emacs now with the screen shot in Figure~\ref{fig:svn-emacs-limit-authors} and see how the file history has changed.

To reset limiting by authors, simply choose the same menu \Menu{Limit by;Set of authors...}{LTC} again and enter an empty author as the first one.  Now the display is back in the original state.

Next, we apply limits on \textbf{revision} and \textbf{date} to control how far back the history of the file is considered.  As we had seen, the first version is not taken into account because it was committed before the next commit by the current author John Adams.  Let us now choose menu item \Menu{Limit by;Start at revision...}{LTC}.  This will prompt the user to specify a known revision number.  Type the first few characters \Code{d6d<RET>} of the SHA-1 key of the first commit.  If unique, it is not necessary to expand the revision number using the TAB key (or whatever key is used for completion in your Emacs configuration).  See how the first version is listed in color and considered in the tracked changes above as seen in Figure~\ref{fig:svn-emacs-limit-rev}.  Since changes by the current author John Adams from the first to the second version are now included, notice the text marked up in red appearing in the editor window. We see that John Adams must have added himself as an author in the LaTeX preamble among other edits in the second commit of the file. 

\begin{figure}
\centering
%\mygraphics{scale=.5}{figures/svn-emacs-limit-rev}
\caption{Effect on commit graph of going back to first revision} \label{fig:svn-emacs-limit-rev}
\end{figure}

Another way of limiting by revision number is to simply left-click the number in the display.  

To remove the limit by revision number, simply choose the same menu \Menu{Limit by;Start at revision...}{LTC} again and enter an empty revision. Or, click into the empty revision column of the first line (denoting the currently active author) to achieve the same effect.  Now the display is back in the original state.

\begin{figure}
\centering
\mygraphics{scale=.5}{figures/svn-emacs-limit-date}
\caption{Effect on commit graph of limiting history to date of third version} \label{fig:svn-emacs-limit-date}
\end{figure}

Limiting the history by date works similarly.  Select menu item \Menu{Limit by;Start at date...}{LTC}. At the prompt, you can enter a date from the history of the file using auto-completion.  For example, enter \Code{2<TAB>2:59:0<TAB> <RET>} to get the exact date of the third revision.  Or, type a date such as \Code{Nov 13, 2012 12:59p} (should yield the same results if you are in the Central Time Zone) into the field.  We employ some software to process times and dates in natural language, and if successful, the field will contain the date string as it was understood translated into the format used in the commit tree. You may also perform a left-click on the date in the history to achieve the same effect.  See Figure~\ref{fig:svn-emacs-limit-date} for a screen shot of the effect of limiting to the date of the third revision. 

To remove the limit by date, either left-click on the empty date column of the first line of the file history or enter an empty date after selecting menu item \Menu{Limit by;Start at date...}{LTC} again.

\subsection{Condensing History}

\begin{figure}
\centering
\mygraphics{scale=.5}{figures/svn-emacs-condense-on}
\caption[Effect of condensing authors]{Effect of condensing authors: ignoring the fifth version by Roger Sherman} \label{fig:svn-emacs-condense-on}  
\end{figure}
Sometimes the list of commits considered is getting long and the resulting markup of the changes confusing.  One additional way to customize how the changes are displayed is a setting to ``condense authors.''  Now check the menu \Menu{Condense authors}{LTC}.  Then, only the latest version of an author of \textit{consecutive} commits is considered -- in our example, only the sixth version is colored while the fifth version by Roger Sherman is now grayed out as seen in Figure~\ref{fig:svn-emacs-condense-on}.

\begin{figure}
\centering
\subfloat{
  \label{subfig:svn-emacs-condense-before} 
  \mygraphics{scale=.5}{figures/svn-emacs-condense-before}}
\hspace{1em}
\subfloat{
  \label{subfig:svn-emacs-condense-after} 
  \mygraphics{scale=.5}{figures/svn-emacs-condense-after}}
\caption[Example of markup change when condensing authors]{Example of markup change before (left) and after (right) condensing authors} \label{fig:svn-emacs-condense-before-after}
\end{figure}
See Figure~\ref{fig:svn-emacs-condense-before-after} for an example of how condensing authors affects the markup.  Since we are only considering the changes that Roger Sherman made in the sixth version, his correction of the name is no longer shown.  Condensing authors makes sense when users commit versions often so that they do not loose too much history.  Their last version after a number of commits generally has the flavor of a ``final'' version, ready to be shared with others.  Hence, the changes made there compared to the last version of another author is commonly of most interest.

\subsection{Editing, Saving, and Committing}

% !TEX root = manual.tex
\chapter{Tutorial with Subversion} \label{ch:tutorial-svn}

\section{Creating the Example File}

This tutorial uses an example svn repository, which is hosted on the Internet.  To create the example file that is under svn version control, go into a directory of your choice (say \Code{\$TUTORIAL}) and do the following.

\begin{CodeVerbatim}
$> cd $TUTORIAL
$> svn co svn://svn.code.sf.net/p/latextrack/tutorial-svn/trunk independence
A    independence/independence.tex
Checked out revision 7.
$> cd independence/
$> svn status
$> svn log -q independence.tex 
------------------------------------------------------------------------
r7 | lilalinda | 2012-11-08 11:38:42 -0600 (Thu, 08 Nov 2012)
------------------------------------------------------------------------
r6 | lilalinda | 2012-11-08 11:38:27 -0600 (Thu, 08 Nov 2012)
------------------------------------------------------------------------
r5 | lilalinda | 2012-11-08 11:38:10 -0600 (Thu, 08 Nov 2012)
------------------------------------------------------------------------
r4 | lilalinda | 2012-11-08 11:37:52 -0600 (Thu, 08 Nov 2012)
------------------------------------------------------------------------
r3 | lilalinda | 2012-11-08 11:37:08 -0600 (Thu, 08 Nov 2012)
------------------------------------------------------------------------
r2 | lilalinda | 2012-11-08 11:35:53 -0600 (Thu, 08 Nov 2012)
------------------------------------------------------------------------
\end{CodeVerbatim}

[TODO: impersonate as different people for svn creation and tutorial!]

\section{Running the LTC Editor}

[TODO]

