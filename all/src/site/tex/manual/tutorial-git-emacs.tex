% !TEX root = manual.tex
\section{Tutorial with Git and Emacs} \label{sec:tutorial-git-emacs}

In this section, we assume that the example git repository has been created according to the instructions in Section~\ref{sec:example-git} above.  And we assume that LTC has been installed using the optional Emacs directory, as well as Emacs configuration adjustments made that are mentioned in Section~\ref{sec:config-emacs}.

\subsection{Starting LTC Server and \texttt{ltc-mode}}

First, we start the LTC Server from the command line.  Assuming you have installed LTC in the directory \Code{\$LTC}, we run this command line for the server.  The output will be similar to the following.  Leave the server running while performing the rest of this tutorial.
\begin{CodeVerbatim}
$> java -jar $LTC/LTC.jar
LaTeX Track Changes (LTC)  Copyright (C) 2009-2013  SRI International
This program comes with ABSOLUTELY NO WARRANTY; for details use command line switch -c.
This is free software, and you are welcome to redistribute it under certain conditions.

<current date> | CONFIG:  Logging configured to level CONFIG
<current date> | CONFIG:  LTC version: <version info>
--log4j: <current date> [main] INFO  org.apache.xmlrpc.webserver.XmlRpcServlet - init
<current date> | INFO:    Started RPC server on port 7777.
\end{CodeVerbatim}

Next, we switch to Emacs and open the tutorial file \Code{\$TUTORIAL/independence/independence.tex}.  This should put Emacs into \Code{latex-mode} but any other mode should work as well.  Then, start LTC mode using the command \Code{M-x ltc-mode} in Emacs.  You will see a few messages appearing briefly in the mini buffer (you can also look at them in the \Code{*Messages*} buffer), such as the following.
\begin{FileVerbatim}
Starting LTC mode for file "\$TUTORIAL/independence/independence.tex"...
Using `xml-rpc' package version: 1.6.8.1
LTC session ID = 1
Starting LTC update...
LTC updates received
\end{FileVerbatim}
Emacs should now look similar to the screen shot in Figure~\ref{fig:emacs-open}.
\begin{figure}[t]
\centering
\mygraphics{scale=.35}{figures/emacs-open}
\caption{Starting \Code{ltc-mode} in Emacs with tutorial file} \label{fig:emacs-open}
\end{figure}
In this figure, we see the changes marked up in various colors and fonts: underlined for additions and inverse for deletions.  There is also a smaller buffer called ``LTC info (session $N$)'' with $N$ as the session ID, at the bottom (or the right, if your Emacs is in landscape mode) of the buffer with the tutorial file.  There, we display the history of the current file under git.  We can also see what git currently perceives as the current user at the top of the graph -- now John Adams because we had overridden the git settings in this tutorial repository.

\subsection{Showing and Hiding Certain Changes}

The LTC menu and ``LTC info'' buffer in Emacs allow us to customize the way LTC displays the changes of the file.  Section~\ref{sec:general-use} contains all the details of how LTC displays the changes including limiting the file history and filtering.  In this tutorial, we will just use some of the options and see their effect.

First, notice the colors assigned to each of the authors.  To change an author color, for example Roger Sherman's,  perform a single left-click on the colored name of Roger Sherman.  This opens another buffer called \Code{*Colors*} with a preview of colors and their names.  Also look at the mini buffer that requests input.   You can enter a name or an RGB value in hex notation.  The color names can also be auto-completed, for example type \Code{Bro<TAB>} (if TAB is your completion key in Emacs) to see \Code{Brown}.  You will want something with contrast to the white background, so brown is a fine choice.  When clicking the RETURN key, notice how the text in the editor panel on the top changes color for those parts that are attributed to Roger Sherman's edits.  To abort choosing a color simply enter an empty value.

Next, focus on the typographical errors in the command ``\textbackslash maketitle'' in line 11 and the beginning of the first paragraph in line thirteen as well as the spelling errors in the word ``political.''  Open the LTC menu (in the menu bar and in the mode line) and then the sub-menu ``Show/Hide'' as seen in Figure~\ref{fig:emacs-LTC-menu}.  
\begin{figure}[t]
\centering
\mygraphics{scale=.35}{figures/emacs-LTC-menu}
\caption{Opening the LTC menu from the mode line in Emacs} \label{fig:emacs-LTC-menu}
\end{figure}
If you first uncheck the item \Menu{Show/Hide;Show small changes}{LTC}, and second, also the item \Menu{Show/Hide;Show deletions}{LTC}, notice how the text rendering in the editor panel changes.
\begin{figure}[t]
  \centering
  \subfloat{
    \label{subfig:emacs-filter-small1} 
    \mygraphics{scale=.5}{figures/emacs-filter-small1}}
  \hspace{2em}
  \subfloat{
    \label{subfig:emacs-filter-small2} 
    \mygraphics{scale=.5}{figures/emacs-filter-small2}}
  \hspace{2em}
  \subfloat{
    \label{subfig:emacs-filter-small3} 
    \mygraphics{scale=.5}{figures/emacs-filter-small3}}
\caption[Effect of hiding ``small'' changes and deletions]{Effect of hiding ``small'' changes first (middle) and then also deletions (right)} \label{fig:emacs-filter-small}
\end{figure}
Figures~\ref{fig:emacs-filter-small} show that ``\textbackslash maketitle'' as well as the typos in the word ``political'' are no longer marked up, and in the third image, the deletion beginning with ``If'' at the beginning of the paragraph is now omitted.

\subsection{Understanding the Commit Graph}

Now draw your attention back to LTC info buffer with the history of the current file under git (located at the bottom or right of your tracked file).  The Emacs representation is using small box characters to draw the graph and its edges.  In our current tutorial repository, there are no branches and the graph is a sequential line.  
\begin{figure}[t]
\centering
\mygraphics{scale=.5}{figures/emacs-commit-graph}
\caption{Example of commit graph} \label{fig:emacs-commit-graph}
\end{figure}
Refer to Figure~\ref{fig:emacs-commit-graph} for a screen shot of the example file history. Versions that are included in the tracked changes are not printed in gray.  How far we go back in history depends on some filtering settings, which are discussed further in Section~\ref{sec:emacs-limit-history} below.  By default, we first include all version of the current author at the top.  In our example with impersonating John Adams, there are currently no further recent commits of him.  Then, we continue down the path and collect all versions of different authors until we find the next version of John Adams in the commit with the message ``second version.''

\subsection{Limiting History} \label{sec:emacs-limit-history}

We allow the user to filter and customize how the potentially rich history of a .tex file is selected, so as to provide a better view of the tracked changes.  The user can show and hide changes as seen above, limit the authors of interest, and specify a date or revision number to tell LTC how far back in time the history should be considered.

%\subsubsection{Limiting by Authors}

\begin{figure}
\centering
\mygraphics{scale=.5}{figures/emacs-limit-authors}
\caption[Effect on commit graph of limiting authors]{Effect on commit graph of limiting authors to Roger Sherman and Thomas Jefferson} \label{fig:emacs-limit-authors}
\end{figure}
To limit the history by \textbf{authors}, choose menu item \Menu{Limit by;Set of authors...}{LTC}. This will prompt the user to enter author names to limit by in the mini buffer.  Again, automatic completion works, so you can enter \Code{Ro<TAB> <RET>} and \Code{Th<TAB> <RET> <RET>} to select authors Roger Sherman and Thomas Jefferson and exit the dialog.  After the last author was selected, the system automatically updates the displayed changes.

Notice how any version by the ignored authors Benjamin Franklin and John Adams is now gray as only commits from the selected authors are considered.  The first line in the LTC info buffer still shows the currently active author John Adams, so this line is not gray.  Again, the history is only taken until the next revision of the current author but since he is being ignored, we go all the way back to the first revision. Compare your Emacs now with the screen shot in Figure~\ref{fig:emacs-limit-authors} and see how the file history has changed.

To reset limiting by authors, simply choose the same menu \Menu{Limit by;Set of authors...}{LTC} again and enter an empty author as the first one.  Now the display is back in the original state.

%\subsubsection{Limiting by Date or Revision}

Next, we apply limits on \textbf{revision} and \textbf{date} to control how far back the history of the file is considered.  As we had seen, the first version is not taken into account because it was committed before the next commit by the current author John Adams.  Let us now choose menu item \Menu{Limit by;Start at revision...}{LTC}.  This will prompt the user to specify a known revision number.  Type the first few characters \Code{d6d<RET>} of the SHA-1 key of the first commit.  If unique, it is not necessary to expand the revision number using the TAB key (or whatever key is used for completion in your Emacs configuration).  See how the first version is listed in color and considered in the tracked changes above as seen in Figure~\ref{fig:emacs-limit-rev}.  Since changes by the current author John Adams from the first to the second version are now included, notice the text marked up in red appearing in the editor window. We see that John Adams must have added himself as an author in the LaTeX preamble among other edits in the second commit of the file. 

\begin{figure}
\centering
\mygraphics{scale=.5}{figures/emacs-limit-rev}
\caption{Effect on commit graph of going back to first revision} \label{fig:emacs-limit-rev}
\end{figure}

Another way of limiting by revision number is to simply left-click the number in the display.  

To remove the limit by revision number, simply choose the same menu \Menu{Limit by;Start at revision...}{LTC} again and enter an empty revision. Or, click into the empty revision column of the first line (denoting the currently active author) to achieve the same effect.  Now the display is back in the original state.

\begin{figure}
\centering
\mygraphics{scale=.5}{figures/emacs-limit-date}
\caption{Effect on commit graph of limiting history to date of third version} \label{fig:emacs-limit-date}
\end{figure}

Limiting the history by date works similarly.  Select menu item \Menu{Limit by;Start at date...}{LTC}. At the prompt, you can enter a date from the history of the file using auto-completion.  For example, enter \Code{2<TAB>11<TAB> <RET>} to get the exact date of the third revision.  Or, type a date such as \Code{Jul 23, 2010 1:11p} into the field.  We employ some software to process times and dates in natural language, and if successful, the field will contain the date string as it was understood translated into the format used in the commit tree. You may also perform a left-click on the date in the history to achieve the same effect.  See Figure~\ref{fig:emacs-limit-date} for a screen shot of the effect of limiting to the date of the third revision. 

To remove the limit by date, either left-click on the empty date column of the first line of the file history or enter an empty date after selecting menu item \Menu{Limit by;Start at date...}{LTC} again.

\subsection{Condensing History}

\begin{figure}
\centering
\mygraphics{scale=.5}{figures/emacs-condense-on}
\caption[Effect of condensing authors]{Effect of condensing authors: ignoring the fifth version by Roger Sherman} \label{fig:emacs-condense-on}  
\end{figure}
Sometimes the list of commits considered is getting long and the resulting markup of the changes confusing.  One additional way to customize how the changes are displayed is a setting to ``condense authors.''  Now check the menu \Menu{Condense authors}{LTC}.  Then, only the latest version of an author of \textit{consecutive} commits is considered -- in our example, only the sixth version is colored while the fifth version by Roger Sherman is now grayed out as seen in Figure~\ref{fig:emacs-condense-on}.

\begin{figure}
\centering
\subfloat{
  \label{subfig:emacs-condense-before} 
  \mygraphics{scale=.5}{figures/emacs-condense-before}}
\hspace{1em}
\subfloat{
  \label{subfig:emacs-condense-after} 
  \mygraphics{scale=.5}{figures/emacs-condense-after}}
\caption[Example of markup change when condensing authors]{Example of markup change before (left) and after (right) condensing authors} \label{fig:emacs-condense-before-after}
\end{figure}
See Figure~\ref{fig:emacs-condense-before-after} for an example of how condensing authors affects the markup.  Since we are only considering the changes that Roger Sherman made in the sixth version, his correction of the name is no longer shown.  Condensing authors makes sense when users commit versions often so that they do not loose too much history.  Their last version after a number of commits generally has the flavor of a ``final'' version, ready to be shared with others.  Hence, the changes made there compared to the last version of another author is commonly of most interest.

\subsection{Editing, Saving, and Committing}

Let us start the next step by resetting all filters to the default configuration, i.e., no limit by authors, date, and revision.  Then, we will edit the text in the top buffer to see the latest changes.

Click into the text buffer and enter some text, for example a LaTeX comment reminding John Adams to work on a list of charges against King George III in line nineteen:
\begin{FileVerbatim}
% list charges against King George III here
\end{FileVerbatim}
The added text will be rendered in red (or the color code for the current author) and underlined.  Notice how the commit graph displays the label ``modified'' in the revision column of the first line of the file history.  The mode line of Emacs also displays the symbol \Code{**} to denote a modified buffer. Now delete some of the characters you have just entered, for example the word \Code{here} at the end.  The characters simply disappear as they were added by the same author.

Now delete other characters that are either rendered black or a different color than red but not marked as deletions (inverse color).  Notice how these characters remain visible but are now colored red and marked up with inverse colors.  If you tried to delete anything that is already marked as deletion (i.e., anything in inverse color), nothing will happen as this text is already deleted in a prior version.  See Figure~\ref{fig:emacs-modified} for a screen shot of the above edits: text in red and underlined was added and text in inverse red was deleted.

\begin{figure}[t]
\centering
\mygraphics{scale=.35}{figures/emacs-modified}
\caption{After editing the text as John Adams} \label{fig:emacs-modified}
\end{figure}

Finally, you will want to save the current file to disk.  This will cause the label ``modified'' to change to ``on disk'' after saving in Emacs, for example with \Code{C-x s}.  If you would then again edit, the label would switch back to ``modified'' of course.

Saving the file, however, does not tell git to create a new version under its management.  In order to commit the current file to git, first, make sure that the .tex file is saved under Emacs and the first line of the commit graph does not say ``modified.'' Then, on a command line, switch to the directory with the tutorial file and perform the following commit command (printed in bold below).  You may want to check the status of git before and after the commit:
\begin{CodeVerbatim}[commandchars=\\\{\}]
$> git status
# On branch master
# Changes not staged for commit:
#   (use "git add <file>..." to update what will be committed)
#   (use "git checkout -- <file>..." to discard changes in working directory)
#
#	modified:   independence.tex
#
no changes added to commit (use "git add" and/or "git commit -a")
$> \textbf{git commit -am "added comment about list of charges"}
[master 8629257] added comment about list of charges
 1 file changed, 3 insertions(+), 1 deletion(-)
$> git status
# On branch master
# Your branch is ahead of 'origin/master' by 1 commit.
#   (use "git push" to publish your local commits)
#
nothing to commit, working directory clean
\end{CodeVerbatim}

To make Emacs aware of the underlying commit, use menu item \Menu{Update buffer}{LTC} or use the command \Code{M-x ltc-update}.  Notice how the recent commit gets included at the top of the list as seen in Figure~\ref{fig:emacs-new-commit}.

\begin{figure}
\centering
\mygraphics{scale=.5}{figures/emacs-new-commit}
\caption{File history after committing latest version and updating Emacs} \label{fig:emacs-new-commit}
\end{figure}

\subsection{Collaborating with Roger Sherman} \label{sec:tutorial-git-emacs:collab}

Next we perform an example collaboration with Roger Sherman's example repository as setup in Section~\ref{sec:collaborating} above.  Remember that Roger Sherman's repository has had two more commits than the original one we used for John Adams.  Since we have made edits and commits as John Adams, both repositories have diverged.  To synchronize them, we first pull Roger Sherman's changes into our working copy after checking that we are in a good state:

\begin{CodeVerbatim}
$> git pull sherman master
remote: Counting objects: 8, done.
remote: Compressing objects: 100% (3/3), done.
remote: Total 6 (delta 2), reused 5 (delta 1)
Unpacking objects: 100% (6/6), done.
From $TUTORIAL/independence-sherman
 * branch            master     -> FETCH_HEAD
Auto-merging independence.tex
CONFLICT (content): Merge conflict in independence.tex
Automatic merge failed; fix conflicts and then commit the result.
\end{CodeVerbatim}

\begin{figure}
\centering
\mygraphics{scale=.35}{figures/emacs-merge-conflict}
\caption{Git conflict markers in merged file} \label{fig:emacs-merge-conflict}
\end{figure}
Unfortunately, the two repositories have diverged too much and a so-called ``merge conflict'' has arisen.  Now we have to tell git how to fix this before we can proceed.  
Next, we look at the markers that git has put into our file.  You can run \Code{M-x ltc-update} in Emacs to see these markers as seen in Figure~\ref{fig:emacs-merge-conflict}.  On the command line, the file looks similar to this:
\begin{CodeVerbatim}
$> cat independence.tex 
[...]

<<<<<<< HEAD
% list charges against King George III
=======
That to secure these rights, Governments are instituted among Men, [...]

%TODO: indictment here
>>>>>>> 39cd6172613d1065a4cddc854cf30067869fc727
\end{CodeVerbatim}

We decide that the comments in the \Code{HEAD} version means the same as the last comment in the merged version \Code{39cd617...} so we modify the text so that it looks like this:
\begin{CodeVerbatim}
$> cat independence.tex 
[...]

That to secure these rights, Governments are instituted among Men, [...]

% list charges against King George III
\end{CodeVerbatim}
It is important to remove the git marker lines starting with \Code{<<<<<<<}, \Code{=======}, and \Code{>>>>>>>} for git to recognize that we have resolved the conflicts.  Now committing on the command line yields:
\begin{CodeVerbatim}
$> git commit -am "merging Roger Sherman's edits"
[master 34c1bde] merging Roger Sherman's edits
\end{CodeVerbatim}

\begin{figure}
\centering
\mygraphics{scale=.35}{figures/emacs-merge-resolve}
\caption{After resolving conflict, committing, and updating in Emacs} \label{fig:emacs-merge-resolve}
\end{figure}
This has resolved the conflict and incorporated Roger Sherman's prior changes, as a look at the git log with the graphing function reveals:
\begin{CodeVerbatim}
$> git log --oneline --graph --date-order
*   34c1bde merging Roger Sherman's edits
|\  
* | 8629257 added comment about list of charges
| * 39cd617 todo item for indictment
| * 45710ff more text for preamble
|/  
* d3f904c sixth version
* 203e0ce fifth version
* 36eeab0 fourth version
* fa2be39 third version
* bac2f51 second version
* d6d1cf8 first version
\end{CodeVerbatim}
Once we update Emacs using for example \Code{M-x ltc-update}, we see the paragraph that was part of the conflicting region now correctly attributed to Roger Sherman.  Furthermore, the git commit graph has gotten more interesting with the branching and merging symbolized by parallel vertical lines and empty rows for connecting them at merge points.  Refer to Figure~\ref{fig:emacs-merge-resolve} for a screen shot of the git commit graph and text changes after resolving the conflict, committing and updating Emacs.

One thing to notice in the commit graph with branches is the grayed out row of the commit \Code{8629257} that we had performed just a little while ago as John Adams.  When we obtain the history of the file and one commit has more than one ancestor (a merge point), we walk the branch that has the older commit time in order to create a sequential path through the commits.  In the future, we want to allow the user to select the branch to use or even overlay parallel branches for better control of the system.  You can watch \href{http://sourceforge.net/p/latextrack/tickets/15/}{ticket \#15} for when we address this problem.

Next we make sure to tell Roger Sherman about our effort to merge changes in the current document.  He will want to pull our effort using the following commands.
\begin{CodeVerbatim}
$> cd $TUTORIAL/independence-sherman
$> git pull adams master
remote: Counting objects: 10, done.
remote: Compressing objects: 100% (4/4), done.
remote: Total 6 (delta 2), reused 0 (delta 0)
Unpacking objects: 100% (6/6), done.
From ../independence
 * branch            master     -> FETCH_HEAD
Updating 39cd617..34c1bde
Fast-forward
 independence.tex | 4 ++--
 1 file changed, 2 insertions(+), 2 deletions(-)
\end{CodeVerbatim}

A look at the git graph on the command line shows that the merge has been applied without conflict.
\begin{CodeVerbatim}
$> git log --oneline --graph --date-order
*   34c1bde merging Roger Sherman's edits
|\  
* | 8629257 added comment about list of charges
| * 39cd617 todo item for indictment
| * 45710ff more text for preamble
|/  
* d3f904c sixth version
* 203e0ce fifth version
* 36eeab0 fourth version
* fa2be39 third version
* bac2f51 second version
* d6d1cf8 first version
\end{CodeVerbatim}

Now both can continue working on their versions but to prevent future merge conflicts it would be wise if they told each other immediately about newer versions and devised a plan to edit the file at different times.  In larger writing projects, we recommend to break up the document into smaller .tex files to be included in a master .tex file.  Then, editing different files at the same time for multiple authors minimizes the risk of merge conflicts.
