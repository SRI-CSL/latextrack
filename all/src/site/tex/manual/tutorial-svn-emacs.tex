% !TEX root = manual.tex
\section{Tutorial with Subversion and Emacs} \label{sec:tutorial-svn-emacs}

In this section, we assume that the example svn repository has been created according to the instructions in Section~\ref{sec:example-svn} above.  The latter subsections require a local svn repository but the beginning can be done with either the remote example repository or a local one.  And we assume that LTC has been installed using the Emacs directory, as well as Emacs configuration adjustments made that are mentioned in Section~\ref{sec:config-emacs}.

\subsection{Starting LTC Server and \texttt{ltc-mode}}

First, we start the LTC Server from the command line.  Assuming you have installed LTC in the directory \Code{\$LTC}, we run this command line for the server.  The output will be similar to the following.  Leave the server running while performing the rest of this tutorial.
\begin{CodeVerbatim}
$> java -jar $LTC/LTC.jar
LaTeX Track Changes (LTC)  Copyright (C) 2009-2013  SRI International
This program comes with ABSOLUTELY NO WARRANTY; for details use command line switch -c.
This is free software, and you are welcome to redistribute it under certain conditions.

<current date> | CONFIG:  Logging configured to level CONFIG
<current date> | CONFIG:  LTC version: <version info>
<current date> | INFO:    Started RPC server on port 7777.
\end{CodeVerbatim}

Next, we switch to Emacs and open the tutorial file \Code{\$TUTORIAL/independence/independence.tex}.  This should put Emacs into \Code{latex-mode} but any other mode should work as well.  Then, start LTC mode using the command \Code{M-x ltc-mode} in Emacs.  Beware that using LTC with a remote subversion server takes longer than using git or a local subversion server, as we have to query the distant server hosting the repository for each version of the .tex file.  You will see a few messages appearing briefly in the mini buffer (you can also look at them in the \Code{*Messages*} buffer), such as the following.  While the potential time intensive task of downloading versions from the remote server happen, the mini buffer should say \Code{Starting LTC update...}, which turns into \Code{LTC updates received} when the process is done.
\begin{FileVerbatim}
Starting LTC mode for file "$TUTORIAL/independence/independence.tex"...
Using `xml-rpc' package version: 1.6.8.2
LTC session ID = 1
Starting LTC update...
LTC updates received
\end{FileVerbatim}
Emacs should now look similar to the screen shot in Figure~\ref{fig:svn-emacs-open}.
\begin{figure}[t]
\centering
\mygraphics{scale=.35}{figures/svn-emacs-open}
\caption{Starting \Code{ltc-mode} in Emacs with tutorial file under svn} \label{fig:svn-emacs-open}
\end{figure}
In this figure, we see the changes marked up in various colors and fonts: underlined for additions and inverse for deletions.  There is also a smaller buffer called ``LTC info (session $N$)'' with $N$ as the session ID, at the bottom (or the right, if your Emacs is in landscape mode) of the buffer with the tutorial file.  There, we display the history of the current file under svn.  We can also see what svn perceives as the current user at the top of the graph -- here the name of the local user.

First, we will override what LTC thinks is the current author in order to make the following tutorial more meaningful.  In real life situations you will rarely have to use this command as you typically want the changes in the repository attributed to yourself.  In Emacs, type the command \Code{M-x ltc-set-self<RET>adams<RET>} to impersonate John Adams.  This updates the contents in the main buffer and info buffer at the bottom automatically, which may again take a little time with a remote subversion server.  The info buffer will then look like the screen shot in Figure~\ref{fig:svn-adams}.
\begin{figure}[t]
\centering
\mygraphics{scale=.5}{figures/svn-adams}
\caption{Emacs info buffer after setting current author to ``adams''} \label{fig:svn-adams}
\end{figure}

\subsection{Showing and Hiding Certain Changes}

The LTC menu and ``LTC info'' buffer in Emacs allow us to customize the way LTC displays the changes of the file.  Section~\ref{sec:general-use} contains all the details of how LTC displays the changes including limiting the file history and filtering.  In this tutorial, we will just use some of the options and see their effect.

First, notice the colors assigned to each of the authors.  To change an author color, for example Roger Sherman's,  perform a single left-click on the name of Roger Sherman.  This opens another buffer called \Code{*Colors*} with a preview of colors and their names.  Also look at the mini buffer that requests input.   You can enter a name or an RGB value in hex notation.  The color names can also be auto-completed, for example type \Code{Bro<TAB>} (if TAB is your completion key in Emacs) to see \Code{Brown}.  You will want something with contrast to the white background, so brown is a fine choice.  When clicking the RETURN key, notice how the text in the editor panel on the top changes color for those parts that are attributed to Roger Sherman's edits.  To abort choosing a color simply enter an empty value.

Next, focus on the typographical errors in the command ``\textbackslash maketitle'' in line 11 and the beginning of the first paragraph in line thirteen as well as the spelling errors in the word ``political.''  Open the LTC menu (in the menu bar and in the mode line) and then the sub-menu ``Show/Hide'' as seen in Figure~\ref{fig:svn-emacs-LTC-menu}.  
\begin{figure}[t]
\centering
\mygraphics{scale=.35}{figures/svn-emacs-LTC-menu}
\caption{Opening the LTC menu from the mode line in Emacs} \label{fig:svn-emacs-LTC-menu}
\end{figure}
If you first uncheck the item \Menu{Show/Hide;Show small changes}{LTC}, and second, also the item \Menu{Show/Hide;Show deletions}{LTC}, notice how the text rendering in the editor panel changes.  Again, if you work with the remote repository, it might take a little while until all the updates are received from the server and the mini buffer shows the message ``LTC updates received.''
\begin{figure}[t]
  \centering
  \subfloat{
    \label{subfig:svn-emacs-filter-small1} 
    \mygraphics{scale=.5}{figures/svn-emacs-filter-small1}}
  \hspace{2em}
  \subfloat{
    \label{subfig:svn-emacs-filter-small2} 
    \mygraphics{scale=.5}{figures/svn-emacs-filter-small2}}
  \hspace{2em}
  \subfloat{
    \label{subfig:svn-emacs-filter-small3} 
    \mygraphics{scale=.5}{figures/svn-emacs-filter-small3}}
\caption[Effect of hiding ``small'' changes and deletions]{Effect of hiding ``small'' changes first (middle) and then also deletions (right)} \label{fig:svn-emacs-filter-small}
\end{figure}
Figures~\ref{fig:svn-emacs-filter-small} show that ``\textbackslash maketitle'' as well as the typos in the word ``political'' are no longer marked up, and in the third image, the deletion beginning with ``If'' at the beginning of the paragraph is now omitted.

\subsection{Understanding the Commit Graph}

Now draw your attention back to LTC info buffer with the history of the current file under svn (located at the bottom or right of your .tex file).  The Emacs representation is using small box characters to draw the graph and its edges.  In our current tutorial repository, there are no branches and the graph is a sequential line.  

Refer back to Figure~\ref{fig:svn-adams} for the screen shot of the example file history. Versions that are included in the tracked changes are not printed in gray.  How far we go back in history depends on some filtering settings, which are discussed further in Section~\ref{sec:svn-emacs-limit-history} below.  By default, we first include all version of the current author at the top.  In our example with impersonating John Adams with the user name ``adams,'' there are currently no further recent commits of him.  Then, we continue down the path and collect all versions of different authors until we find the next version of John Adams in the commit with the message ``second version.''

\subsection{Limiting History} \label{sec:svn-emacs-limit-history}

We allow the user to filter and customize how the potentially rich history of a .tex file is selected, so as to provide a better view of the tracked changes.  The user can show and hide changes as seen above, limit the authors of interest, and specify a date or revision number to tell LTC how far back in time the history should be considered.

\begin{figure}
\centering
\mygraphics{scale=.5}{figures/svn-emacs-limit-authors}
\caption[Effect on commit graph of limiting authors]{Effect on commit graph of limiting authors to ``sherman'' and ``jefferson''} \label{fig:svn-emacs-limit-authors}
\end{figure}
To limit the history by \textbf{authors}, choose menu item \Menu{Limit by;Set of authors...}{LTC}. This will prompt the user to enter author names to limit by in the mini buffer.  Again, automatic completion works, so you can enter \Code{sh<TAB> <RET>} and \Code{je<TAB> <RET> <RET>} to select authors ``sherman'' and ``jefferson'' and exit the dialog.  After the last author was selected, the system automatically updates the displayed changes.

Notice how any version by the ignored authors ``franklin'' and ``adams'' is now gray as only commits from the selected authors are considered.  The first line in the LTC info buffer still shows the currently active author ``adams,'' so this line is not gray.  Again, the history is only taken until the next revision of the current author but since he is being ignored, we go all the way back to the first revision. Compare your Emacs now with the screen shot in Figure~\ref{fig:svn-emacs-limit-authors} and see how the file history has changed.

To reset limiting by authors, simply choose the same menu \Menu{Limit by;Set of authors...}{LTC} again and enter an empty author as the first one.  Now the display is back in the original state.

Next, we apply limits on \textbf{revision} and \textbf{date} to control how far back the history of the file is considered.  As we had seen, the first version is not taken into account because it was committed before the next commit by the current author John Adams.  Let us now choose menu item \Menu{Limit by;Start at revision...}{LTC}.  This will prompt the user to specify a known revision number.  Type \Code{1} as the revision number of the first commit and hit ENTER.  See how the first version is now listed in color and considered in the tracked changes above as seen in Figure~\ref{fig:svn-emacs-limit-rev}.  Since changes by the current author John Adams from the first to the second version are now included, notice the text marked up in red appearing in the editor window. We see that John Adams must have added himself as an author in the LaTeX preamble among other edits in the second commit of the file. 

\begin{figure}
\centering
\mygraphics{scale=.5}{figures/svn-emacs-limit-rev}
\caption{Effect on commit graph of going back to first revision} \label{fig:svn-emacs-limit-rev}
\end{figure}

Another way of limiting by revision number is to simply left-click the number in the display of the commit graph.  

To remove the limit by revision number, simply choose the same menu \Menu{Limit by;Start at revision...}{LTC} again and enter an empty revision. Or, click into the empty revision column of the first line (denoting the currently active author) to achieve the same effect.  Now the display is back in the original state.

\begin{figure}
\centering
\mygraphics{scale=.5}{figures/svn-emacs-limit-date}
\caption{Effect on commit graph of limiting history to date of third version} \label{fig:svn-emacs-limit-date}
\end{figure}

Limiting the history by date works similarly.  Select menu item \Menu{Limit by;Start at date...}{LTC}. At the prompt, you can enter a date from the history of the file using auto-completion.  For example, enter \Code{2<TAB>2:59:0<TAB> <RET>} to get the exact date of the third revision.  Or, type a date such as \Code{Nov 13, 2012 12:59p} (should yield the same results if you are in the Central Time Zone) into the field.  We employ some software to process times and dates in natural language, and if successful, the field will contain the date string as it was understood translated into the format used in the commit tree. You may also perform a left-click on the date in the history to achieve the same effect.  See Figure~\ref{fig:svn-emacs-limit-date} for a screen shot of the effect of limiting to the date of the third revision. 

To remove the limit by date, either left-click on the empty date column of the first line of the file history or enter an empty date after selecting menu item \Menu{Limit by;Start at date...}{LTC} again.

\subsection{Condensing History}

\begin{figure}
\centering
\mygraphics{scale=.5}{figures/svn-emacs-condense-on}
\caption[Effect of condensing authors]{Effect of condensing authors: ignoring the fifth version by Roger Sherman} \label{fig:svn-emacs-condense-on}  
\end{figure}
Sometimes the list of commits considered is getting long and the resulting markup of the changes confusing.  One additional way to customize how the changes are displayed is a setting to ``condense authors.''  Now check the menu \Menu{Condense authors}{LTC}.  Then, only the latest version of an author of \textit{consecutive} commits is considered -- in our example, only the sixth version is colored while the fifth version by Roger Sherman is now grayed out as seen in Figure~\ref{fig:svn-emacs-condense-on}.

\begin{figure}
\centering
\subfloat{
  \label{subfig:svn-emacs-condense-before} 
  \mygraphics{scale=.5}{figures/svn-emacs-condense-before}}
\hspace{1em}
\subfloat{
  \label{subfig:svn-emacs-condense-after} 
  \mygraphics{scale=.5}{figures/svn-emacs-condense-after}}
\caption[Example of markup change when condensing authors]{Example of markup change before (left) and after (right) condensing authors} \label{fig:svn-emacs-condense-before-after}
\end{figure}
See Figure~\ref{fig:svn-emacs-condense-before-after} for an example of how condensing authors affects the markup.  Since we are only considering the changes that Roger Sherman made in the sixth version, his correction of the name is no longer shown.  Condensing authors makes sense when users commit versions often so that they do not loose too much history.  Their last version after a number of commits generally has the flavor of a ``final'' version, ready to be shared with others.  Hence, the changes made there compared to the last version of another author is commonly of most interest.

\subsection{Editing and Saving} \label{sec:svn-editing}

Let us start the next step by resetting all filters to the default configuration, i.e., no limit by authors, date, and revision.  Then, we will edit the text in the top buffer to see the latest changes.

Click into the text buffer and enter some text, for example a LaTeX comment reminding John Adams to work on a list of charges against King George III in line nineteen:
\begin{FileVerbatim}
% list charges against King George III here
\end{FileVerbatim}
The added text will be rendered in red (or the color code for the current author) and underlined.  Notice how the commit graph displays the label ``modified'' in the revision column of the first line of the file history.  The mode line of Emacs also displays the symbol \Code{**} to denote a modified buffer. Now delete some of the characters you have just entered, for example the word \Code{here} at the end.  The characters simply disappear as they were added by the same author.

Now delete other characters that are either rendered black or a different color than red but not marked as deletions (inverse color).  Notice how these characters remain visible but are now colored red and marked up with inverse colors.  If you tried to delete anything that is already marked as deletion (i.e., anything in inverse color), nothing will happen as this text is already deleted in a prior version.  See Figure~\ref{fig:svn-emacs-modified} for a screen shot of the above edits: text in red and underlined was added and text in inverse red was deleted.

\begin{figure}[t]
\centering
\mygraphics{scale=.35}{figures/svn-emacs-modified}
\caption{After editing the text as ``adams''} \label{fig:svn-emacs-modified}
\end{figure}

Finally, you will want to save the current file to disk.  This will cause the label ``modified'' to change to ``on disk'' after saving in Emacs, for example with \Code{C-x s}.  If you would then again edit, the label would switch back to ``modified'' of course.

\subsection{Collaborating Through Commits}

In Subversion, your repository is a central entity that all collaborators commit to.  Therefore, unlike distributed version control systems such as git, the collaboration happens when users commit their version.  It is a good practice to update your working copy regularly to avoid conflicts during committing.  Furthermore, users should communicate with each other to decide who is editing what file at a time.

The following assumes that you are working with a local svn repository per the setup in Sections~\ref{sec:example-svn-local} and {sec:svn-collaborating} above.  Next, we will simulate that Roger Sherman commits his modified version before John Adams can upload his version, resulting in a merge conflict.  To prepare this scenario, first commit the modifications from Roger Sherman's working copy:
\begin{CodeVerbatim}[commandchars=\\\{\}]
$> cd $TUTORIAL/independence-sherman/
$> svn status
M       independence.tex
$> svn commit -m "todo item for indictment" --username sherman 
Sending        independence.tex
Transmitting file data .
Committed revision 7.
\end{CodeVerbatim}

Now the shared repository is already at revision 7 while we (as John Adams) are still editing from revision six.  First, check again that you have saved the file in Emacs and that the first entry in the commit graph says ``on disk.'' When we try to commit our latest changes from the Section~\ref{sec:svn-editing} above, using the command line, we get the following error message.  
\begin{CodeVerbatim}[commandchars=\\\{\}]
$> cd $TUTORIAL/independence/
$> svn status
M       independence.tex
$> svn commit -m "added comment about list of charges" --username adams
Sending        independence.tex
Transmitting file data .svn: Commit failed (details follow):
svn: File '/tutorial-svn/independence.tex' is out of date
\end{CodeVerbatim}

Then, we try to update our repository first to mend the situation, which results in another error message about the conflicting versions.  If Roger Sherman and John Adams had been modifying different .tex files in the same repository, we would have not gotten this conflict.  To resolve, we choose to postpone so that we can see the differences in Emacs and resolve it there.  Our input is marked in bold below.
\begin{CodeVerbatim}[commandchars=\\\{\}]
$> svn update
Conflict discovered in 'independence.tex'.
Select: (p) postpone, (df) diff-full, (e) edit,
        (mc) mine-conflict, (tc) theirs-conflict,
        (s) show all options: \textbf{p}
C    independence.tex
Updated to revision 7.
Summary of conflicts:
  Text conflicts: 1
\end{CodeVerbatim}

\begin{figure}
\centering
\mygraphics{scale=.35}{figures/svn-emacs-merge-conflict}
\caption{Subversion conflict markers in merged file} \label{fig:svn-emacs-merge-conflict}
\end{figure}
Our current file is now marked as in conflict, so let us update Emacs using menu item \Menu{Update buffer}{LTC} or \Code{M-x ltc-update}, to see something similar to the screen shot in Figure~\ref{fig:svn-emacs-merge-conflict}.  The conflicting portion at the end is marked with additional lines and we see revision 7 in the history of the file.  On the command line, the file looks similar to this:
\begin{CodeVerbatim}
$> cat independence.tex 
[...]

<<<<<<< .mine
% list charges against King George III

=======
That to secure these rights, Governments are instituted among Men, [...]
%TODO: indictment here

>>>>>>> .r7
\end{CodeVerbatim}

We decide that the comments in the version \Code{.mine} version means the same as the last comment in version \Code{.r7} so we modify the text in Emacs and save, so that it looks like this on the command line:
\begin{CodeVerbatim}
$> cat independence.tex 
[...]

That to secure these rights, Governments are instituted among Men, [...]

% list charges against King George III
\end{CodeVerbatim}
It is important to remove the svn marker lines starting with \Code{<<<<<<<}, \Code{=======}, and \Code{>>>>>>>} for svn to recognize that we have resolved the conflicts. We also have to tell svn that the conflict has been resolved.  Then we can finally  perform the following commit command.  The two important commands are printed in bold below.  
You may want to check the status of svn before and after the commit:
\begin{CodeVerbatim}[commandchars=\\\{\}]
$> \textbf{svn resolved independence.tex}
Resolved conflicted state of 'independence.tex'
$> svn status
M       independence.tex
$> \textbf{svn commit -m "added comment about list of charges" --username adams}
Sending        independence.tex
Transmitting file data .
Committed revision 8.
$> svn status
\end{CodeVerbatim}
\begin{figure}
\centering
\mygraphics{scale=.35}{figures/svn-emacs-merge-resolve}
\caption{After resolving conflict and updating in Emacs} \label{fig:svn-emacs-merge-resolve}
\end{figure}
Once we update Emacs using for example \Code{M-x ltc-update}, we see the latest revision 8 in the commit graph and the marked up latest edits attributed to John Adams and Roger Sherman similar to Figure~\ref{fig:svn-emacs-merge-resolve}.
