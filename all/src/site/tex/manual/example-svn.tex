% !TEX root = manual.tex
\section{Creating the Example Subversion Repository} \label{sec:example-svn}

This tutorial uses an example svn repository, which is either hosted on the Internet or on your local computer.  The first Section~\ref{sec:example-svn-remote} shows how to use a publicly accessible repository with an example file.  This is the quickest way to try out LTC with Subversion but you cannot commit new versions to this repository so we cannot go through such advanced topics in the later parts of the tutorial.  The second Section~\ref{sec:example-svn-local} shows how to create a local svn server and populates it with an example repository.  This takes a bit more time to setup but then you can go through more advanced topics such as committing to the repository.

\subsection{Using the Remote Subversion Repository} \label{sec:example-svn-remote}

To create the example file that is under remote svn version control, go into a directory of your choice (say \Code{\$TUTORIAL}) and do the following.  If the server causes a certificate alert, you can accept it permanently by using \Code{p} as shown in bold below.

\begin{CodeVerbatim}[commandchars=\\\{\}]
$> cd $TUTORIAL
$> svn co https://spartan.csl.sri.com/svn/public/LTC/tutorial-svn independence 
Error validating server certificate for 'https://spartan.csl.sri.com:443':
 - The certificate is not issued by a trusted authority. Use the
   fingerprint to validate the certificate manually!
Certificate information:
 - Hostname: spartan.csl.sri.com
 - Valid: from Fri, 03 May 2013 00:00:00 GMT until Sat, 03 May 2014 23:59:59 GMT
 - Issuer: Thawte, Inc., US
 - Fingerprint: f9:3f:b3:27:65:89:c9:af:bf:05:1b:5f:60:f0:8c:df:4d:bc:47:7e
(R)eject, accept (t)emporarily or accept (p)ermanently? \textbf{p}
A    independence/independence.tex
Checked out revision 6.
\end{CodeVerbatim}

Now change into the new directory and confirm that the file has six revisions in its history
\begin{CodeVerbatim}
$> cd independence/
$> svn log -q independence.tex 
------------------------------------------------------------------------
r6 | sherman | 2012-11-13 13:01:00 -0600 (Tue, 13 Nov 2012)
------------------------------------------------------------------------
r5 | sherman | 2012-11-13 13:00:35 -0600 (Tue, 13 Nov 2012)
------------------------------------------------------------------------
r4 | jefferson | 2012-11-13 12:59:45 -0600 (Tue, 13 Nov 2012)
------------------------------------------------------------------------
r3 | franklin | 2012-11-13 12:59:03 -0600 (Tue, 13 Nov 2012)
------------------------------------------------------------------------
r2 | adams | 2012-11-13 12:58:04 -0600 (Tue, 13 Nov 2012)
------------------------------------------------------------------------
r1 | jefferson | 2012-11-13 12:51:35 -0600 (Tue, 13 Nov 2012)
------------------------------------------------------------------------
\end{CodeVerbatim}

Unfortunately, we cannot accept changes to this repository so the tutorials based on svn do not cover how to commit new revisions and how to collaborate.  We advise to install a local svn server and repository per the instructions below or to go through the git-based tutorial to cover those points.

\subsection{Using a Local Subversion Repository} \label{sec:example-svn-local}

%TODO