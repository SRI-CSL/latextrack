%!TEX TS-program = xelatex
%!TEX encoding = UTF-8 Unicode
\documentclass[twoside]{report}

% ----- useful packages
\usepackage[letterpaper,verbose,xetex]{geometry}
\usepackage[cc]{titlepic} % for picture in title
\usepackage{graphicx} % for including figures
\usepackage{tikz} % for portable graphics
\usepackage{hhline} % for double \cline's in tables
\usepackage{fancyvrb} % for more elaborate verbatim envs 
\usepackage[parfill]{parskip} % paragraphs begin with empty line rather than indent
\usepackage{url} % for typesetting URL's
\usepackage{color}
\usepackage[%
  unicode,pdfencoding=auto, % to make xelatex work 
  hyperfootnotes=false, % if using "footnotes" package
  colorlinks=true, 
  bookmarks=true,bookmarksnumbered=true, % show numbered bookmarks
  pdfsubject={},pdftitle={},pdfauthor={Linda Briesemeister} % document info
]{hyperref} % more hyperlinks in PDF
\usepackage{caption}[2013/02/03]
\usepackage{fancyhdr}
\usepackage{ulem} % for strike-out font
\usepackage{pbox} % for line breaks in table cells
\usepackage{subfig}
\usepackage{xparse} % for \Menu command definition
\usepackage{hanging} % for hanging paragraphs
\usepackage{wrapfig}
\usepackage[subfigure]{tocloft} % sans-serif fonts in TOC etc.

%% distinguish tex4ht and xelatex...
\makeatletter
\@ifpackageloaded{tex4ht}{%
	\usepackage[utf8]{inputenc}
%	\usepackage{devng4ht}
}{%
	\usepackage{fontspec}
	\usepackage{xltxtra}
	\usepackage{xunicode} % unicode character macros 
	\setmainfont[Ligatures=TeX,Numbers=OldStyle]{TeX Gyre Bonum}
	\setsansfont[Ligatures=TeX,Numbers=OldStyle]{TeX Gyre Adventor}
	\setmonofont[Ligatures=TeX,Numbers=OldStyle,Scale=0.9]{TeX Gyre Cursor}
	\newfontface\lining[Numbers=Lining]{TeX Gyre Pagella}
	% headings in sans serif and bold 
	\usepackage[sf,bf]{titlesec}%
}
\makeatother

%% pgf/tikz customization
\usepgflibrary{shapes.arrows}
\usetikzlibrary{external} % to create tikz pictures in .pdf/.png for tex4ht
\usetikzlibrary{arrows,automata}
\usetikzlibrary{positioning}
\tikzsetexternalprefix{figures/} % all generated figures go into this subdirectory
\tikzset{
    external/system call={%
    xelatex \tikzexternalcheckshellescape
    -halt-on-error -interaction=batchmode --shell-escape
    -jobname "\image" "\texsource"}}
% using tikz pictures with tex4ht:
% (see: http://tex.stackexchange.com/questions/40135/htlatex-and-tikz-creates-sometimes-incorrect-svgs)
\makeatletter
\@ifpackageloaded{tex4ht}{%
  \tikzexternalize[mode=only graphics]
  \tikzset{png export/.style={/pgf/images/external info,/pgf/images/include external/.code={%
    \includegraphics[width=\pgfexternalwidth,height=\pgfexternalheight]{##1.png}%
  }}}
  \tikzset{png export}% 
}{% else
  \tikzexternalize[mode=list and make]%
  \tikzset{pdf export/.style={/pgf/images/external info,/pgf/images/include external/.code={%
    \includegraphics[width=\pgfexternalwidth,height=\pgfexternalheight]{##1.pdf}%
  }}}
  \tikzset{pdf export}%
}
\makeatother

%% \nameref without hyperlink (to be used within \hyperref etc):
\makeatletter
\@ifdefinable\nolinknameref{%
  \DeclareRobustCommand*\nolinknameref[1]{%
    \csname @safe@activestrue\endcsname%
    \expandafter\real@setref%
    \csname r@#1\endcsname\@thirdoffive{#1}%
    \csname @safe@activesfalse\endcsname}%
}%
\makeatother

%% version string
\newcommand{\version}{}
\input{version.txt}

%% base URL of web site
\newcommand{\baseurl}{}
\input{baseurl.txt}

%% specifying different links for PDF vs. HTML
% 1 - URL for pdf, 2 - URL for html, 3 - text of link
\makeatletter
\@ifpackageloaded{tex4ht}{%
  \newcommand{\hlink}[3]{\href{#2}{#3}}%
}{% else
  \newcommand{\hlink}[3]{\href{#1}{#3}}%
}
\makeatother

%% custom verbatim commands and environments (PDF vs. HTML)
%\DefineVerbatimEnvironment{BaseVerbatim}{Verbatim}{fontsize=\normalsize} % common definitions to PDF and HTML
\definecolor{BoxBackground}{gray}{0.1} % 10%
\makeatletter
\@ifpackageloaded{tex4ht}{%
  \DefineVerbatimEnvironment{CodeVerbatim}{Verbatim}{frame=none}
  \DefineVerbatimEnvironment{FileVerbatim}{Verbatim}{frame=lines}
  \newcommand{\Code}[1]{{\tt #1}}%
}{% else
  \DefineVerbatimEnvironment{CodeVerbatim}{Verbatim}{fontsize=\small,frame=single}
  \DefineVerbatimEnvironment{FileVerbatim}{Verbatim}{fontsize=\small,frame=lines}
  \newcommand{\Code}[1]{{\small\tt #1}}%
}
\makeatother

%% figures under PDF and HTML
%\graphicspath{{./figures/}}
\DeclareGraphicsExtensions{.png}
\makeatletter
\@ifpackageloaded{tex4ht}{%
  \newcommand{\mygraphics}[2]{\includegraphics{#2}}%
}{% else
  \newcommand{\mygraphics}[2]{\includegraphics[#1]{#2}}%
}
\makeatother

%% LaTeX pseudoconditional to distinguish HTML and PDF generation in document body:
% see: http://tex.stackexchange.com/a/53943/12211
% use: \iftexforht{<code for TeX4ht>}{<code when TeX4ht isn't used>}
\makeatletter
\@ifpackageloaded{tex4ht}
  {\let\iftexforht\@firstoftwo}
  {\let\iftexforht\@secondoftwo}
\makeatother

%% captions customization
\captionsetup{font={small,sf},format=hang,justification=RaggedRight}

%% fancy headers and footers
% clear default layout:
\fancyhead{}
\fancyfoot{}
% handle lines:
\renewcommand{\headrulewidth}{0.2pt} 
\renewcommand{\footrulewidth}{0pt}
% custom header: chapter on left, section on right for odd; even reversed
\fancyhead[LE,RO]{\sffamily\nouppercase\rightmark}
\fancyhead[LO,RE]{\sffamily\nouppercase\leftmark}
% custom footer: version -- copyright -- page number for odd; even reversed
\fancyfoot[LO,RE]{\small{LTC\lining~v\version}}
\fancyfoot[C]{\small{\textcopyright~SRI International}}
\fancyfoot[RO,LE]{\sffamily\small{\thepage}}

%% TOC fonts etc.
% see also http://www.khirevich.com/latex/font/
% ToC
\renewcommand{\cfttoctitlefont}{\sffamily\Huge\bfseries} % ToC title            
\renewcommand{\cftchapfont}{\sffamily\bfseries}
\renewcommand{\cftchappagefont}{\sffamily\bfseries}
\renewcommand{\cftsecfont}{\sffamily}
\renewcommand{\cftsecpagefont}{\sffamily}
\renewcommand{\cftsubsecfont}{\sffamily}
\renewcommand{\cftsubsecpagefont}{\sffamily}
% LoF
\renewcommand{\cftloftitlefont}{\sffamily\Huge\bfseries} % LoF title            
\renewcommand\cftfignumwidth{3em}
\renewcommand{\cftfigfont}{\sffamily}
\renewcommand{\cftfigpagefont}{\sffamily}

%% colors
\definecolor{LightGray}{gray}{0.8}

%% type-setting menu items
\NewDocumentCommand\Menu{>{\SplitList{;}}m m}
{%
	\fbox{\small #2}%	
	\ProcessList{#1}{\insertitem}%
	$\,$% add a tiny bit of space     
}
\newcommand\insertitem[1]{\ $\triangleright$\ \fbox{\small #1}}


% ----- title information
\title{The \LaTeX{} Track Changes Manual \\ {\large\lining v\version}}
\author{%
Linda Briesemeister\\
\texttt{linda.briesemeister@sri.com}\\
SRI International
}
\titlepic{\hlink{\baseurl /index.html}{../../index.html}{\includegraphics[width=2cm]{figures/LTC-logo.png}}}
\date{\today} 

\begin{document}
\maketitle
%{\pagestyle{empty}\cleardoublepage} % empty page after title (since we are using 'twopage' option)

\chapter*{Acknowledgements and Disclaimer}

This project would not exist without Peter Karp, who had the original idea to bring track changes to the LaTeX world.  My colleagues Grit Denker and Tomer Altman have also been involved in furthering this project.  Recently, Skip Breidbach has joined the development effort.  Finally, we thank everyone who tested the prototype and gave us feedback and SRI International to provide the funding to pursue this project.

%Disclaimer
Please note that LaTeX Track Changes (LTC) is free software: you can redistribute it and/or modify it
under the terms of the GNU General Public License as published by the Free
Software Foundation, either version 3 of the License, or (at your option)
any later version. This program is distributed in the hope that it will be
useful, but WITHOUT ANY WARRANTY; without even the implied warranty of
MERCHANTABILITY or FITNESS FOR A PARTICULAR PURPOSE.  See the GNU General
Public License for more details.

See Appendix~\ref{ch:license} or \url{http://www.gnu.org/licenses/} for the complete license.

\thispagestyle{plain}
\clearpage % due to tocloft package
\tableofcontents
\thispagestyle{empty}

\iftexforht{}{% else
  \clearpage % due to tocloft package
  \listoffigures % list of figures only in PDF version
  \thispagestyle{empty}%
}

% for the first page of each chapter adjust 'plain' style:
\fancypagestyle{plain}{%
  \fancyhf{} % clear all header and footer fields
  \fancyfoot[LO,RE]{\small{LTC\lining~v\version}}
  \fancyfoot[C]{\small{\textcopyright~SRI International}}
  \fancyfoot[RO,LE]{\sffamily\small{\thepage}}
}
\pagestyle{fancy}
% !TEX root = manual.tex
\chapter{Installation \& Configuration} \label{ch:install}

\section{Requirements}

We have tested the system on Mac OS X and Linux. It is designed to run on Unix platforms. Windows is not supported although it may run with a Unix-like shell under Windows.

Further requirements for running LTC are as follows.
\begin{description}
\item[Java] 1.6 or above.  If using the \Code{jar} tool to extract Emacs Lisp files, you will need the JDK, otherwise the JRE should suffice.
\item[Version Control System]  One of the version control systems below:
  \begin{description}
  \item[git] 1.7.2 or above.
  \item[svn] 1.6.3 or above.
  \end{description}
\item[Emacs] 23 or above.  Only if the user wants to use \Code{ltc-mode} in Emacs to interact with LTC.  XEmacs is not supported.
\end{description}

\section{Installation and Update}

Currently, we provide a shell script \Code{ltc-install.sh} to perform LTC installation and updates.  It can be downloaded from \url{http://sourceforge.net/projects/latextrack/files/}.  Use it to install the LTC JAR file in a location of your choice.  Also, if you use the Emacs \Code{ltc-mode}, be prepared to supply the location where to put the Emacs Lisp files (see more details below in Section~\ref{sec:config-emacs}).

In the future, we may provide installers for the target platforms.

Let us assume for the remainder of this manual, that you have used a directory called \Code{\$LTC} as the installation location. Then, you would install LTC using the script in the following way.  Note that you will need a second argument to install LTC for Emacs as explained in more detail below (Section~\ref{sec:config-emacs}).  It doesn't hurt to run the script twice, as it will replace LTC with the latest version.

\begin{CodeVerbatim}
$> bash ltc-install.sh -h
[... prints help message about using LTC install script]
$> bash ltc-install.sh $LTC
[...]
Done with installing LTC in $LTC
To start LTC server with default options, use the following command:

  java -jar $LTC/LTC.jar

\end{CodeVerbatim}

To update from an earlier LTC version, run the \Code{ltc-install.sh} script again with the same argument(s).  The script will download the latest version from the web site and configure the link in \Code{\$LTC/LTC.jar} so that future invocations  will resolve to the newest version.

After installing LTC, you can look at the command line options of LTC Server using the switch \Code{-h} or omit the switch to start LTC Server with default values.
\begin{CodeVerbatim}
$> java -jar $LTC/LTC.jar -h
[... prints help message about using LTC Server]
\end{CodeVerbatim}

\section{Configuration}

This section contains details of configuring git or svn, LTC and Emacs to work together.  These steps typically only need to be carried out once per installation of LTC.

Note that the system decides automatically whether your LaTeX file is under git or svn version control.

\subsection{General Git Configuration}

The current version of LTC supports git as an underlying version control system, which manages the history of the LaTeX files of interest. If you are already using git for other things, you may skip the following few steps as your git is probably already configured. However, we do recommend to add the common LaTeX build products with wildcards to the list of ignored files as outlined at the end of this section, which may not be configured if git has not been used to manage repositories with LaTeX files.

First, test that the git installation is found. Otherwise, you may want to add the git binary directory to the \Code{PATH} environment variable.  
\begin{CodeVerbatim}
$> git --version
git version 1.7.2
\end{CodeVerbatim}

If you haven't done already, configure git with your name and email address:
\begin{CodeVerbatim}
$> git config --global user.name "John Doe"
$> git config --global user.email doe@inter.net
\end{CodeVerbatim}

Typically, you don't want to track automatic backups and build products of your LaTeX project, so create a file \Code{\char`\~/.gitignore\_global} (or any name and location of your choice) and add the following lines as contents. 
\begin{CodeVerbatim}
*~
*.out
*.aux
*.bbl
*.blg
*.bst
*.dvi
*.idx
*.lof
*.log
*.toc
*.lol
*.lot
\end{CodeVerbatim}
Then, issue the git config command below (with a possibly adjusted file name and location).

\begin{CodeVerbatim}
$> git config --global core.excludesfile ~/.gitignore_global
\end{CodeVerbatim}

To learn how to set up a writing project under a git repository for using it with LTC refer to Section~\ref{sec:git-use}.

\subsection{General Subversion Configuration}

First, test that the svn installation is found. Otherwise, you may want to add the svn binary directory to the \Code{PATH} environment variable.  
\begin{CodeVerbatim}
$> svn --version
svn, version 1.6.18 (r1303927)
   compiled Aug  4 2012, 19:46:53

...
\end{CodeVerbatim}

[TODO: write more about subversion configuration here!]

%\subsection{LTC Server}

\subsection{Emacs \texttt{ltc-mode}} \label{sec:config-emacs}

To use the supplied \Code{ltc-mode} in Emacs, you will have to put the relevant mode files into a directory where Emacs can load them. There are two alternatives of letting Emacs know where to find Emacs Lisp files:
\begin{enumerate}
\item Use a location that is already included in the \Code{load-path}. To view the contents of this path in your Emacs, execute the command \Code{C-h v load-path}.  On Mac OS X systems with Aquamacs, this could be for example \Code{\char`\~/Library/Preferences/Emacs}.
\item Add a new directory where you will extract the Emacs Lisp files to the \Code{load-path}. Assuming the Emacs Lisp files will be installed in directory \Code{\char`\~/.emacs.d/}, add the following line to your \Code{.emacs} or other Emacs configuration file:
  \begin{CodeVerbatim}
(add-to-list 'load-path "~/.emacs.d")
  \end{CodeVerbatim}
\end{enumerate}

Now based on which method of the above you choose, supply the directory \Code{\$EMACS\_DIR} as the second argument to the install script:
\begin{CodeVerbatim}
$> bash ltc-install.sh $LTC $EMACS_DIR
\end{CodeVerbatim}

In order to enable the LTC mode in Emacs, add the following line to your \Code{.emacs} or other Emacs configuration file (for example, \Code{\char`\~/Library/Preferences/Emacs/Preferences.el} is the default for Aquamacs under Mac OS X) :
\begin{CodeVerbatim}
(autoload 'ltc-mode "ltc-mode" "" t)
\end{CodeVerbatim}

We recommend to avoid adding a hook from \Code{latex-mode} to \Code{ltc-mode} (usually done with \Code{add-hook}) as our mode requires the \Code{latex-mode} to be fully executed before it works. The hooks are not guaranteed to be executed in particular order, so it is best to manually invoke LTC mode after you have opened a \Code{.tex} file.

\begin{figure}[t]
\centering
\mygraphics{width=\textwidth,height=.5\textheight,keepaspectratio}{figures/emacs-port}
\caption{Customizing LTC port number in Emacs} \label{fig:emacs-port}
\end{figure}

If you need to change the port number that Emacs uses to communicate with the LTC Server (for example, if the default number is already in use on your computer), you first have to load \Code{ltc-mode} at least once (possibly with a failure) using command \Code{M-x ltc-mode}.  Then, you can view the current port setting using \Code{C-h v ltc-port <RET>}.  You can customize the port number using \Code{M-x customize-group <RET> ltc <RET>} or open the customization buffer and browse to the LTC group under the Tex group, which may be located under the Wp (word processing) top-level group.  See Figure~\ref{fig:emacs-port} for a screenshot when customizing the port number using the customization buffer in Aquamacs under Mac OS X.

%\subsubsection{Git from Emacs}
%
%The user can interact with git from within Emacs.  One of the many implementations is called ``magit'' mode.

\section{Troubleshooting}

We are keeping a list of \hlink{\baseurl /faq.html}{../../faq.html}{frequently-asked-questions} at the project's web site that may help for troubleshooting.  There will also be mailing lists and a way to report bugs available in the future.


% !TEX root = manual.tex
\chapter{Tutorials} \label{ch:tutorials}

This chapter contains a number of tutorials adjusted to the user's preference of text editor and version control system.  The following matrix allows to easily identify the best fit for your situation and contains links to the respective tutorial sections.  Whether you are using git or svn as a version control system, you will want to visit the respective section to setup your example repository first, which is linked from the gray column in the table below.

{
\setlength{\aboverulesep}{0pt} % remove space around rules as it will not be colored
\setlength{\belowrulesep}{0pt}
\setlength{\extrarowheight}{.75ex} % increase row height to compensate
\begin{tabular}{l>{\columncolor{LightGray}}ccc}
\toprule
 & Example & Emacs & LTC Editor \\
 & Repository & & \\
\midrule 
\Code{git} & Sec.~\ref{sec:example-git} & Sec.~\ref{sec:tutorial-git-emacs} & Sec.~\ref{sec:tutorial-git} \\
\Code{svn} & Sec.~\ref{sec:example-svn} & Sec.~\ref{sec:tutorial-svn-emacs} & Sec.~\ref{sec:tutorial-svn} \\
\bottomrule
\end{tabular}
}

% !TEX root = manual.tex
\section{Creating the Example Git Repository} \label{sec:example-git}

The git-based tutorials use two example git repositories called ``independence.bundle'' and ``independence-sherman.bundle,'' which can be downloaded from 
\url{http://sourceforge.net/projects/latextrack/files/examples/}.
First, save the bundled repositories into a directory of your choice.  We call this directory \Code{\$TUTORIAL}.  Then, clone from the first bundle to obtain a valid git working tree.
\begin{CodeVerbatim}
$> cd $TUTORIAL/
$> git clone independence.bundle independence
Cloning into 'independence'...
Receiving objects: 100% (18/18), done.
Resolving deltas: 100% (4/4), done.
$> cd independence/
$> git status
# On branch master
nothing to commit (working directory clean)
$> git log --oneline
d3f904c sixth version
203e0ce fifth version
36eeab0 fourth version
fa2be39 third version
bac2f51 second version
d6d1cf8 first version
\end{CodeVerbatim}

Now we impersonate John Adams to work on this writing project for the Declaration of Independence.

\begin{CodeVerbatim}
$> git config --add user.name "John Adams"
$> git config --add user.email "adams@usa.gov"
$> git config --list | grep -e "[Aa]dams"
user.name=John Adams
user.email=adams@usa.gov
\end{CodeVerbatim}

Another way to investigate the current git repository are graphical tools such as gitk (comes with git distribution) or GitX under Mac OS X.  Note that GitX is not required to run LTC.  Figure~\ref{fig:gitx-screen} for using GitX on the just created repository.
\begin{figure}[t]
\centering
\mygraphics{width=\textwidth,height=.5\textheight,keepaspectratio}{figures/gitx-screen}
\caption{Investigating example git repository with a graphical tool such as GitX} \label{fig:gitx-screen}
\end{figure}

The other point to note here is the way that GitX displays the changes in the file \Code{independence.tex} when using the graphical git interface.  It shows the lines in the file that have changed (much like a standard Unix diff would) -- however, when looking at changes in LaTeX source code, the granularity of the line-based difference is much too coarse.  An author would most likely only care about the change in words of line 15 or even characters such as removing the mistaken `d' in the word ``Roger'' in line seven.

\subsection{Collaborating} \label{sec:collaborating}

Collaboration on your writing project mainly happens through git so we show how to setup an example here.  Your actual setup for writing projects may differ.  Whatever the configuration, it will probably show up under the list of registered remotes for your repository.  In our example, we cloned from the downloaded .bundle file, so looking at the remotes will look like this:
\begin{CodeVerbatim}
$> git remote -v
origin	$TUTORIAL/independence.bundle (fetch)
origin	$TUTORIAL/independence.bundle (push)
\end{CodeVerbatim}

As an example of interacting with another repository, let us create a second one on our local file system.  In practice, the remote repository will most likely be on a different computer and accessed via certain network protocols reflected in the address.  Feel free to adjust the file locations in the example below to your taste.

\begin{CodeVerbatim}
$> cd $TUTORIAL/
$> git clone independence-sherman.bundle independence-sherman
Cloning into 'independence-sherman'...
Receiving objects: 100% (24/24), done.
Resolving deltas: 100% (6/6), done.
$> cd independence-sherman/
$> git log --oneline 
39cd617 todo item for indictment
45710ff more text for preamble
d3f904c sixth version
203e0ce fifth version
36eeab0 fourth version
fa2be39 third version
bac2f51 second version
d6d1cf8 first version
\end{CodeVerbatim}

Now we impersonate Roger Sherman in the newly created repository above, and also check the setting for its remotes.
\begin{CodeVerbatim}
$> git config --add user.name "Roger Sherman"
$> git config --add user.email "sherman@usa.gov"
$> git config --list | grep -e "[Ss]herman"
remote.origin.url=$TUTORIAL/independence-sherman.bundle
user.name=Roger Sherman
user.email=sherman@usa.gov
$> git remote -v
origin	$TUTORIAL/independence-sherman.bundle (fetch)
origin	$TUTORIAL/independence-sherman.bundle (push)
\end{CodeVerbatim}

Next, we make the first repository aware of the second and vice versa.  At the same time, we may want to remove the reference to the original bundle so as to not get confused with which repository to synchronize.  So in both repositories do
\begin{CodeVerbatim}
$> git remote remove origin  # this is optional!
\end{CodeVerbatim}

Then, we go into the first one and add a new remote location there:
\begin{CodeVerbatim}
$> cd $TUTORIAL/independence/
$> git remote add sherman $TUTORIAL/independence-sherman
$> git remote -v
sherman	$TUTORIAL/independence-sherman (fetch)
sherman	$TUTORIAL/independence-sherman (push)
\end{CodeVerbatim}

Afterwards, we go into the second one and add a new remote location there:
\begin{CodeVerbatim}
$> cd $TUTORIAL/independence-sherman/
$> git remote add adams $TUTORIAL/independence
$> git remote -v
adams	$TUTORIAL/independence (fetch)
adams	$TUTORIAL/independence (push)
\end{CodeVerbatim}

Now you can pull from each directory what the other person has done.  Notice that you cannot push changes to the other directory, as these git repositories are not ``bare.''  This means, they contain working copies and thus cannot be altered remotely.  However, in most situations you may be using a central repository (such as GitHub or a server) that indeed contains a bare repository.  Then, you are typically able to pull and push changes with such a remote repository while your coauthors can do the same to synchronize your work.

We will see examples below in Sections~\ref{sec:tutorial-git-emacs:collab} and \ref{sec:tutorial-git:collab} how John Adams and Roger Sherman synchronize changes with each other.

%At this point, the two repositories are synched so the operations do not perform any changes.  These two commands are also available through the LTC Editor user interface---after choosing the remote from the pull-down menu in the lower-right corner, click the ``Pull'' or ``Push'' buttons.  If the menu is empty you may have to update the editor to obtain the latest git changes that you may have performed on the command line or using a graphical git tool.
%\begin{CodeVerbatim}
%$> git push philadelphia master
%Everything up-to-date
%$> git pull philadelphia master
%From /Users/linda/git/independence
% * branch            master     -> FETCH_HEAD
%Already up-to-date.
%\end{CodeVerbatim}

% TODO: create example that shows branches in commit graph => independence2.bundle?

%\subsection{Resolving Merge Conflicts}


% !TEX root = manual.tex
\section{Tutorial with Git and Emacs} \label{sec:tutorial-git-emacs}

In this section, we assume that the example git repository has been created according to the instructions in Section~\ref{sec:example-git} above.  And we assume that LTC has been installed using the optional Emacs directory, as well as Emacs configuration adjustments made that are mentioned in Section~\ref{sec:config-emacs}.


% !TEX root = manual.tex
\section{Tutorial with Git and LTC Editor} \label{sec:tutorial-git}

In this section, we assume that the example git repository has been created according to the instructions in Section~\ref{sec:example-git} above.

\subsection{Running the LTC Editor}

First, we start the LTC Editor to interact with LTC and track the changes of the file.  Assuming you have installed LTC in the directory \Code{\$LTC}, we can look at the command line options of the editor:
\begin{CodeVerbatim}[commandchars=\\\{\}]
$> java -cp $LTC/LTC.jar com.sri.ltc.editor.LTCEditor -h
LaTeX Track Changes (LTC)  Copyright (C) 2009-2013  SRI International
This program comes with ABSOLUTELY NO WARRANTY; for details use command line switch -c.
This is free software, and you are welcome to redistribute it under certain conditions.

usage: java -cp ... com.sri.ltc.editor.LTCEditor [options...] [FILE] 
with
 FILE     : load given file to track changes
 -c       : display copyright/license information and exit
 -h       : display usage and exit
 -l LEVEL : set console log level
            SEVERE, WARNING, INFO, CONFIG (default), FINE, FINER, FINEST
 -r       : reset to default settings
\end{CodeVerbatim}

To open our tutorial file at \Code{\$TUTORIAL/independence/independence.tex} when starting the editor, execute the following command.  This will open the editor as a window similar to the screen shot in Figure~\ref{fig:editor-open}.
\begin{CodeVerbatim}[samepage=true,commandchars=\\\{\}]
$> java -cp $LTC/LTC.jar \textbackslash
   com.sri.ltc.editor.LTCEditor $TUTORIAL/independence/independence.tex
\end{CodeVerbatim}
\begin{figure}[t]
\centering
\mygraphics{width=\textwidth,height=.5\textheight,keepaspectratio}{figures/editor-open}
\caption{Initial opening of tutorial file in LTC Editor} \label{fig:editor-open}
\end{figure}
In this figure, we see a panel at the bottom-right that resembles the upper part of the GitX graphical interface to git.  There, we display the history of the current LaTeX file under git.  We can also see what git currently perceives as the current user -- now John Adams because we had overridden the git settings in this tutorial repository.

\subsection{Showing and Hiding Certain Changes}

The bottom-left panels of the editor allows us to customize the way LTC displays the changes of the file.  Section~\ref{sec:general-use} contains all the details of how LTC displays the changes including limiting the file history and filtering.  In this tutorial, we will just use some of the options and see their effect.

First, notice the colors assigned to each of the authors.  To change an author color, for example Roger Sherman's,  perform a double-click on the colored square next to Roger Sherman to open a dialog and  choose a dark color such as brown (you will want something with contrast to the white background).  Notice how the text in the editor panel on the top changes color for those parts that are attributed to Roger Sherman's edits.

Next, focus on the typographical errors in the command ``\textbackslash maketitle'' in line 11 and the beginning of the first paragraph in line thirteen as well as the spelling errors in the word ``political.''  If you first uncheck the box for ``small'' changes and second, also the box for deletions, notice how the text rendering in the editor panel changes.
\begin{figure}[t]
  \centering
  \subfloat{
    \label{subfig:editor-filter-small1} 
    \mygraphics{width=.25\textwidth}{figures/editor-filter-small1}}
  \hspace{2em}
  \subfloat{
    \label{subfig:editor-filter-small2} 
    \mygraphics{width=.25\textwidth}{figures/editor-filter-small2}}
  \hspace{2em}
  \subfloat{
    \label{subfig:editor-filter-small3} 
    \mygraphics{width=.25\textwidth}{figures/editor-filter-small3}}
\caption{Effect of hiding ``small'' changes first (middle) and then also deletions (right)} \label{fig:editor-filter-small}
\end{figure}
Figures~\ref{fig:editor-filter-small} show that ``\textbackslash maketitle'' as well as the typos in the word ``political'' are no longer marked up, and in the third image, the deletion beginning with ``If'' at the beginning of the paragraph is now omitted.

\subsection{Understanding the Commit Graph}

Now draw your attention back to the graph with the history of the current file under git (located in the bottom-right panel).  In our current tutorial repository, this graph is just a line as the authors committed their versions in sequential order.  %For the tutorial example, in which the writing project contained only one .tex file, the history graph in the LTC Editor looks almost the same as viewing the git repository in a graphical tool such as GitX.  However, if the writing project contained more .tex files, these graphs would not look similar anymore.  The git repository is tracking the history of all files that have been added to it.  In contrast, the LTC Editor only displays the history of the current .tex file loaded.
\begin{figure}[t]
\centering
\mygraphics{scale=.5}{figures/commit-graph}
\caption{Example of commit graph} \label{fig:commit-graph}
\end{figure}
Refer to Figure~\ref{fig:commit-graph} for a screen shot of the example file history. Versions that are included in the tracked changes are printed in black and denoted with a filled circle.  How far we go back in history depends on some filtering settings, which are discussed further in Section~\ref{sec:limit-history} below.  By default, we first include all version of the current author at the top.  In our example with impersonating John Adams, there are currently no further recent commits of him.  Then, we continue down the path and collect all versions of different authors until we find the next version of John Adams in the commit with the message ``second version.''

%After we have obtained an ordered sequence of commits by traversing the graph from the top down, we ignore subsequent commits of the same author---only the latest version of the same author is considered.  The idea behind this is that we do not want to penalize an author who commits often to the repository.  Instead, we treat all his or her changes from the time the file was changed by a different author as one big change event.  In our example, see that Roger Sherman committed version 5 and 6 to the repository, but version 5 is depicted in gray and not taken into account when displaying the changes.  

\subsection{Limiting History} \label{sec:limit-history}

We allow the user to filter and customize how the potentially rich history of a .tex file is selected, so as to provide a better view of the tracked changes.  The user can show and hide changes as seen above, limit the authors of interest, and specify a date or revision number to tell LTC how far back in time the history should be considered.

%\subsubsection{Limiting by Authors}

\begin{figure}
\centering
  % first sub-figure
  \begin{minipage}[t]{0.35\linewidth}
  \centering
  \mygraphics{scale=.5}{figures/editor-select-authors}
  \caption{Selecting authors for filtering} \label{fig:editor-select-authors}
  \end{minipage}%
\hspace{0.04\linewidth}%
  % second sub-figure
  \begin{minipage}[t]{0.61\linewidth}
  \centering
  \mygraphics{scale=.5}{figures/editor-limit-authors}
  \caption{Effect of limiting authors to Roger Sherman and Thomas Jefferson after clicking ``Update''} \label{fig:editor-limit-authors}
  \end{minipage}  
\end{figure}
To limit the history by \textbf{authors}, select both authors Roger Sherman and Thomas Jefferson through clicking while holding down the CTRL or CMD key in the list of authors in the middle lower panel.  Then, click the button ``Limit'' below the list, which will gray out the unselected authors.  For a limiting action to take effect, you need to click ``Update.''  This is different from showing and hiding various changes as well as changing author colors, which is applied instantly.

Notice how any version by the ignored authors Benjamin Franklin and John Adams is now gray as only commits from the selected authors are considered.  Again, the history is only taken until the next revision of the current author but since he is being ignored, we go all the way back to the first revision. Compare your editor window with the screen shot in Figure~\ref{fig:editor-limit-authors} and see how the commit graph has changed.

Then, clicking the ``Reset'' button followed by ``Update'' will remove and limits on the history by author, so the original view is restored.

%\subsubsection{Limiting by Date or Revision}

Next, we apply limits on \textbf{revision} and \textbf{date} to control how far back the history of the file is considered.  As we had seen, the first version is not taken into account because it was committed before the next commit by the current author John Adams.  Let us now type the first few characters \Code{d6d} of the SHA-1 key of the first commit into the field labeled ``Start at revision:'' (refer to Figure~\ref{fig:editor-select-rev}) or simply drag the key from the entry in the commit graph, which will copy the complete SHA-1 sequence.  Now click ``Update'' and see how the first version is listed in black and considered in the tracked changes above as seen in Figure~\ref{fig:editor-limit-rev}.  Since changes by the current author John Adams from the first to the second version are now included, notice the text marked up in red appearing in the editor window. We see that John Adams must have added himself as an author in the LaTeX preamble among other edits.

\begin{figure}
\centering
  % first sub-figure
  \begin{minipage}[t]{0.35\linewidth}
  \centering
  \mygraphics{scale=.5}{figures/editor-select-rev}
  \caption{Selecting revision for filtering} \label{fig:editor-select-rev}
  \end{minipage}%
\hspace{0.04\linewidth}%
  % second sub-figure
  \begin{minipage}[t]{0.61\linewidth}
  \centering
  \mygraphics{scale=.5}{figures/editor-limit-rev}
  \caption{Effect of going back to first revision after clicking ``Update''} \label{fig:editor-limit-rev}
  \end{minipage}  
\end{figure}

To remove the limit by revision number, simply erase the text in the field ``Start at revision:'' and click ``Update'' again.

\begin{figure}
\centering
  % first sub-figure
  \begin{minipage}[t]{0.35\linewidth}
  \centering
  \mygraphics{scale=.5}{figures/editor-select-date}
  \caption{Selecting date for filtering} \label{fig:editor-select-date}
  \end{minipage}%
\hspace{0.04\linewidth}%
  % second sub-figure
  \begin{minipage}[t]{0.61\linewidth}
  \centering
  \mygraphics{scale=.5}{figures/editor-limit-date}
  \caption{Effect of limiting history to date of third version after clicking ``Update''} \label{fig:editor-limit-date}
  \end{minipage}  
\end{figure}

Limiting the history by date works similarly.  You may drag a date from the commit graph on the right, for example the date of the third version commit, and drop it into the field ``Start at date:'' on the left.  Or, type a date such as \Code{Jul 23, 2010 1:11p} into the field.  We employ some software to process times and dates in natural language, and if successful, the field will contain the date string as it was understood translated into the format used in the commit tree.
Again, you will need to click ``Update'' for the change to take effect or click RETURN while editing the text field.  See Figures~\ref{fig:editor-select-date} and \ref{fig:editor-limit-date} for screenshots. To remove the limit by date, erase all text in the text field and update.

\subsection{Condensing History}

\begin{figure}
\centering
  \begin{minipage}[b]{0.57\linewidth}
    \centering
    \mygraphics{scale=.5}{figures/editor-condense-on}
    \caption{Effect of condensing authors:\\ ignoring the fifth version by Roger Sherman} \label{fig:editor-condense-on}  
  \end{minipage}  
\hspace{0.04\linewidth}%
  \begin{minipage}[b]{0.38\linewidth}
    \subfloat{
      \label{subfig:editor-condense-before} 
      \mygraphics{width=0.8\linewidth}{figures/editor-condense-before}}
  \hspace{1em}
    \subfloat{
      \label{subfig:editor-condense-after} 
      \mygraphics{width=0.8\linewidth}{figures/editor-condense-after}}
    \caption{Example of markup change before (left) and after (right) condensing authors} \label{fig:editor-condense-before-after}
  \end{minipage}%
\end{figure}
Sometimes the list of commits considered is getting long and the resulting markup of the changes confusing.  One additional way to customize how the changes are displayed is a setting to ``condense authors.''  Find a check box with that name under the list of authors for filtering.  If checked, then only the latest version of an author of \textit{consecutive} commits is considered -- in our example, only the sixth version is shown in black while the fifth version by Roger Sherman is now grayed out as seen in Figure~\ref{fig:editor-condense-on}.

See Figure~\ref{fig:editor-condense-before-after} for an example of how condensing authors affects the markup.  Since we are only considering the changes that Roger Sherman made in the sixth version, his correction of the name is no longer shown.  Condensing authors makes sense when users commit versions often so that they do not loose too much history.  Their last version after a number of commits generally has the flavor of a ``final'' version, ready to be shared with others.  Hence, the changes made there compared to the last version of another author is commonly of most interest.

\subsection{Editing, Saving, and Committing}

Let us start the next step by resetting all filters to the default configuration, i.e., no limit by authors, date, and revision.  Then, we will edit the text in the editor panel to see the latest changes.

Click into the text field and enter some text, for example a LaTeX comment reminding John Adams to work on a list of charges against King George III in line nineteen:
\begin{FileVerbatim}
% list charges against King George III here
\end{FileVerbatim}
The added text will be rendered in red (or the color code for the current author) and underlined.  Notice how the commit graph adds a first line with the label ``modified'' and the ``Save'' button becomes enabled.  Now delete some of the characters you have just entered, for example the word \Code{here} at the end.  The characters simply disappear as they were added by the same author.

Now delete other characters that are either rendered black or a different color than red but not marked as deletions (strike-through font).  Notice how these characters remain visible but are now colored red and marked up with strike-through.  If you tried to delete anything that is already marked as deletion (i.e., anything in strike-through font), nothing will happen as this text is already deleted in a prior version.  See Figure~\ref{fig:editor-modified} for a screen shot of the above edits: text in red and underlined was added and text in red and strike-through was deleted.

\begin{figure}[t]
\centering
\mygraphics{width=\textwidth,height=.5\textheight,keepaspectratio}{figures/editor-modified}
\caption{After editing the text as John Adams} \label{fig:editor-modified}
\end{figure}

Finally, you will want to click ``Save'' to save the current file to disk.  This will cause the label ``modified'' to change to ``on disk.''  If you would then again edit, the label would switch back to ``modified'' of course.

Saving the file, however, does not tell git to create a new version under its management.  In order to commit the current file to git, first, make sure that the .tex file is saved and  the first line in the commit graph is set to ``on disk.''  Then, on a command line, switch to the directory with the tutorial file and perform the following commit command (printed in bold below).  You may want to check the status of git before and after the commit:
\begin{CodeVerbatim}[commandchars=\\\{\}]
$> git status
# On branch master
# Changes not staged for commit:
#   (use "git add <file>..." to update what will be committed)
#   (use "git checkout -- <file>..." to discard changes in working directory)
#
#	modified:   independence.tex
#
no changes added to commit (use "git add" and/or "git commit -a")
$> \textbf{git commit -am "added comment about list of charges"}
[master 930d080] added comment about list of charges
 1 file changed, 3 insertions(+), 1 deletion(-)
\end{CodeVerbatim}

To make LTC Editor aware of the underlying commit from the command line, click the ``Update'' button.  Notice how the recent commit gets included at the top of the list as seen in Figure~\ref{fig:commit-cmd-line}.
\begin{figure}[t]
\centering
\mygraphics{scale=.5}{figures/commit-cmd-line}
\caption{Updated commit graph after command line commit} \label{fig:commit-cmd-line}
\end{figure}

Now two things can happen depending on your setting for showing changes in comments (lower-left most panel).  If the checkbox was off, the newly added comment is no longer marked up in red with underlining.  After updating the editor from the commit history, the settings of which changes to show influence the markup of the text.  Now the newly entered comment is recognized as such, and if we hide changes in comments, the markup will not show.  If the box for ``changes in comments'' is checked, you will see your latest text still marked up as an addition.  Your editor should now look similar to the part shown in either Figure~\ref{fig:commit-no-comment} or Figure~\ref{fig:commit-comment}.
\begin{figure}
\centering
  % first sub-figure
  \begin{minipage}[t]{0.48\linewidth}
  \centering
  \mygraphics{width=\linewidth,height=.5\textheight,keepaspectratio}{figures/editor-commit-no-comment}
  \caption{Committing a comment but not showing it as an addition} \label{fig:commit-no-comment}
  \end{minipage}%
\hspace{0.04\linewidth}%
  % second sub-figure
  \begin{minipage}[t]{0.48\linewidth}
  \centering
  \mygraphics{width=\textwidth,height=.5\textheight,keepaspectratio}{figures/editor-commit-comment}
  \caption{Committing a comment and showing it as an addition} \label{fig:commit-comment}
  \end{minipage}  
\end{figure}

\subsection{Collaborating with Roger Sherman} \label{sec:tutorial-git:collab}

Next we perform an example collaboration with Roger Sherman's example repository as setup in Section~\ref{sec:collaborating} above. Remember that Roger Sherman's repository has had two more commits than the original one we used for John Adams.  Since we have made edits and commits as John Adams, both repositories have diverged.  To synchronize them, we first pull Roger Sherman's changes into our working copy after checking that we are in a good state:

\begin{CodeVerbatim}
$> git pull sherman master
remote: Counting objects: 8, done.
remote: Compressing objects: 100% (3/3), done.
remote: Total 6 (delta 2), reused 5 (delta 1)
Unpacking objects: 100% (6/6), done.
From $TUTORIAL/independence-sherman
 * branch            master     -> FETCH_HEAD
Auto-merging independence.tex
CONFLICT (content): Merge conflict in independence.tex
Automatic merge failed; fix conflicts and then commit the result.
\end{CodeVerbatim}

\begin{figure}
\centering
\mygraphics{scale=.35}{figures/editor-merge-conflict}
\caption{Git conflict markers in merged file} \label{fig:editor-merge-conflict}
\end{figure}
Unfortunately, the two repositories have diverged too much and a so-called ``merge conflict'' has arisen.  Now we have to tell git how to fix this before we can proceed.  So we look at the markers that git has put into our file.  You can click the ``Update'' button in LTC Editor to see these markers there similar to Figure~\ref{fig:editor-merge-conflict}.  On the command line, the file looks similar to this:
\begin{CodeVerbatim}
$ cat independence.tex
[...]

<<<<<<< HEAD
% list charges against King George III
=======
That to secure these rights, Governments are instituted among Men, [...] 

%TODO: indictment here
>>>>>>> 39cd6172613d1065a4cddc854cf30067869fc727
\end{CodeVerbatim}

We decide that the comments in the \Code{HEAD} version means the same as the last comment in the merged version \Code{39cd617...} so we modify the text so that it looks like this:
\begin{CodeVerbatim}
$> cat independence.tex 
[...]

That to secure these rights, Governments are instituted among Men, [...]

% list charges against King George III
\end{CodeVerbatim}
It is important to remove the git marker lines starting with \Code{<<<<<<<}, \Code{=======}, and \Code{>>>>>>>} for git to recognize that we have resolved the conflicts.  Now committing on the command line yields:
\begin{CodeVerbatim}
$> git commit -am "merging Roger Sherman's edits"
[master 2a5b3c3] merging Roger Sherman's edits
\end{CodeVerbatim}

\begin{figure}
\centering
\mygraphics{scale=.35}{figures/editor-merge-resolve}
\caption{After resolving conflict, committing, and updating in LTC Editor} \label{fig:editor-merge-resolve}
\end{figure}
This has resolved the conflict and incorporated Roger Sherman's prior changes, as a look at the git log with the graphing function reveals:
\begin{CodeVerbatim}
$> git log --oneline --graph --date-order
*   2a5b3c3 merging Roger Sherman's edits
|\  
* | 930d080 added comment about list of charges
| * 39cd617 todo item for indictment
| * 45710ff more text for preamble
|/  
* d3f904c sixth version
* 203e0ce fifth version
* 36eeab0 fourth version
* fa2be39 third version
* bac2f51 second version
* d6d1cf8 first version
\end{CodeVerbatim}
Once we update LTC Editor, we see the paragraph that was part of the conflicting region now correctly attributed to Roger Sherman.  Furthermore, the git commit graph has gotten more interesting with the branching and merging in the first column of the commit graph.  Refer to Figure~\ref{fig:editor-merge-resolve} for a screen shot of the git commit graph and text changes after resolving the conflict, committing and updating LTC Editor.


% !TEX root = manual.tex
\section{Creating the Example Subversion Repository} \label{sec:example-svn}

This tutorial uses an example svn repository, which is either hosted on the Internet or on your local computer.  The first Section~\ref{sec:example-svn-remote} shows how to use a publicly accessible repository with an example file.  This is the quickest way to try out LTC with Subversion but you cannot commit new versions to this repository so we cannot go through such advanced topics in the later parts of the tutorial.  The second Section~\ref{sec:example-svn-local} shows how to create a local svn server and populates it with an example repository.  This takes a bit more time to setup but then you can go through more advanced topics such as committing to the repository.

\subsection{Using the Remote Subversion Repository} \label{sec:example-svn-remote}

To create the example file that is under remote svn version control, go into a directory of your choice (say \Code{\$TUTORIAL}) and do the following.  If the server causes a certificate alert, you can accept it permanently by using \Code{p} as shown in bold below.

\begin{CodeVerbatim}[commandchars=\\\{\}]
$> cd $TUTORIAL
$> svn co https://spartan.csl.sri.com/svn/public/LTC/tutorial-svn independence 
Error validating server certificate for 'https://spartan.csl.sri.com:443':
 - The certificate is not issued by a trusted authority. Use the
   fingerprint to validate the certificate manually!
Certificate information:
 - Hostname: spartan.csl.sri.com
 - Valid: from Fri, 03 May 2013 00:00:00 GMT until Sat, 03 May 2014 23:59:59 GMT
 - Issuer: Thawte, Inc., US
 - Fingerprint: f9:3f:b3:27:65:89:c9:af:bf:05:1b:5f:60:f0:8c:df:4d:bc:47:7e
(R)eject, accept (t)emporarily or accept (p)ermanently? \textbf{p}
A    independence/independence.tex
Checked out revision 6.
\end{CodeVerbatim}

Now change into the new directory and confirm that the file has six revisions in its history
\begin{CodeVerbatim}
$> cd independence/
$> svn log -q independence.tex 
------------------------------------------------------------------------
r6 | sherman | 2012-11-13 13:01:00 -0600 (Tue, 13 Nov 2012)
------------------------------------------------------------------------
r5 | sherman | 2012-11-13 13:00:35 -0600 (Tue, 13 Nov 2012)
------------------------------------------------------------------------
r4 | jefferson | 2012-11-13 12:59:45 -0600 (Tue, 13 Nov 2012)
------------------------------------------------------------------------
r3 | franklin | 2012-11-13 12:59:03 -0600 (Tue, 13 Nov 2012)
------------------------------------------------------------------------
r2 | adams | 2012-11-13 12:58:04 -0600 (Tue, 13 Nov 2012)
------------------------------------------------------------------------
r1 | jefferson | 2012-11-13 12:51:35 -0600 (Tue, 13 Nov 2012)
------------------------------------------------------------------------
\end{CodeVerbatim}

Unfortunately, we cannot accept changes to this repository so the tutorials based on svn do not cover how to commit new revisions and how to collaborate.  We advise to install a local svn server and repository per the instructions below or to go through the git-based tutorial to cover those points.

\subsection{Using a Local Subversion Repository} \label{sec:example-svn-local}

%TODO
% !TEX root = manual.tex
\section{Tutorial with Subversion and Emacs} \label{sec:tutorial-svn-emacs}

In this section, we assume that the example svn repository has been created according to the instructions in Section~\ref{sec:example-svn} above.  And we assume that LTC has been installed using the Emacs directory, as well as Emacs configuration adjustments made that are mentioned in Section~\ref{sec:config-emacs}.

\subsection{Starting LTC Server and \texttt{ltc-mode}}

First, we start the LTC Server from the command line.  Assuming you have installed LTC in the directory \Code{\$LTC}, we run this command line for the server.  The output will be similar to the following.  Leave the server running while performing the rest of this tutorial.
\begin{CodeVerbatim}
$> java -jar $LTC/LTC.jar
LaTeX Track Changes (LTC)  Copyright (C) 2009-2013  SRI International
This program comes with ABSOLUTELY NO WARRANTY; for details use command line switch -c.
This is free software, and you are welcome to redistribute it under certain conditions.

<current date> | CONFIG:  Logging configured to level CONFIG
<current date> | CONFIG:  LTC version: <version info>
<current date> | INFO:    Started RPC server on port 7777.
\end{CodeVerbatim}

Next, we switch to Emacs and open the tutorial file \Code{\$TUTORIAL/independence/independence.tex}.  This should put Emacs into \Code{latex-mode} but any other mode should work as well.  Then, start LTC mode using the command \Code{M-x ltc-mode} in Emacs.  Beware that using LTC with a remote subversion server takes longer than using git or a local subversion server, as we have to query the distant server hosting the repository for each version of the .tex file.  You will see a few messages appearing briefly in the mini buffer (you can also look at them in the \Code{*Messages*} buffer), such as the following.  While the potential time intensive task of downloading versions from the remote server happen, the mini buffer should say \Code{Starting LTC update...}, which turns into \Code{LTC updates received} when the process is done.
\begin{FileVerbatim}
Starting LTC mode for file "$TUTORIAL/independence/independence.tex"...
Using `xml-rpc' package version: 1.6.8.2
LTC session ID = 1
Starting LTC update...
LTC updates received
\end{FileVerbatim}
Emacs should now look similar to the screen shot in Figure~\ref{fig:svn-emacs-open}.
\begin{figure}[t]
\centering
\mygraphics{scale=.35}{figures/svn-emacs-open}
\caption{Starting \Code{ltc-mode} in Emacs with tutorial file under svn} \label{fig:svn-emacs-open}
\end{figure}
In this figure, we see the changes marked up in various colors and fonts: underlined for additions and inverse for deletions.  There is also a smaller buffer called ``LTC info (session $N$)'' with $N$ as the session ID, at the bottom (or the right, if your Emacs is in landscape mode) of the buffer with the tutorial file.  There, we display the history of the current file under svn.  We can also see what svn perceives as the current user at the top of the graph -- here the name of the local user.

First, we will override what LTC thinks is the current author in order to make the following tutorial more meaningful.  In real life situations you will rarely have to use this command as you typically want the changes in the repository attributed to yourself.  In Emacs, type the command \Code{M-x ltc-set-self<RET>adams<RET>} to impersonate John Adams.  This updates the contents in the main buffer and info buffer at the bottom automatically, which may again take a little time with a remote subversion server.  The info buffer will then look like the screen shot in Figure~\ref{fig:svn-adams}.
\begin{figure}[t]
\centering
\mygraphics{scale=.5}{figures/svn-adams}
\caption{Emacs info buffer after setting current author to ``adams''} \label{fig:svn-adams}
\end{figure}

\subsection{Showing and Hiding Certain Changes}

The LTC menu and ``LTC info'' buffer in Emacs allow us to customize the way LTC displays the changes of the file.  Section~\ref{sec:general-use} contains all the details of how LTC displays the changes including limiting the file history and filtering.  In this tutorial, we will just use some of the options and see their effect.

First, notice the colors assigned to each of the authors.  To change an author color, for example Roger Sherman's,  perform a single left-click on the name of Roger Sherman.  This opens another buffer called \Code{*Colors*} with a preview of colors and their names.  Also look at the mini buffer that requests input.   You can enter a name or an RGB value in hex notation.  The color names can also be auto-completed, for example type \Code{Bro<TAB>} (if TAB is your completion key in Emacs) to see \Code{Brown}.  You will want something with contrast to the white background, so brown is a fine choice.  When clicking the RETURN key, notice how the text in the editor panel on the top changes color for those parts that are attributed to Roger Sherman's edits.  To abort choosing a color simply enter an empty value.

Next, focus on the typographical errors in the command ``\textbackslash maketitle'' in line 11 and the beginning of the first paragraph in line thirteen as well as the spelling errors in the word ``political.''  Open the LTC menu (in the menu bar and in the mode line) and then the sub-menu ``Show/Hide'' as seen in Figure~\ref{fig:svn-emacs-LTC-menu}.  
\begin{figure}[t]
\centering
\mygraphics{scale=.35}{figures/svn-emacs-LTC-menu}
\caption{Opening the LTC menu from the mode line in Emacs} \label{fig:svn-emacs-LTC-menu}
\end{figure}
If you first uncheck the item \Menu{Show/Hide;Show small changes}{LTC}, and second, also the item \Menu{Show/Hide;Show deletions}{LTC}, notice how the text rendering in the editor panel changes.  Again, if you work with the remote repository, it might take a little while until all the updates are received from the server and the mini buffer shows the message ``LTC updates received.''
\begin{figure}[t]
  \centering
  \subfloat{
    \label{subfig:svn-emacs-filter-small1} 
    \mygraphics{scale=.5}{figures/svn-emacs-filter-small1}}
  \hspace{2em}
  \subfloat{
    \label{subfig:svn-emacs-filter-small2} 
    \mygraphics{scale=.5}{figures/svn-emacs-filter-small2}}
  \hspace{2em}
  \subfloat{
    \label{subfig:svn-emacs-filter-small3} 
    \mygraphics{scale=.5}{figures/svn-emacs-filter-small3}}
\caption[Effect of hiding ``small'' changes and deletions]{Effect of hiding ``small'' changes first (middle) and then also deletions (right)} \label{fig:svn-emacs-filter-small}
\end{figure}
Figures~\ref{fig:svn-emacs-filter-small} show that ``\textbackslash maketitle'' as well as the typos in the word ``political'' are no longer marked up, and in the third image, the deletion beginning with ``If'' at the beginning of the paragraph is now omitted.

\subsection{Understanding the Commit Graph}

Now draw your attention back to LTC info buffer with the history of the current file under git (located at the bottom or right of your .tex file).  The Emacs representation is using small box characters to draw the graph and its edges.  In our current tutorial repository, there are no branches and the graph is a sequential line.  

Refer back to Figure~\ref{fig:svn-adams} for the screen shot of the example file history. Versions that are included in the tracked changes are not printed in gray.  How far we go back in history depends on some filtering settings, which are discussed further in Section~\ref{sec:svn-emacs-limit-history} below.  By default, we first include all version of the current author at the top.  In our example with impersonating John Adams with the user name ``adams,'' there are currently no further recent commits of him.  Then, we continue down the path and collect all versions of different authors until we find the next version of John Adams in the commit with the message ``second version.''

\subsection{Limiting History} \label{sec:svn-emacs-limit-history}

We allow the user to filter and customize how the potentially rich history of a .tex file is selected, so as to provide a better view of the tracked changes.  The user can show and hide changes as seen above, limit the authors of interest, and specify a date or revision number to tell LTC how far back in time the history should be considered.

\begin{figure}
\centering
\mygraphics{scale=.5}{figures/svn-emacs-limit-authors}
\caption[Effect on commit graph of limiting authors]{Effect on commit graph of limiting authors to ``sherman'' and ``jefferson''} \label{fig:svn-emacs-limit-authors}
\end{figure}
To limit the history by \textbf{authors}, choose menu item \Menu{Limit by;Set of authors...}{LTC}. This will prompt the user to enter author names to limit by in the mini buffer.  Again, automatic completion works, so you can enter \Code{sh<TAB> <RET>} and \Code{je<TAB> <RET> <RET>} to select authors ``sherman'' and ``jefferson'' and exit the dialog.  After the last author was selected, the system automatically updates the displayed changes.

Notice how any version by the ignored authors ``franklin'' and ``adams'' is now gray as only commits from the selected authors are considered.  The first line in the LTC info buffer still shows the currently active author ``adams,'' so this line is not gray.  Again, the history is only taken until the next revision of the current author but since he is being ignored, we go all the way back to the first revision. Compare your Emacs now with the screen shot in Figure~\ref{fig:svn-emacs-limit-authors} and see how the file history has changed.

To reset limiting by authors, simply choose the same menu \Menu{Limit by;Set of authors...}{LTC} again and enter an empty author as the first one.  Now the display is back in the original state.

Next, we apply limits on \textbf{revision} and \textbf{date} to control how far back the history of the file is considered.  As we had seen, the first version is not taken into account because it was committed before the next commit by the current author John Adams.  Let us now choose menu item \Menu{Limit by;Start at revision...}{LTC}.  This will prompt the user to specify a known revision number.  Type the first few characters \Code{d6d<RET>} of the SHA-1 key of the first commit.  If unique, it is not necessary to expand the revision number using the TAB key (or whatever key is used for completion in your Emacs configuration).  See how the first version is listed in color and considered in the tracked changes above as seen in Figure~\ref{fig:svn-emacs-limit-rev}.  Since changes by the current author John Adams from the first to the second version are now included, notice the text marked up in red appearing in the editor window. We see that John Adams must have added himself as an author in the LaTeX preamble among other edits in the second commit of the file. 

\begin{figure}
\centering
%\mygraphics{scale=.5}{figures/svn-emacs-limit-rev}
\caption{Effect on commit graph of going back to first revision} \label{fig:svn-emacs-limit-rev}
\end{figure}

Another way of limiting by revision number is to simply left-click the number in the display.  

To remove the limit by revision number, simply choose the same menu \Menu{Limit by;Start at revision...}{LTC} again and enter an empty revision. Or, click into the empty revision column of the first line (denoting the currently active author) to achieve the same effect.  Now the display is back in the original state.

\begin{figure}
\centering
\mygraphics{scale=.5}{figures/svn-emacs-limit-date}
\caption{Effect on commit graph of limiting history to date of third version} \label{fig:svn-emacs-limit-date}
\end{figure}

Limiting the history by date works similarly.  Select menu item \Menu{Limit by;Start at date...}{LTC}. At the prompt, you can enter a date from the history of the file using auto-completion.  For example, enter \Code{2<TAB>2:59:0<TAB> <RET>} to get the exact date of the third revision.  Or, type a date such as \Code{Nov 13, 2012 12:59p} (should yield the same results if you are in the Central Time Zone) into the field.  We employ some software to process times and dates in natural language, and if successful, the field will contain the date string as it was understood translated into the format used in the commit tree. You may also perform a left-click on the date in the history to achieve the same effect.  See Figure~\ref{fig:svn-emacs-limit-date} for a screen shot of the effect of limiting to the date of the third revision. 

To remove the limit by date, either left-click on the empty date column of the first line of the file history or enter an empty date after selecting menu item \Menu{Limit by;Start at date...}{LTC} again.

\subsection{Condensing History}

\begin{figure}
\centering
\mygraphics{scale=.5}{figures/svn-emacs-condense-on}
\caption[Effect of condensing authors]{Effect of condensing authors: ignoring the fifth version by Roger Sherman} \label{fig:svn-emacs-condense-on}  
\end{figure}
Sometimes the list of commits considered is getting long and the resulting markup of the changes confusing.  One additional way to customize how the changes are displayed is a setting to ``condense authors.''  Now check the menu \Menu{Condense authors}{LTC}.  Then, only the latest version of an author of \textit{consecutive} commits is considered -- in our example, only the sixth version is colored while the fifth version by Roger Sherman is now grayed out as seen in Figure~\ref{fig:svn-emacs-condense-on}.

\begin{figure}
\centering
\subfloat{
  \label{subfig:svn-emacs-condense-before} 
  \mygraphics{scale=.5}{figures/svn-emacs-condense-before}}
\hspace{1em}
\subfloat{
  \label{subfig:svn-emacs-condense-after} 
  \mygraphics{scale=.5}{figures/svn-emacs-condense-after}}
\caption[Example of markup change when condensing authors]{Example of markup change before (left) and after (right) condensing authors} \label{fig:svn-emacs-condense-before-after}
\end{figure}
See Figure~\ref{fig:svn-emacs-condense-before-after} for an example of how condensing authors affects the markup.  Since we are only considering the changes that Roger Sherman made in the sixth version, his correction of the name is no longer shown.  Condensing authors makes sense when users commit versions often so that they do not loose too much history.  Their last version after a number of commits generally has the flavor of a ``final'' version, ready to be shared with others.  Hence, the changes made there compared to the last version of another author is commonly of most interest.

\subsection{Editing, Saving, and Committing}

% !TEX root = manual.tex
\chapter{Tutorial with Subversion} \label{ch:tutorial-svn}

\section{Creating the Example File}

This tutorial uses an example svn repository, which is hosted on the Internet.  To create the example file that is under svn version control, go into a directory of your choice (say \Code{\$TUTORIAL}) and do the following.

\begin{CodeVerbatim}
$> cd $TUTORIAL
$> svn co svn://svn.code.sf.net/p/latextrack/tutorial-svn/trunk independence
A    independence/independence.tex
Checked out revision 7.
$> cd independence/
$> svn status
$> svn log -q independence.tex 
------------------------------------------------------------------------
r7 | lilalinda | 2012-11-08 11:38:42 -0600 (Thu, 08 Nov 2012)
------------------------------------------------------------------------
r6 | lilalinda | 2012-11-08 11:38:27 -0600 (Thu, 08 Nov 2012)
------------------------------------------------------------------------
r5 | lilalinda | 2012-11-08 11:38:10 -0600 (Thu, 08 Nov 2012)
------------------------------------------------------------------------
r4 | lilalinda | 2012-11-08 11:37:52 -0600 (Thu, 08 Nov 2012)
------------------------------------------------------------------------
r3 | lilalinda | 2012-11-08 11:37:08 -0600 (Thu, 08 Nov 2012)
------------------------------------------------------------------------
r2 | lilalinda | 2012-11-08 11:35:53 -0600 (Thu, 08 Nov 2012)
------------------------------------------------------------------------
\end{CodeVerbatim}

[TODO: impersonate as different people for svn creation and tutorial!]

\section{Running the LTC Editor}

[TODO]



% !TEX root = manual.tex
\chapter{Using LTC}

The intended use of LTC is running the base system as a server and have a plugin of the user's preferred LaTeX editor connect to it.  The provided Emacs \Code{ltc-mode} is an example of such a plugin and its use is described in Section~\ref{sec:emacs}.  We hope to add more plugins for other popular LaTeX editors in the future.

As a minimum implementation of a user interface to LTC, we also provide a Java application ``LTC Editor'' that uses the base system API but does not rely on running a separate LTC server.  Its use is explained in Section~\ref{sec:java}.

Before we address using these specific user interfaces, we discuss how to setup your repository for a LaTeX writing project with git (Section~\ref{sec:git-use}) or svn (Section~\ref{sec:svn-use}) and then the general usage of LTC in Section~\ref{sec:general-use}.

% !TEX root = manual.tex
\section{Using a Git Repository} \label{sec:git-use}

For each writing project, LTC expects the history of the .tex files managed by a version control system, for example contained in a git repository.  As git is a distributed version control system, this repository is local to your machine.  If you need to exchange data with a collaborating author, you will push your repository or pull their repository and merge it with your local copy.

This section only covers a small subset of what git can do with respect to setting up a repository for your writing project.  Please refer to other git documentation about general git usage and how to further manage your writing project with git.  Also note our suggestions for general, one-time git configuration in Section~\ref{sec:git-install}.

\subsection{Initializing a Local Repository}

Assuming that your current LaTeX source files (and other files) are located in a directory structure called \Code{\$PROJECT}.  To initialize the top-level directory for git perform the following commands.
\begin{CodeVerbatim}
$> cd $PROJECT/
$> git init 
Initialized empty Git repository in $PROJECT
\end{CodeVerbatim}

Decide, what the final build products in your project will be.  These should be ignored by git so as not to complain every time you recompile your LaTeX project.  Let's assume your project will create a file called ``proposal.pdf,'' then create a file called \Code{.gitignore} in this directory that contains in each line the name of every build product.  In a bash shell, you can do the following.
\begin{CodeVerbatim}
$> cat > .gitignore <<EOF
> proposal.pdf
> EOF
\end{CodeVerbatim}

Then check the contents of the file:
\begin{CodeVerbatim}
$> less .gitignore
proposal.pdf
\end{CodeVerbatim}

If you decide on more build products (e.g., files called ``proposal-vol1.pdf'' and ``proposal-vol2.pdf'') in the future, simply edit the \Code{.gitignore} file to include these file names in new lines.  Make sure to do this before using a command such as \Code{git add .}, which would mark any existing build products for addition.

Checking the current status of git, the output should be similar to the following.

\begin{CodeVerbatim}
$> git status
# On branch master
#
# Initial commit
#
# Untracked files:
#   (use "git add <file>..." to include in what will be committed)
#
#	.gitignore
#	proposal.tex
nothing added to commit but untracked files present (use "git add" to track)
\end{CodeVerbatim}

Then, add the files already in your directory as well as the newly created file \Code{.gitignore} and commit the first version, for example this way:
\begin{CodeVerbatim}
$> git add .
$> git commit -a -m "initial commit of project"
[master (root-commit) dfbf239] initial commit of project
 2 files changed, 7 insertions(+), 0 deletions(-)
 create mode 100644 .gitignore
 create mode 100644 proposal.tex
\end{CodeVerbatim}

The option \Code{-m} stands for a brief message that you would like to attach to your commit. Note that you have to give some sort of message for every commit you make. Do \textit{not} try to skip the message part. Moreover, having meaningful one-line message for commit is always useful as other and you yourself can refer to later on to see what changes you made and why.

\subsection{Uploading Your Initial Repository}\label{sec:upload-git}

To share your local git repository, you can clone it to a shared file system or to a file server that each collaborator can access via \Code{ssh}.  Another option is that your system administrators provide you with a central git repository server.  Contact them for details on how to upload your git repository there.

In the following commands, we are giving the remote repository the name \Code{project.git} but you can choose whatever you want.  The ending \Code{.git} is somewhat standard, though, so we advise to keep it.  

To clone to a \textit{shared file system}, you will want to find a suitable directory \Code{\$SHARED\_PATH} where you want to create the shared repository. The following command is issued from inside your initial repository.

\begin{CodeVerbatim}
$> cd $PROJECT/
$> git clone --bare . $SHARED_PATH/project.git
\end{CodeVerbatim}

To clone to a \textit{file server accessible via ssh and scp}, you would do the following from the top-level directory of your initial repository. The remote repository should be located under \Code{\$REMOTE\_PATH} on the server.  We assume that you have performed at least one commit since initializing the git repository.
\begin{CodeVerbatim}
$> cd $PROJECT/..
$> git clone --bare $PROJECT project.git
Cloning into bare repository 'project.git'...
done.
$ touch project.git/git-daemon-export-ok
$ scp -rq project.git username@server:$REMOTE_PATH
\end{CodeVerbatim}

In either case, make sure that file permissions allow collaborators to access and read the files on the server.

Next, you will want to add a short name for the newly designated shared location called ``origin'' so that the push and pull command below work as if you had cloned the repository from someone else.  If you used a git server, your system administrators can tell you how to configure your original repository to include the new remote location, or you can clone the remote repository anew as shown in the next section.
\begin{CodeVerbatim}
$> git remote add origin $SHARED_PATH/project.git
\end{CodeVerbatim}
or
\begin{CodeVerbatim}
$> git remote add origin username@server:$REMOTE_PATH/project.git
\end{CodeVerbatim}

Finally, your system administrator may already have a server for git repositories set up that you and your collaborators can use.  Refer to their instructions on how to upload or create an initial repository.

\subsection{Cloning from a Remote Repository}

To start a git repository from an existing one (either on a shared file server or a central repository), you want to clone it.  You need to know the remote location in terms of user name, server address and path to the git repository.  Your system administrator can tell you these in the form of \Code{username@server:path-to-git-repos/project.git}, or, if you used a shared file system as in Section~\ref{sec:upload-git}, you simply use \Code{\$SHARED\_PATH/project.git} instead of the address above. 

Assuming that your local copy should be located in a directory \Code{my\_project}, you would need to execute the following command from the parent directory of \Code{my\_project}.  Feel free to call your new working directory something else by substituting the last argument.
\begin{CodeVerbatim}
$> git clone username@server:path-to-git-repos/project.git my_project
Cloning into my_project...
done.
$> cd my_project
$> git remote -v
\end{CodeVerbatim}

\subsection{Push and Pull}

To exchange data with the central repository, you typically push and pull.  In the simplest case, the following should work (if this is the original and not a cloned repository, you must have added the new short name ``origin'' via the \Code{git remote add origin} command above).
\begin{CodeVerbatim}
$> git push origin master
$> git pull origin master
\end{CodeVerbatim}

If git complains about uncommitted changes and that the working copy is not clean, you may have to commit or stash changes before these commands can run successfully.

Please refer to the many online resources to learn more about git, or ask you system administrator.

% !TEX root = manual.tex
\section{Using a Subversion Repository} \label{sec:svn-use}

For each writing project, LTC expects the history of the .tex files managed by a version control system, for example contained in a svn repository.  As svn is a centralized version control system, the repository is typically in a remote location.  To use LTC meaningfully, it has to download the different versions of the file history so you will need constant connectivity with the server.  If you need to exchange data with a collaborating author, you will update from and commit to your remote repository, which also requires online access.

This section only covers a small subset of what svn can do with respect to setting up a repository for your writing project.  Please refer to other svn documentation about general svn usage and how to further manage your writing project with svn.  Also note our suggestions for general, one-time svn configuration in Section~\ref{sec:svn-install}.

\subsection{Initializing a Repository}

To create a working copy of an existing svn repository your system administrator will tell you the URL where the repository is hosted.  Then, you will \textit{check out} a working copy in a directory, say \Code{\$PROJECT} with that URL, which we call \Code{\$REPOSITORY\_URL}.  From the directory where you want \Code{\$PROJECT} to reside, call:
\begin{CodeVerbatim}
$> svn checkout $REPOSITORY_URL $PROJECT
\end{CodeVerbatim}

If this a new writing project, you may want to perform some initializations.  For example, decide what the final build products in your project will be.  These should be ignored by svn so as not to complain every time you recompile your LaTeX project.  Let's assume your project will create a file called ``proposal.pdf,'' then perform the following.  First, we check whether there are already files ignored.  Then, we will set a property to ignore ``proposal.pdf'' using a few bash commands.  If you are running a different shell, you may have to adjust these commands.
\begin{CodeVerbatim}[commandchars=\|\{\}]
$> cd $PROJECT/
$> svn propget svn:ignore .
$> svn propedit svn:ignore .  # will open temporary editor in your terminal, \ 
                                where you type |textbf{proposal.pdf}, save and exit
Set new value for property 'svn:ignore' on '.'
\end{CodeVerbatim}

Setting or updating a property puts a modification flag on the current directory \Code{.}, which you will have to commit to the repository at the next opportunity for others to obtain this setting.  Also, check that the property is now active in your working copy.
\begin{CodeVerbatim}
$> svn status
 M      .
$> svn commit -m "ignoring build product proposal.pdf"
Sending        .
Committed revision XXX.
$> svn propget svn:ignore .
proposal.pdf
\end{CodeVerbatim}

\subsection{Other Typical Subversion Commands}

If you have a new \Code{FILE.tex} file to add to the repository, do
\begin{CodeVerbatim}
$> $ svn st
?       FILE.tex
$> svn add FILE.tex
A         FILE.tex
$> svn commit -m "adding first version of FILE.tex"
Adding         FILE.tex
Transmitting file data ...
Committed revision XXX.
\end{CodeVerbatim}

When editing a \Code{FILE.tex} file and saving it, it will have the modification flag set, which you can check using the \Code{status} command.  It is also a good idea to update your working copy before you start editing a file, in case others have committed any changes.
\begin{CodeVerbatim}
$> svn update
[...]  # any potential updates
At revision XXX.
$> svn status
M       FILE.tex
$> svn commit -m "<message about recent edits in FILE.tex>"
Sending        FILE.tex
Transmitting file data ...
Committed revision XXX.
\end{CodeVerbatim}

If you decide on more build products (e.g., files called ``proposal-vol1.pdf'' and ``proposal-vol2.pdf'') in the future, call \Code{svn propedit svn:ignore .} to edit the property and commit the changes to the svn repository.  Make sure to do this before using a command such as \Code{svn add *}, which would mark any existing build products for addition.

With a centralized repository, it is even more important to coordinate writing and editing activities among collaborators.  Many say ``commit early, commit often'' and also make it a habit to update your working copy regularly and before beginning work.  LaTeX is very well suited to be managed under version control as you can split the writing document into various files and then assign writing tasks on a one-author-per-file-at-a-time to avoid merge conflicts.

% !TEX root = manual.tex
\section{General Usage} \label{sec:general-use}
\newcommand{\generalscale}{0.9}  % scale for figures in the general section

In this section we describe abstractly how one would use LTC as a pattern of a work cycle. See the tutorials for more concrete examples.  Also, the Sections~\ref{sec:emacs} and \ref{sec:java} below contain more details on the specific user interface of interest, namely Emacs and the LTC Editor, respectively.

Typically, more than one author collaborate on a writing project that is kept under version control but it might be a good practice to put all your work under version control.  Especially git is a well suited version control system to run locally on your computer and keeping track of your own changes if you are just interested in how a .tex file evolves over time.

We are assuming that the .tex file or files of interest are kept under version control so as to obtain a history of significant changes that have been made in the past.  Significant changes are usually made through a ``commit'' action to the version control system.  This is in contrast to merely saving edits to the file on the local file system.  Such an operation can be done many more times just to preserve your current work in case of a problem with the editor or computer.

\begin{figure}[t]
\centering
  \subfloat{
    \mygraphics{width=0.48\linewidth}{figures/work-cycle}}
  \hfill %space{1em}
  \subfloat{
    \mygraphics{width=0.48\linewidth}{figures/work-cycle-with-LTC}}
\caption{A typical work cycle for a version controlled file and when using LaTeX Track Changes} \label{fig:work-cycle}
\label{fig:editor-condense-before-after}
\end{figure}

See Figure~\ref{fig:work-cycle} on the left hand side for a diagram that shows a typical work cycle for a version controlled file from the perspective of one author.  Often a user starts working by downloading changes that others have done---this step may be omitted if only one author is working with the revision control system, thus the action is drawn with dashed lines.  Then, an author may edit and save the file.  Finally, when significant changes have been made, it is often time to commit those and possibly upload them to a server where other authors can update from.

Now look at the right hand side of Figure~\ref{fig:work-cycle}; here we added a state for tracking changes.  The user typically switches from editing and saving into tracking changes.  In this mode, one can still edit and save the file.  And also perform version control commands such as downloading and uploading changes.  While in track changes mode, the .tex file of interest is marked up with information about changes in past versions of the file, so the text looks busier and can be longer when displaying deletions.  Thus, most authors will want to switch in and out of tracking changes in order to work at times with only the latest version of the file to avoid being overwhelmed by the information shown.

\subsection{Filtering What is Shown}

show/hide

heuristic of ``small'' changes

\subsection{History of a File}

git -- directed, acyclic graph with branches
svn -- often sequential history, so only a straight line

LTC currently chooses one path from the first to the latest version of the file, traversing branches in the order of the most recent commit of the last commits before merge.  In the future: user can select

condensing authors



\section{Using \texttt{ltc-mode}}

\section{Using the ``LTC Editor''} \label{sec:java}

The LTC Editor is a Java application that allows to use LTC without a separate LTC server.  It may also serve as a reference implementation showing how to use our API.  Chapter~\ref{ch:plugins} contains more details on writing your own editor plugin using our LTC API.




% !TEX root = manual.tex
\chapter{Writing Your Own Front End} \label{ch:plugins}

In this chapter, we explain the general flow that a new editor plugin should follow.  We follow these flows in general in our reference implementation of the Emacs \Code{ltc-mode} and the Java LTC Editor.

\begin{figure}
\centering
\mygraphics{scale=0.85}{figures/plugin/flow-LTC-on-off}
\caption{Flow for turning LTC on and off} \label{fig:flow-LTC-on-off}
\end{figure}

\begin{figure}
\centering
\mygraphics{scale=0.85}{figures/plugin/flow-LTC-edit}
\caption{Flow for editing while viewing changes} \label{fig:flow-LTC-edit}
\end{figure}

\begin{figure}
\centering
\mygraphics{scale=0.85}{figures/plugin/flow-LTC-save-commit}
\caption{Flow when saving or committing while LTC is on} \label{fig:flow-LTC-save-commit}
\end{figure}


% !TEX root = manual.tex
\chapter{Algorithms \& Utilities}

\section{How Are Changes Calculated?} \label{sec:algorithms}

[or is this too technical for the manual?]

\section{Utility Programs} \label{sec:utils}

\subsection{LTC Editor}

\subsection{LTC File Viewer}

\subsection{LatexDiff}

\subsection{Lexicographical Analysis}
Lexer

\subsection{Testing the XML-RPC Server}

HelloLTC


\appendix
\chapter{License} \label{ch:license}
\VerbatimInput[fontsize=\footnotesize]{../../../../LICENSE}

\end{document}     