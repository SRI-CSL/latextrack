\documentclass[twoside]{report}

% ----- useful packages
\usepackage[letterpaper,verbose,pdftex]{geometry}
\usepackage{graphicx} % for including figures
%\usepackage{pgf,pgfarrows} % for portable graphics
%\usepackage{amsmath} % align and other math constructs
%\usepackage{amssymb} % math symbols
%\usepackage{booktabs} % tables in publication quality
\usepackage{fancyvrb,courier} % for more elaborate verbatim envs 
                              % (and bold-face fonts)
\usepackage[parfill]{parskip} % paragraphs begin with empty line rather than indent
\usepackage{url} % for typesetting URL's
\usepackage{color}
\usepackage[%pdftex, % driver
  hyperfootnotes=false, % if using "footnotes" package
  colorlinks=false, % don't color links
  bookmarks=true,bookmarksnumbered=true, % show numbered bookmarks
  pdfsubject={},pdftitle={},pdfauthor={Linda Briesemeister} % document info
]{hyperref} % more hyperlinks in PDF
%\usepackage{lipsum} % for test text

% --- version string
\newcommand{\version}{}
\input{version.txt}

% ----- title information
\title{The \LaTeX{} Track Changes Manual \version}
\author{%
Linda Briesemeister\\
\texttt{linda.briesemeister@sri.com}\\
SRI International
}
\date{\today} 

\begin{document}
\maketitle
{\pagestyle{empty}\cleardoublepage} % empty page after title (since we are using 'twopage' option)

\tableofcontents

\chapter*{Acknowledgements and Disclaimer}

This project would not exist without Peter Karp, who had the original idea to bring track changes to the LaTeX world.  My colleagues Grit Denker and Tomer Altman have also been involved in furthering this project.  Finally, we thank SRI International to provide the funding to pursue this project.

%TODO: Disclaimer

% !TEX root = manual.tex
\chapter{Installation \& Configuration} \label{ch:install}

\section{Requirements}

We have tested the system on Mac OS X and Linux. It is designed to run on Unix platforms. Windows is not supported although it may run with a Unix-like shell under Windows.

Further requirements for running LTC are as follows.
\begin{description}
\item[Java] 1.6 or above.  If using the \Code{jar} tool to extract Emacs Lisp files, you will need the JDK, otherwise the JRE should suffice.
\item[Version Control System]  One of the version control systems below:
  \begin{description}
  \item[git] 1.7.2 or above.
  \item[svn] 1.6.3 or above.
  \end{description}
\item[Emacs] 23 or above.  Only if the user wants to use \Code{ltc-mode} in Emacs to interact with LTC.  XEmacs is not supported.
\end{description}

\section{Installation and Update}

Currently, we provide a shell script \Code{ltc-install.sh} to perform LTC installation and updates.  It can be downloaded from \url{http://sourceforge.net/projects/latextrack/files/}.  Use it to install the LTC JAR file in a location of your choice.  Also, if you use the Emacs \Code{ltc-mode}, be prepared to supply the location where to put the Emacs Lisp files (see more details below in Section~\ref{sec:config-emacs}).

In the future, we may provide installers for the target platforms.

Let us assume for the remainder of this manual, that you have used a directory called \Code{\$LTC} as the installation location. Then, you would install LTC using the script in the following way.  Note that you will need a second argument to install LTC for Emacs as explained in more detail below (Section~\ref{sec:config-emacs}).  It doesn't hurt to run the script twice, as it will replace LTC with the latest version.

\begin{CodeVerbatim}
$> bash ltc-install.sh -h
[... prints help message about using LTC install script]
$> bash ltc-install.sh $LTC
[...]
Done with installing LTC in $LTC
To start LTC server with default options, use the following command:

  java -jar $LTC/LTC.jar

\end{CodeVerbatim}

To update from an earlier LTC version, run the \Code{ltc-install.sh} script again with the same argument(s).  The script will download the latest version from the web site and configure the link in \Code{\$LTC/LTC.jar} so that future invocations  will resolve to the newest version.

After installing LTC, you can look at the command line options of LTC Server using the switch \Code{-h} or omit the switch to start LTC Server with default values.
\begin{CodeVerbatim}
$> java -jar $LTC/LTC.jar -h
[... prints help message about using LTC Server]
\end{CodeVerbatim}

\section{Configuration}

This section contains details of configuring git or svn, LTC and Emacs to work together.  These steps typically only need to be carried out once per installation of LTC.

Note that the system decides automatically whether your LaTeX file is under git or svn version control.

\subsection{General Git Configuration}

The current version of LTC supports git as an underlying version control system, which manages the history of the LaTeX files of interest. If you are already using git for other things, you may skip the following few steps as your git is probably already configured. However, we do recommend to add the common LaTeX build products with wildcards to the list of ignored files as outlined at the end of this section, which may not be configured if git has not been used to manage repositories with LaTeX files.

First, test that the git installation is found. Otherwise, you may want to add the git binary directory to the \Code{PATH} environment variable.  
\begin{CodeVerbatim}
$> git --version
git version 1.7.2
\end{CodeVerbatim}

If you haven't done already, configure git with your name and email address:
\begin{CodeVerbatim}
$> git config --global user.name "John Doe"
$> git config --global user.email doe@inter.net
\end{CodeVerbatim}

Typically, you don't want to track automatic backups and build products of your LaTeX project, so create a file \Code{\char`\~/.gitignore\_global} (or any name and location of your choice) and add the following lines as contents. 
\begin{CodeVerbatim}
*~
*.out
*.aux
*.bbl
*.blg
*.bst
*.dvi
*.idx
*.lof
*.log
*.toc
*.lol
*.lot
\end{CodeVerbatim}
Then, issue the git config command below (with a possibly adjusted file name and location).

\begin{CodeVerbatim}
$> git config --global core.excludesfile ~/.gitignore_global
\end{CodeVerbatim}

To learn how to set up a writing project under a git repository for using it with LTC refer to Section~\ref{sec:git-use}.

\subsection{General Subversion Configuration}

First, test that the svn installation is found. Otherwise, you may want to add the svn binary directory to the \Code{PATH} environment variable.  
\begin{CodeVerbatim}
$> svn --version
svn, version 1.6.18 (r1303927)
   compiled Aug  4 2012, 19:46:53

...
\end{CodeVerbatim}

[TODO: write more about subversion configuration here!]

%\subsection{LTC Server}

\subsection{Emacs \texttt{ltc-mode}} \label{sec:config-emacs}

To use the supplied \Code{ltc-mode} in Emacs, you will have to put the relevant mode files into a directory where Emacs can load them. There are two alternatives of letting Emacs know where to find Emacs Lisp files:
\begin{enumerate}
\item Use a location that is already included in the \Code{load-path}. To view the contents of this path in your Emacs, execute the command \Code{C-h v load-path}.  On Mac OS X systems with Aquamacs, this could be for example \Code{\char`\~/Library/Preferences/Emacs}.
\item Add a new directory where you will extract the Emacs Lisp files to the \Code{load-path}. Assuming the Emacs Lisp files will be installed in directory \Code{\char`\~/LTC/emacs/}, add the following line to your \Code{.emacs} or other Emacs configuration file:
  \begin{CodeVerbatim}
(add-to-list 'load-path "~/LTC/emacs")
  \end{CodeVerbatim}
\end{enumerate}

Now based on which method of the above you choose, supply the directory \Code{\$EMACS\_DIR} as the second argument to the install script:
\begin{CodeVerbatim}
$> bash ltc-install.sh $LTC $EMACS_DIR
\end{CodeVerbatim}

In order to enable the LTC mode in Emacs, add the following line to your \Code{.emacs} or other Emacs configuration file (for example, \Code{\char`\~/Library/Preferences/Emacs/Preferences.el} is the default for Aquamacs under Mac OS X) :
\begin{CodeVerbatim}
(autoload 'ltc-mode "ltc-mode" "" t)
\end{CodeVerbatim}

We recommend to avoid adding a hook from \Code{latex-mode} to \Code{ltc-mode} (usually done with \Code{add-hook}) as our mode requires the \Code{latex-mode} to be fully executed before it works. The hooks are not guaranteed to be executed in particular order, so it is best to manually invoke LTC mode.

\begin{figure}[t]
\centering
\mygraphics{width=\textwidth,height=.5\textheight,keepaspectratio}{figures/emacs-port}
\caption{Customizing LTC port number in Emacs} \label{fig:emacs-port}
\end{figure}

If you need to change the port number that Emacs uses to communicate with the LTC Server (for example, if the default number is already in use on your computer), you first have to load \Code{ltc-mode} at least once (possibly with a failure) using command \Code{M-x ltc-mode}.  Then, you can view the current port setting using \Code{C-h v ltc-port <RET>}.  You can customize the port number using \Code{M-x customize-group <RET> ltc <RET>} or open the customization buffer and browse to the LTC group under the Tex group, which may be located under the Wp (word processing) top-level group.  See Figure~\ref{fig:emacs-port} for a screenshot when customizing the port number using the customization buffer in Aquamacs under Mac OS X.

%\subsubsection{Git from Emacs}
%
%The user can interact with git from within Emacs.  One of the many implementations is called ``magit'' mode.

\section{Troubleshooting}

We are keeping a list of \hlink{\baseurl /faq.html}{../../faq.html}{frequently-asked-questions} at the project's web site that may help for troubleshooting.  There will also be mailing lists and a way to report bugs available in the future.


\chapter{Tutorial} \label{ch:tutorial}



\chapter{Using LTC}

The intended use of LTC is running the base system as a server and have a plugin of the user's preferred LaTeX editor connect to it.  The provided Emacs \texttt{ltc-mode} is an example of such a plugin and its use is described in Section~\ref{sec:emacs}.  We hope to add more plugins for other popular LaTeX editors in the future.

As a minimum implementation of a user interface to LTC, we also provide a Java application ``LTC Editor'' that uses the base system API but does not rely on running a separate LTC server.  Its use is explained in Section~\ref{sec:java}.

\section{Using Emacs} \label{sec:emacs}

In this section, we explain how to use Emacs as the editor of the .tex-file with LTC.

\begin{center}
\fcolorbox{red}{yellow}{
\begin{minipage}[t]{.9\textwidth}
  \textbf{CAVEAT:} due to current bugs, it is best to use \Code{ltc-mode} defensively without editing the file in Emacs while LTC is turned on.  It is safe to view changes in the file and use different filtering settings.  If the text becomes scrambled, try to exit \Code{ltc-mode} by toggling it (e.g., using command \Code{M-x ltc-mode}) or quitting Emacs altogether before saving the file.  As long as you do not save while in \Code{ltc-mode} nothing bad should happen.
\end{minipage}}
\end{center}

\subsection{Starting the LTC Server}

Before you can use \Code{ltc-mode} in Emacs, you must start the LTC server first.  To do so, you call Java with the JAR-file that you downloaded. 

\begin{CodeVerbatim}
$> java -jar $LTC/LTC-${project.version}.jar
 | CONFIG: 	Default logging configuration complete
 | CONFIG: 	Logging configured to level CONFIG
 | CONFIG: 	git version: X.Y.Z
--log4j: [main] INFO  org.apache.xmlrpc.webserver.XmlRpcServlet - init
 | INFO: 	Started RPC server on port 7777.
\end{CodeVerbatim}

If you need to customize LTC, for example to change the port number, start the base system with option \Code{-p <PORT>}:

\begin{CodeVerbatim}
$> java -jar $LTC/LTC-${project.version}.jar -p 5555
[...]
 | INFO: 	Started RPC server on port 5555.
\end{CodeVerbatim}

The log messages from the LTC server are also saved in file \Code{~/.LTC.log}.  This log file is created every time that the server starts, so if you need the contents, please make a copy before starting the server anew.

\subsection{Entering and Exiting \texttt{ltc-mode}}

\begin{figure}[t]
\centering
\mygraphics{height=.37\textheight}{figures/emacs-latex-mode}
\hspace{2ex}
\begin{tikzpicture}
  \node [single arrow,draw] {};
\end{tikzpicture}
\hspace{2ex}
\mygraphics{height=.37\textheight}{figures/emacs-ltc-started}
\caption{Starting \texttt{ltc-mode} in Emacs} \label{fig:emacs-ltc-started}
\end{figure}

Once your Emacs has a .tex-file loaded in plain \Code{latex-mode} as seen on the left in Figure~\ref{fig:emacs-ltc-started}, you can invoke the minor-mode \Code{ltc-mode} in different ways.  Usually, the mode launching command \Code{M-x ltc-mode} is available.  If Emacs has its own window and supports context menus, you may be able to right-click on the word ``LaTeX'' in the mode line at the bottom of Emacs. Then, a list of minor modes appears from which you can select ``LTC Mode.''  Once the mode starts successfully, a number of status messages, a smaller window at the bottom of the text that contains LTC information, and changes in the text marked up by color and style appear.  Emacs then looks similar to the right in Figure~\ref{fig:emacs-ltc-started}.

The mode can be toggled, i.e., the same command starts and stops \Code{ltc-mode}.

Once the mode is started, a new menu ``LTC'' should appear in Emacs.  This can be part of Emacs' main menu or included in the mode line.  From there, you can interact with the LTC system as shown in the following examples.

\subsubsection{The ``LTC info'' Buffer}

The new, smaller window at the bottom of the frame is called ``LTC info'' and contains a table resembling the commit graph of the current file.  Newer versions of the file are at the top of the list.  An asterisk at the beginning  indicated that this version is taken into account when the history of changes is calculated.  Conversely, if the asterisk is missing and the whole row is shown in gray color, then this version is currently skipped when calculating changes.  The next column contains the SHA-1 key (abbreviated) and the third column the date  of the corresponding git revision.  The author colored in his or her key and finally the commit message follow.  The first row of the commit table contains the current author.

\subsubsection{Updating the Buffer}

An important command is \Code{M-x ltc-update} (currently bound to Ctrl-L U but this will change once we adhere better to Emacs conventions) or the menu item LTC $>$ Update buffer.  This will update the  view of the file based on current filtering and other settings.  Use this command when the ``LTC info'' buffer ended up in a different frame or tab, or when you make changes such as a git commit from the command line.

\subsection{Filtering Changes}

\begin{figure}[t]
\centering
\mygraphics{height=.37\textheight}{figures/emacs-hide-preamble}
\caption{Hiding changes in the preamble using the ``LTC'' menu} \label{fig:emacs-hide-preamble}
\end{figure}

To show and hide certain changes, simply choose from the menu LTC $>$ Show/Hide $>$ ... the appropriate entry.  Figure~\ref{fig:emacs-hide-preamble} shows for example hiding changes in the preamble via the context menu.  %If the menu is not available, the Emacs commands \Code{M-x } have the same effect.
After selecting any of the menu items to show or hide certain changes, the contents in the main Emacs window is immediately updated to reflect the filtering settings.

To change the default limitation of the file's history that is taken into account for viewing changes, you can do so by specifying a set of authors to limit to, and a revision number or date to indicate how far back in time you want to go.  These functions are available via the menu LTC $>$ Limit by $>$ ...  Each of these actions require additional input, therefore the menu item ends with ellipsis. 

To define a set of authors to limit the history to, select the menu item or use command \Code{M-x ltc-limit-authors}. This will prompt you to input the names of authors in the mini-buffer.  You may use auto-completion with the TAB key.  End the input of names by providing an empty name.  If the first name is empty, this will reset to taking all known authors into account (i.e., not limiting by authors anymore).

Similarly, you can define how far back the file history is taken into account.  Use commands \Code{M-x ltc-limit-rev} 
and \Code{M-x ltc-limit-date} respectively.  Again, user input is needed via the mini-buffer.  Use auto completion with the TAB key in order to automatically extend to known SHA-1 keys or revision dates.  To reset the limit to the default, simply enter an empty value when prompted.

\subsection{Author Color Keys}

To customize the color used to designate changes in the marked up text, click on the author name in the ``LTC info'' buffer. [TODO: MORE NEEDED]

\subsection{Emacs Help}

[TODO]

\subsection{Table of Commands}

[TODO]
\section{Using the ``LTC Editor''} \label{sec:java}

The LTC Editor is a Java application that allows to use LTC without a separate LTC server.  It may also serve as a reference implementation showing how to use our API.  Chapter~\ref{ch:plugins} contains more details on writing your own editor plugin using our LTC API.



% !TEX root = manual.tex
\chapter{Writing Your Own Front End} \label{ch:plugins}

In this chapter, we explain the general flow that a new editor plugin should follow.  We follow these flows in general in our reference implementation of the Emacs \Code{ltc-mode} and the Java LTC Editor.

\begin{figure}
\centering
\mygraphics{scale=0.85}{figures/plugin/flow-LTC-on-off}
\caption{Flow for turning LTC on and off} \label{fig:flow-LTC-on-off}
\end{figure}

\begin{figure}
\centering
\mygraphics{scale=0.85}{figures/plugin/flow-LTC-edit}
\caption{Flow for editing while viewing changes} \label{fig:flow-LTC-edit}
\end{figure}

\begin{figure}
\centering
\mygraphics{scale=0.85}{figures/plugin/flow-LTC-save-commit}
\caption{Flow when saving or committing while LTC is on} \label{fig:flow-LTC-save-commit}
\end{figure}


% ----- references 
%
\bibliographystyle{plain}
\bibliography{ltc}
%\addcontentsline{toc}{chapter}{Bibliography} 

\end{document}     