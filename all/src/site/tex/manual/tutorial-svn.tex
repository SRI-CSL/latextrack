% !TEX root = manual.tex
\section{Tutorial with Subversion and LTC Editor} \label{sec:tutorial-svn}

In this section, we assume that the example svn repository has been created according to the instructions in Section~\ref{sec:example-svn} above.  The latter subsections require a local svn repository but the beginning can be done with either the remote example repository or a local one.  

\subsection{Running the LTC Editor}

First, we start the LTC Editor to interact with LTC and track the changes of the file.  Assuming you have installed LTC in the directory \Code{\$LTC}, we can look at the command line options of the editor:
\begin{CodeVerbatim}[commandchars=\\\{\}]
$> java -cp $LTC/LTC.jar com.sri.ltc.editor.LTCEditor -h
LaTeX Track Changes (LTC)  Copyright (C) 2009-2013  SRI International
This program comes with ABSOLUTELY NO WARRANTY; for details use command line switch -c.
This is free software, and you are welcome to redistribute it under certain conditions.

usage: java -cp ... com.sri.ltc.editor.LTCEditor [options...] [FILE] 
with
 FILE     : load given file to track changes
 -c       : display copyright/license information and exit
 -h       : display usage and exit
 -l LEVEL : set console log level
            SEVERE, WARNING, INFO, CONFIG (default), FINE, FINER, FINEST
 -r       : reset to default settings
\end{CodeVerbatim}

To open our tutorial file at \Code{\$TUTORIAL/independence/independence.tex} when starting the editor, execute the following command.  After the editor opens as a window, we open the combo box under the name for ``Self:'' by clicking on the small arrow right next to the name and select the author ``adams'' to impersonate John Adams for the remainder of this tutorial.  This will again contact the server to obtain updates about the file history, which may take a little time if you are working with a remote repository.  Finally, the window should look similar to the screen shot in Figure~\ref{fig:svn-editor-open}.
\begin{CodeVerbatim}[samepage=true,commandchars=\\\{\}]
$> java -cp $LTC/LTC.jar \textbackslash
   com.sri.ltc.editor.LTCEditor $TUTORIAL/independence/independence.tex
\end{CodeVerbatim}
\begin{figure}[t]
\centering
\mygraphics{width=\textwidth,height=.5\textheight,keepaspectratio}{figures/svn-editor-open}
\caption{Initial opening of tutorial file under svn in LTC Editor} \label{fig:svn-editor-open}
\end{figure}
In this figure, we see a panel at the bottom-right that displays the history of the current LaTeX file under svn as well as the name of the current user -- now John Adams because we had overridden the default, which was your user name.

\subsection{Showing and Hiding Certain Changes}

The bottom-left panels of the editor allows us to customize the way LTC displays the changes of the file.  Section~\ref{sec:general-use} contains all the details of how LTC displays the changes including limiting the file history and filtering.  In this tutorial, we will just use some of the options and see their effect.

First, notice the colors assigned to each of the authors.  To change an author color, for example Roger Sherman's,  perform a double-click on the colored square next to ``sherman'' in the list of authors to open a dialog and  choose a dark color such as brown (you will want something with contrast to the white background).  Notice, after the updates have come through from the server, how the text in the editor panel on the top changes color for those parts that are attributed to Roger Sherman's edits.

Next, focus on the typographical errors in the command ``\textbackslash maketitle'' in line 11 and the beginning of the first paragraph in line thirteen as well as the spelling errors in the word ``political.''  If you first uncheck the box for ``small'' changes and second, also the box for deletions, notice how the text rendering in the editor panel changes.
\begin{figure}[t]
  \centering
  \subfloat{
    \label{subfig:svn-editor-filter-small1} 
    \mygraphics{width=.25\textwidth}{figures/svn-editor-filter-small1}}
  \hspace{2em}
  \subfloat{
    \label{subfig:svn-editor-filter-small2} 
    \mygraphics{width=.25\textwidth}{figures/svn-editor-filter-small2}}
  \hspace{2em}
  \subfloat{
    \label{subfig:svn-editor-filter-small3} 
    \mygraphics{width=.25\textwidth}{figures/svn-editor-filter-small3}}
\caption[Effect of hiding ``small'' changes and deletions]{Effect of hiding ``small'' changes first (middle) and then also deletions (right)} \label{fig:svn-editor-filter-small}
\end{figure}
Figures~\ref{fig:svn-editor-filter-small} show that ``\textbackslash maketitle'' as well as the typos in the word ``political'' are no longer marked up, and in the third image, the deletion beginning with ``If'' at the beginning of the paragraph is now omitted.

\subsection{Understanding the Commit Graph}

Now draw your attention back to the graph with the history of the current file under svn (located in the bottom-right panel).  In our current tutorial repository, there are no branches and the graph is a sequential line.  
\begin{figure}[t]
\centering
\mygraphics{scale=.5}{figures/svn-commit-graph}
\caption{Example of commit graph} \label{fig:svn-commit-graph}
\end{figure}
Refer to Figure~\ref{fig:svn-commit-graph} for a screen shot of the example file history. Versions that are included in the tracked changes are not printed in gray.  How far we go back in history depends on some filtering settings, which are discussed further in Section~\ref{sec:svn-limit-history} below.  By default, we first include all version of the current author at the top.  In our example with impersonating John Adams with the user name ``adams,'' there are currently no further recent commits of him.  Then, we continue down the path and collect all versions of different authors until we find the next version of John Adams in the commit with the message ``second version.''

\subsection{Limiting History} \label{sec:svn-limit-history}

We allow the user to filter and customize how the potentially rich history of a .tex file is selected, so as to provide a better view of the tracked changes.  The user can show and hide changes as seen above, limit the authors of interest, and specify a date or revision number to tell LTC how far back in time the history should be considered.

\begin{figure}
\centering
  % first sub-figure
  \begin{minipage}[t]{0.35\linewidth}
  \centering
  \mygraphics{scale=.5}{figures/svn-editor-select-authors}
  \caption{Selecting authors for filtering} \label{fig:svn-editor-select-authors}
  \end{minipage}%
\hspace{0.04\linewidth}%
  % second sub-figure
  \begin{minipage}[t]{0.61\linewidth}
  \centering
  \mygraphics{scale=.5}{figures/svn-editor-limit-authors}
  \caption[Effect of limiting authors]{Effect of limiting authors to to ``sherman'' and ``jefferson'' after clicking ``Update''} \label{fig:svn-editor-limit-authors}
  \end{minipage}  
\end{figure}
To limit the history by \textbf{authors}, select both authors ``sherman'' and ``jefferson'' through clicking while holding down the CTRL or CMD key in the list of authors in the middle lower panel.  Then, click the button ``Limit'' below the list, which will gray out the unselected authors.  For a limiting action to take effect, you need to click ``Update.''  This is different from showing and hiding various changes as well as changing author colors, which is applied instantly.

Notice how any version by the ignored authors Benjamin Franklin and John Adams is now gray as only commits from the selected authors are considered.  Again, the history is only taken until the next revision of the current author but since he is being ignored, we go all the way back to the first revision. Compare your editor window with the screen shot in Figure~\ref{fig:svn-editor-limit-authors} and see how the commit graph has changed.

Then, clicking the ``Reset'' button followed by ``Update'' will remove and limits on the history by author, so the original view is restored.

Next, we apply limits on \textbf{revision} and \textbf{date} to control how far back the history of the file is considered.  As we had seen, the first version is not taken into account because it was committed before the next commit by the current author John Adams.  Let us now type \Code{1} as the revision number of the first commit into the field labeled ``Start at revision:'' (refer to Figure~\ref{fig:svn-editor-select-rev}) or simply drag the number from the entry in the commit graph into the field.  Now click ``Update'' and see how the first version is listed in black and considered in the tracked changes above as seen in Figure~\ref{fig:svn-editor-limit-rev}.  Since changes by the current author John Adams from the first to the second version are now included, notice the text marked up in red appearing in the editor window. We see that John Adams must have added himself as an author in the LaTeX preamble among other edits.

\begin{figure}
\centering
  % first sub-figure
  \begin{minipage}[t]{0.35\linewidth}
  \centering
  \mygraphics{scale=.5}{figures/svn-editor-select-rev}
  \caption{Selecting revision for filtering} \label{fig:svn-editor-select-rev}
  \end{minipage}%
\hspace{0.04\linewidth}%
  % second sub-figure
  \begin{minipage}[t]{0.61\linewidth}
  \centering
  \mygraphics{scale=.5}{figures/svn-editor-limit-rev}
  \caption[Effect of going back to first revision]{Effect of going back to first revision after clicking ``Update''} \label{fig:svn-editor-limit-rev}
  \end{minipage}  
\end{figure}

To remove the limit by revision number, simply erase the text in the field ``Start at revision:'' and click ``Update'' again.

\begin{figure}
\centering
  % first sub-figure
  \begin{minipage}[t]{0.35\linewidth}
  \centering
  \mygraphics{scale=.5}{figures/svn-editor-select-date}
  \caption{Selecting date for filtering} \label{fig:svn-editor-select-date}
  \end{minipage}%
\hspace{0.04\linewidth}%
  % second sub-figure
  \begin{minipage}[t]{0.61\linewidth}
  \centering
  \mygraphics{scale=.5}{figures/svn-editor-limit-date}
  \caption[Effect of limiting history to date of third version]{Effect of limiting history to date of third version after clicking ``Update''} \label{fig:svn-editor-limit-date}
  \end{minipage}  
\end{figure}

Limiting the history by date works similarly.  You may drag a date from the commit graph on the right, for example the date of the third version commit, and drop it into the field ``Start at date:'' on the left.  Or, type a date such as \Code{Nov 13, 2012 12:59p} into the field.  We employ some software to process times and dates in natural language, and if successful, the field will contain the date string as it was understood translated into the format used in the commit tree.
Again, you will need to click ``Update'' for the change to take effect or click RETURN while editing the text field.  See Figures~\ref{fig:svn-editor-select-date} and \ref{fig:svn-editor-limit-date} for screenshots. To remove the limit by date, erase all text in the text field and update.

\subsection{Condensing History}

Sometimes the list of commits considered is getting long and the resulting markup of the changes confusing.  One additional way to customize how the changes are displayed is a setting to ``condense authors.''  Find a check box with that name under the list of authors for filtering.  If checked, then only the latest version of an author of \textit{consecutive} commits is considered -- in our example, only the sixth version is shown in black while the fifth version by Roger Sherman is now grayed out as seen in Figure~\ref{fig:svn-editor-condense-on}.
\begin{figure}
\centering
  \begin{minipage}[b]{0.57\linewidth}
    \centering
    \mygraphics{scale=.5}{figures/svn-editor-condense-on}
    \ifpdf
      \caption[Effect of condensing authors]{Effect of condensing authors:\\ ignoring the fifth version by Roger Sherman} 
    \else
      \caption[Effect of condensing authors]{Effect of condensing authors: ignoring the fifth version by Roger Sherman} 
    \fi
    \label{fig:svn-editor-condense-on}  
  \end{minipage}  
\hspace{0.04\linewidth}%
  \begin{minipage}[b]{0.38\linewidth}
    \subfloat{
      \label{subfig:svn-editor-condense-before} 
      \mygraphics{width=0.8\linewidth}{figures/svn-editor-condense-before}}
  \hspace{1em}
    \subfloat{
      \label{subfig:svn-editor-condense-after} 
      \mygraphics{width=0.8\linewidth}{figures/svn-editor-condense-after}}
    \caption[Example of condensing authors]{Example of markup change before (left) and after (right) condensing authors} \label{fig:svn-editor-condense-before-after}
  \end{minipage}%
\end{figure}

See Figure~\ref{fig:svn-editor-condense-before-after} for an example of how condensing authors affects the markup.  Since we are only considering the changes that Roger Sherman made in the sixth version, his correction of the name is no longer shown.  Condensing authors makes sense when users commit versions often so that they do not loose too much history.  Their last version after a number of commits generally has the flavor of a ``final'' version, ready to be shared with others.  Hence, the changes made there compared to the last version of another author is commonly of most interest.

\subsection{Editing and Saving} %\label{sec:svn-editor-editing}

Let us start the next step by resetting all filters to the default configuration, i.e., no limit by authors, date, and revision.  Then, we will edit the text in the editor panel to see the latest changes.

Click into the text panel and enter some text, for example a LaTeX comment reminding John Adams to work on a list of charges against King George III in line nineteen:
\begin{FileVerbatim}
% list charges against King George III here
\end{FileVerbatim}
The added text will be rendered in red (or the color code for the current author) and underlined.  Notice how the commit graph adds a first line with the label ``modified'' and the ``Save'' button becomes enabled.  Now delete some of the characters you have just entered, for example the word \Code{here} at the end.  The characters simply disappear as they were added by the same author.

Now delete other characters that are either rendered black or a different color than red but not marked as deletions (strike-through font).  Notice how these characters remain visible but are now colored red and marked up with strike-through.  If you tried to delete anything that is already marked as deletion (i.e., anything in strike-through font), nothing will happen as this text is already deleted in a prior version.  See Figure~\ref{fig:svn-editor-modified} for a screen shot of the above edits: text in red and underlined was added and text in red and strike-through was deleted.

\begin{figure}[t]
\centering
\mygraphics{scale=.35}{figures/svn-editor-modified}
\caption{After editing the text as ``adams''} \label{fig:svn-editor-modified}
\end{figure}

Finally, you will want to click ``Save'' to save the current file to disk.  This will cause the label ``modified'' to change to ``on disk.''  If you would then again edit, the label would switch back to ``modified'' of course.

\subsection{Collaborating Through Commits}

In Subversion, your repository is a central entity that all collaborators commit to.  Therefore, unlike distributed version control systems such as git, the collaboration happens when users commit their version.  It is a good practice to update your working copy regularly to avoid conflicts during committing.  Furthermore, users should communicate with each other to decide who is editing what file at a time.

The following assumes that you are working with a local svn repository per the setup in Sections~\ref{sec:example-svn-local} and {sec:svn-collaborating} above.  Next, we will simulate that Roger Sherman commits his modified version before John Adams can upload his version, resulting in a merge conflict.  To prepare this scenario, first commit the modifications from Roger Sherman's working copy:
\begin{CodeVerbatim}[commandchars=\\\{\}]
$> cd $TUTORIAL/independence-sherman/
$> svn status
M       independence.tex
$> svn commit -m "todo item for indictment" --username sherman 
Sending        independence.tex
Transmitting file data .
Committed revision 7.
\end{CodeVerbatim}

Now the shared repository is already at revision 7 while we (as John Adams) are still editing from revision six.  First, check again that you have saved the file in LTC Editor using the ``Save'' button and that the first entry in the commit graph says ``on disk.'' When we try to commit our latest changes from the Section~\ref{sec:svn-editing} above, using the command line, we get the following error message.  
\begin{CodeVerbatim}[commandchars=\\\{\}]
$> cd $TUTORIAL/independence/
$> svn status
M       independence.tex
$> svn commit -m "added comment about list of charges" --username adams
Sending        independence.tex
Transmitting file data .svn: Commit failed (details follow):
svn: File '/tutorial-svn/independence.tex' is out of date
\end{CodeVerbatim}

Then, we try to update our repository first to mend the situation, which results in another error message about the conflicting versions.  If Roger Sherman and John Adams had been modifying different .tex files in the same repository, we would have not gotten this conflict.  To resolve, we choose to postpone so that we can see the differences in Emacs and resolve it there.  Our input is marked in bold below.
\begin{CodeVerbatim}[commandchars=\\\{\}]
$> svn update
Conflict discovered in 'independence.tex'.
Select: (p) postpone, (df) diff-full, (e) edit,
        (mc) mine-conflict, (tc) theirs-conflict,
        (s) show all options: \textbf{p}
C    independence.tex
Updated to revision 7.
Summary of conflicts:
  Text conflicts: 1
\end{CodeVerbatim}

\begin{figure}
\centering
\mygraphics{scale=.35}{figures/svn-editor-merge-conflict}
\caption{Subversion conflict markers in merged file} \label{fig:svn-editor-merge-conflict}
\end{figure}
Our current file is now marked as in conflict, so let us click the ``Update'' button in LTC Editor to see something similar to the screen shot in Figure~\ref{fig:svn-editor-merge-conflict}.  The conflicting portion at the end is marked with additional lines and we see revision 7 in the history of the file.  On the command line, the file looks similar to this:
\begin{CodeVerbatim}
$> cat independence.tex 
[...]

<<<<<<< .mine
% list charges against King George III

=======
That to secure these rights, Governments are instituted among Men, [...]
%TODO: indictment here

>>>>>>> .r7
\end{CodeVerbatim}

We decide that the comments in the version \Code{.mine} version means the same as the last comment in version \Code{.r7} so we modify the text in LTC Editor and save, so that it looks like this on the command line:
\begin{CodeVerbatim}
$> cat independence.tex 
[...]

That to secure these rights, Governments are instituted among Men, [...]

% list charges against King George III
\end{CodeVerbatim}
It is important to remove the svn marker lines starting with \Code{<<<<<<<}, \Code{=======}, and \Code{>>>>>>>} for svn to recognize that we have resolved the conflicts. We also have to tell svn that the conflict has been resolved.  Then we can finally  perform the following commit command.  The two important commands are printed in bold below.  
You may want to check the status of svn before and after the commit:
\begin{CodeVerbatim}[commandchars=\\\{\}]
$> \textbf{svn resolved independence.tex}
Resolved conflicted state of 'independence.tex'
$> svn status
M       independence.tex
$> \textbf{svn commit -m "added comment about list of charges" --username adams}
Sending        independence.tex
Transmitting file data .
Committed revision 8.
$> svn status
\end{CodeVerbatim}
\begin{figure}
\centering
\mygraphics{scale=.35}{figures/svn-editor-merge-resolve}
\caption{After resolving conflict and updating in LTC Editor} \label{fig:svn-editor-merge-resolve}
\end{figure}
Once we click the ``Update'' button in LTC Editor, we see the latest revision 8 in the commit graph and the marked up latest edits attributed to John Adams and Roger Sherman similar to Figure~\ref{fig:svn-editor-merge-resolve}.
